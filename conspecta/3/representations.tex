\chapter{Теория представлений конечных групп}


\newlection{12 октября 2023 г.}


\section{Теорема Жордана --- Гёльдера}
Рассмотрим ряд подмодулей $\{0\} = M_0 \le \dots \le M_m= M$.

Теорема Жордана --- Гёльдера говорит о том, что такой ряд в некотором смысле единственный.
Для начала, определим, в каком смысле понимать эту единственность.
Для этого рассмотрим второй ряд $\{0\} = N_0 \le \dots \le N_n = M$
\definition[Ряды $M_i$ и $N_j$ эквивалентны]{
    Равны наборы факторов соседних: $n = m$ и $\exists \sigma \in S_n: M_i /M_{i - 1} \cong N_{\sigma(i)}/N_{\sigma(i) - 1}$
}
\definition[Ряд без повторений]{
    $\forall i: M_{i - 1} \ne M_i$.
}
\definition[Простой модуль]{Модуль, в котором нет собственных подмодулей.}
\definition[Неуплотняемый ряд $M_i$]{
    Все факторы простые: $M_i \le K \le M_{i + 1} \iff \any{K = M_i \\ K = M_{i + 1}}$, то есть $M_{i + 1}/M_i$ прост.
}
\definition[Композиционный ряд]{Неуплотняемый ряд без повторений}
\definition[Артинов модуль]{
    Модуль, удовлетворяющий условию обрыва убывающих цепей (для подмодулей), DCC.
}
\definition[Нётеров модуль]{
    Модуль, удовлетворяющий условию обрыва возрастающих цепей (для подмодулей), ACC.
}
Согласно лемме Цорна в любом непустом наборе подмодулей артинова модуля есть минимальный элемент, в нётеровом модуле --- максимальный.
\proposal{
    В модуле $M$ есть композиционный ряд $\iff$ модуль $M$ и артинов, и нётеров.
    \provewthen{
        Построим композиционный ряд по индукции. $\{0\} = M_0 \lneq M_1$, где $M_1$ выбирается, как минимальный элемент (существует из-за артиновости) в множестве $\defset{N \le M}{M_0 \le N}$.
        Таким образом, строится цепочка $M_0 \lneq M_1 \lneq \dots \lneq M_n \lneq \cdots$.
        Данная цепочка за счёт нётеровости обрывается, то есть $\exists n \in \N: M_n = M$
    }{
        Пусть модуль не артинов или не нётеров.
        Тогда существует сколь угодно длинная цепочка из подмодулей.
        Но длина любого ряда не превосходит длины композиционного~(\ref{length}).
    }
}
\theorem{
    У любых двух рядов существует их общее уплотнение.
    \provehere{
        Пусть даны два ряда $\{0\} = M_0 \le M_1 \le \dots \le M_m = M$ и $\{0\} = N_0 \le N_1 \le \dots \le N_n = M$.

        Определим $M'_{i,j} = (M_i + N_j) \cap M_{i + 1}$ для $0 \le i < m, 0 \le j \le n$.

        Определим $N'_{j,i} = (N_j + M_i) \cap N_{j + 1}$ для $0 \le j < n, 0 \le i \le m$.
        Тогда
        \[\frac{M'_{i,j+1}}{M'_{i,j}} = \frac{(M_i + N_{j+1}) \cap M_{i + 1}}{(M_i + N_j) \cap M_{i + 1}} \overset?\cong \frac{(N_j + M_{i + 1}) \cap N_{j + 1}}{(N_j + M_i) \cap N_{j + 1}} = \frac{N'_{j,i + 1}}{N'_{j,i}}\]
        \indentlemma{\label{plus_is_not_distributive_by_cap}
        Если $A \le B$, то $(A + X) \cap B = A + X \cap B$.
        }{
            Так как $A \le (A + X) \cap B$, и $X \cap B \subset (A + X) \cap B$, то $(A + X \cap B) \le (A + X) \cap B$.

            В другую сторону, рассмотрим $a + x = b \in (A + X) \cap B$. Тогда $x = b - a$, то есть $x \in X \cap B$, и $a + x \in A + X \cap B$.
        }
        \indentlemma[О бабочке]{
            Пусть $A \le B \le M$ и $C \le D \le M$.
            Тогда утверждается, что
            \[\frac{(A + D) \cap B}{(A + C) \cap B} \cong \frac{B \cap D}{A \cap D + B \cap C} \cong\frac{(B + C) \cap D}{(A + C) \cap D}\]
        }{
            При замене $B \leftrightarrow D, A \leftrightarrow C$ среднее не меняется, а левое изменяется на правое.
            Значит, достаточно доказать первый знак изоморфности.

            \[\frac{A + D \cap B}{A + C \cap B} = \frac{(A + C \cap B) + D \cap B}{A + (C \cap B)} \underset{\text{теорема Нётер об изоморфизме}}\cong \frac{D \cap B}{(A + C \cap B) \cap D \cap B}\]
            Осталось показать, что $(A + C) \cap B \cap D = A \cap D + B \cap C$.
            Используя лемму~(\ref{plus_is_not_distributive_by_cap}), получаем $((A + C) \cap B) \cap D = (A + C \cap B) \cap D = A \cap D + C \cap B$.
        }
        Применяя лемму о бабочке, получаем, что искомая изоморфность фактормодулей (отмеченная вопросиком) действительно имеет место.
    }
}
\theorem[Жордан --- Гёльдер]{
    Любые два композиционных ряда без повторений эквивалентны.
    \provehere{
        Построим общее уплотнение данных рядов.
        Так как ряды неуплотняемы, то уплотнение только добавляет нуль-факторы.
    }
}
\corollary{\label{length}
Длина любого ряда без повторений не больше длины композиционного ряда.
}
\definition[Длина модуля]{
    Длина композиционного ряда данного модуля.
}
\corollary[Теорема Ремака --- Крулля --- Шмидта]{
    Если $M = \bigoplus\limits_{i = 1}^{n}M_i$, где $M_i$ --- простые, то $\{M_i\}$ определено однозначно с точностью до перестановки.
}
\note{
    Данная теорема также будет доказана в другой общности с использованием техники характеров:~(\ref{Krull–Schmidt-Remak-theorem})
}
\proposal{
    Пусть $M$ --- артинов модуль.
    Тогда $M = \bigoplus\limits_{i = 1}^{n}M_i$ $\iff \forall N \le M: \exists N': N \oplus N' = M$.
    \provewthen{
        Рассматриваем модуль $M$. Если он не простой, то $\exists N, N': N \oplus N' = M$.
        Дальше ветвимся относительно $N, N'$ (если $L \le N$, то $\exists L': L \oplus L' = M$, откуда $L \oplus (L' \cap N) = N$, то есть посылка теоремы верна и для $N,N'$).

        Из-за артиновости дерево ветвления конечно (в бесконечном дереве есть бесконечная ветвь).
    }{
        Выберем максимальное $k \le n$, такое, что после перенумерации $\left(\bigoplus\limits_{i = 1}^{k}M_i\right) \cap N = \{0\}$.
        Положим $M' \coloneqq \bigoplus\limits_{i = 1}^k M_i$.
        Из максимальности $\forall j > k: \left(M' \oplus M_j\right) \cap N = \{0\}$.

        Докажем, что $M = M' \oplus N$.
        Достаточно доказать, что $\forall j > k: M_j \le M' \oplus N$.

        В силу максимальности $k$: $N' \coloneqq N \cap (M' \oplus M_j) \ne \{0\}$.
        Выберем $x \in N' \sm \{0\}$.
        Он раскладывается в сумму $x = m' + y$, где $m' \in M', y \in M_j$.

        $y \ne 0$, так как $N \cap M' = \o$.
        Устроим проекцию $\pi: M' \oplus M_j \map M_j$, заметим, что $\pi(x) = y \ne 0$, отсюда $\pi(N')$ нетривиально.
        Так как образ модуля --- подмодуль, то $\pi(N') \le M_j$. Из простоты $M_j: \pi(N') = M_j$.
    }
}
\newlection{17 октября 2023 г.}


\section{Немножко фактов про кольца}
Кольца, как и модули, тоже бывают артиновы и нётеровы, в них ACC и DCC --- условия на цепочки идеалов.
\precaution{
    Артиновость слева и справа --- разные вещи.
    Кольцо $\vect{\Q & \R \\ 0 & \R}$ справа артиново, слева --- не артиново и даже не нётерово.
}
Так как идеал в кольце является подмодулем, то для поля $F$: $F[G]$ является артиновым (хотя бы из соображений размерности).

Пусть $R$ --- некоммутативное кольцо.
Радикал Джекобсона не выделить прямым слагаемым, но точную формулировку того, как и когда конкретно он мешает, я не уловил.

\definition[Регулярный $R$-модуль]{$R$ как модуль над $R$.}
Следует различать левый и правый регулярные $R$-модули.
\proposal{
    Пусть $r \in R$.
    Следующие условия эквивалентны.
    \bullets{
        \item[(1L)] $r$ лежит в любом максимальном левом идеале.
        \item[(1R)] $r$ лежит в любом максимальном правом идеале.
        \item[(2L)] $r$ можно исключить из любой системы образующих левого регулярного $R$-модуля.
        \item[(2R)] $r$ можно исключить из любой системы образующих правого регулярного $R$-модуля.
        \item[(3L)] $\forall x \in R: 1 + xr$ обратимо слева.
        \item[(3R)] $\forall x \in R: 1 + rx$ обратимо справа.
        \item[(4L)] $\forall x \in R: 1 + xr$ двусторонне обратим.
        \item[(4R)] $\forall x \in R: 1 + rx$ двусторонне обратим.
        \item[(5)] $\forall x, y \in R: 1 + xry$ обратим.
    }
    \provehere{
        Как уверяет лектор, $3L \then 4L$ --- сложный трюк, остальное --- более простые упражнения.
    }
}
\definition[Радикал Джекобсона]{
    Множество $r \in R$, удовлетворяющих пунктам леммы выше. Обозначается $\Rad(R) = \text{JRad}(R)$.
}
$\Rad(R)$ --- двусторонний идеал, как пересечение левых идеалов, и как пересечение правых иделаов.
\proposal[Радикальность радикала]{
    $\Rad(R/\Rad(R)) = \{0\}$.
}
\theorem{
    Если $R$ --- артиново кольцо, то $\exists n \in \N: \Rad(R)^n = 0$.
    Таким образом, в артиновых кольцах $\Rad(R)$ --- нильпотентный идеал.
}
\note{
    Напоминание: $A, B \normeq R \then AB \bydef \defset{\sum\limits_{i = 1}^{n}x_i y_i}{n \in \N, x_i \in A, y_i \in B} \normeq R$.

    Так, если $A = B = (x, y) \normeq \Ff_2[x, y]$, то $AB = (x^2, y^2, xy) \normeq \Ff_2[x, y]$.

    Можно заметить, что $AB$ не совпадает с множеством произведений $xy, (x\in A, y \in B)$, ни тем более с множеством квадратов элементов $A$.
}
Если $\Rad(R) \ne \{0\}$, то регулярный модуль не является вполне приводимым.

Предположим, что $R = \Rad(R) \oplus M$.
Тогда $R = M$, так как все элементы радикала можно выкинуть из системы образующих.

Таким образом, радикал никогда не выделяется прямым слагаемым.
\definition[Полупростое кольцо]{$\Rad(R) = \{0\}$.}
\theorem[Веддербарн --- Артин]{
    Если $R$ --- полупростое артиново кольцо (эквивалентно, классически полупростое), то $R = \bigoplus_{i = 1}^{n}M_{k_i}(D_i)$, где $D_i$ --- тела.
    \provehere{
        Схема доказательства
        \numbers{
            \item $I$ --- нильпотентный правый идеал $\exists a, x \in I: (a^2 - a)x = 0, ax \ne 0$.
            \item Любой правый идеал содержит идемпотент.
            \item Любой двусторонний идеал содержит центральный идемпотент $\then$ выделяется прямым слагаемым.
            \item $R$ --- прямая сумма простых колец без двусторонних идеалов.
            \item $R$ --- простое, $I$ --- минимальный правый идеал $\then \exists n \in \N: R \cong I^{\oplus n}$, как $R$-модуль.
            \item $R \cong \End_R(I^{\oplus n}) \cong M(n, D)$, где $D = \End_R(I)$.
        }
    }
}
\proposal{
    Если $R = \bigoplus\limits_{i = 1}^{n} R_i$, $M$ --- левый $R$-модуль, то $M = \bigoplus\limits_{i = 1}^{n}R_i \cdot M$.
    Если $M$ --- простой, то все слагаемые, кроме одного, равны нулю.
}
Таким образом, если $R$ --- полупростое артиново кольцо, то можно интересоваться только модулями над матричными кольцами над телами.

Никаких конечных тел над алгебраически замкнутым полем нет (представим в матричном виде, теорема Кэли --- Гамильтона, минимальный многочлен $x - \alpha$).
\proposal{
    Пусть $M$ --- простой левый модуль над $M(k, D)$, где $D$ --- тело.
    Тогда $M \cong D^k$.
}
Используя теорему Веддербарна --- Артина $\sum\limits_{i = 1}^{n}k_i^2 = \dim F[G] = |G|$.
Это мы также докажем впоследствии, используя технику характеров~(\ref{sum-of-squares-theorem}).

\lemma{
    Если $D$ --- конечномерная алгебра без делителей нуля (например, тело) над алгебраически замкнутым полем $F$, то $D \cong F$.
    \provehere{
        Пусть $a \in D$. Устроим \begin{align*}
                                     \psi: F[t] \map D \\ t \mapsto a
        \end{align*} $\Ker(\psi) \ne \{0\}$, так как $\dim_F(F[t]) = \infty$, но $\dim_F(D) < \infty$.
        Тогда $F[t]/\Ker(\psi) \cong \Image(\psi)$ без делителей нуля.

        Таким образом, $\Ker(\psi)$ --- простой идеал, но кольцо многочленов евклидово, поэтому это максимальный идеал.

        $\Ker(\psi) = p \cdot F[t]$, где $p$ неприводим, тогда $p(t) = t - \alpha$ и $F[a] = F[t]/(p) \cong F$.

        Тогда если $a \in 1_A \cdot F$, то $D = 1_D \cdot F = F$.
    }
}
\newlection{7 сентября 2016 г.}
% Представления конечных групп, полилинейная алгебра, теория категорий, гомологическая алгебра.
Чаще всего у нас будут иметься предположения о конечности группы $|G| < \infty$ и алгебраической замкнутости базового поля $\overline{K} = K$.

Пусть $\chr(K) = p$.
Случай $p \notdivs |G|$ более простой, о нём говорит \emph{теория обыкновенных представлений}.

Другой случай $p \divs |G|$ изучает \emph{теория модулярных представлений}.


\section{Три с половиной языка}
Есть несколько эквивалентных языков, чтобы говорить о представлениях групп.
\bullets{
    \item Линейные представления $G$ над $R$.
    \item Линейные действия $G$ на $R$-модулях.
    \item Модули над $R[G]$, где $R[G]$ --- групповая алгебра $G$ над $R$.
    \item Частный случай линейных представлений --- матричные представления (на свободных $R$-модулях с фиксированным базисом).
}
Эквивалентность данных языков установлена Эмми Нётер в 1926 году.

\subsection{Линейные представления группы $G$}
Пусть $R$ --- коммутативное ассоциативное кольцо с единицей (обычно поле).
Коммутативность нужна для того, чтобы работать с матрицами было приятно.

Пусть $V$ --- $R$-модуль (скоро станет векторным пространством, или по крайней мере свободным модулем конечного ранга).
\definition[Линейное представление группы $G$ над $R$ с модулем представления $V$]{
    Гомоморфизм $\pi: G \map GL(V) = \Aut_R(V)$ --- в \emph{полную линейную группу модуля}.
}
Обычно образ $g$ при действии $\pi$ обозначается $\pi_g$, чтобы не плодить скобок.

Здесь $\pi$ --- представление (representation), и $V$ --- модуль представления (presentation module).

Свойствами гомоморфизма являются
\bullets{
    \item $\pi_h \cdot \pi_g = \pi_{hg}$. В частности, $\pi_e = \id$ и $\pi_{g^{-1}} = \pi_g^{-1}$.
}

\subsection{Линейные действия}
Если $G$ действует просто на $V$, как на множестве, то задано отображение
\begin{align*}
    G \times V &\map V \\ g,x&\mapsto gx
\end{align*}
со свойством внешней ассоциативности $(hg)x = h(gx)$.

Если дано представление, то действие можно определить так:
\begin{align*}
    G \times V &\map V \\ g,v &\mapsto \pi_g(v) = gv
\end{align*}
Это действие, так как $\pi$ --- гомоморфизм.
При этом, получилось не просто действие, а линейное действие: $\forall u, v \in V, \lambda \in R$:
\bullets{
    \item $\pi_g(u + v) = \pi_g(u) + \pi_g(v)$ \quadили же\quad $g(u + v) = gu + gv$
    \item $\pi_g(\lambda u) = \lambda \pi_g(u)$ \quadили же\quad $g(\lambda u) = \lambda gu$.
}
Обратно, если задано действие $G \curvearrowright V$ то ему можно сопоставить представление
\begin{align*}
    \pi: G &\map GL(V) \\ g &\mapsto (v \mapsto gv)
\end{align*}
\fact{Таким образом, линейные представления --- то же самое, что и линейное действие.}

\subsection{Структура $R[G]$ модуля над каким-то $R$-модулем}
Здесь будет существенно, что группа конечна.

Помним, что групповая алгебра $R[G]$ --- это алгебра, элементы которой интерпретируются как $\sum\limits_{g \in G}a_g g$, $a_g \in R$.
\precaution{
    Не стоит путать групповую алгебру с алгеброй функций $R^G$ --- двойственной к групповой алгебре.
    Элементы алгебры функций --- $\sum\limits_{g \in G}a_g \delta_g$, где функция \[\delta_g(h) = \delta_{g,h} = \all{1,&g = h \\ 0,&g \ne h}\]
}
Сумма и произведение элементов $R[G]$ определены в виде
\gather{
    \sum\limits_{g \in G}a_g g + \sum\limits_{g \in G}b_g g = \sum\limits_{g \in G}(a_g + b_g)g \\
    \left(\sum\limits_{g \in G}a_g g\right) \cdot \left(\sum\limits_{h \in G}b_h h\right) = \sum\limits_{h, g \in G}(a_gb_h)g = \sum\limits_{f \in G}\left(\sum\limits_{h \in G}a_h b_{h^{-1}f}\right)f \\
}
Линейному действию $G \curvearrowright V$ сопоставим действие $R[G] \curvearrowright V$, определённое в виде
\[\left(\sum\limits_{g \in G}a_g g\right) v = \sum\limits_{g \in G}a_g (gv)\]
Можно проверить, что данная формула задаёт на $V$ структуру левого $R[G]$-модуля.

Поскольку $G \hookrightarrow R[G]$, то верно и обратное --- $R[G]$-модуль определяет линейное действие $G$ на $V$.

\subsection{Глоссарий терминов}
\begin{tabular}{|c|c|}
    \hline
    Представление                               & Модуль                                              \\\hline
    Факторпредставление                         & Фактормодуль                                        \\\hline
    Сплетающий оператор (intertwining operator) & Гомоморфизм                                         \\\hline
    Неприводимое представление                  & Простой модуль (у которого ровно два подмодуля)     \\\hline
    Неразложимое представление                  & Неразложимый (в прямую сумму) модуль                \\\hline
    Эквивалентность                             & Изоморфизм                                          \\\hline
    Инвариантное подпространство                & Инвариантное подпространство                        \\\hline
    Полная приводимость                         & Полупростота (прямая сумма конечного числа простых) \\\hline
\end{tabular}

В дальнейшем мы будем предполагать, что $V = R^n$ --- свободный модуль конечного ранга.
Над полем, очевидно, достаточно считать, что $\dim(V) < \infty$.
\definition[Конечномерное линейное представление над $K$]{
    Представление, в котором модуль представления конечномерен.
}
Если зафиксировать $e_1, \dots, e_n$ --- базис $V$, то $V = R^n$ и $GL(R^n) = GL(n, R)$.

Здесь линейные операторы --- матрицы.

В этом случае можно определять $\pi: G \map GL(n, R)$.
Это \emph{матричное представление}.

\ok
Матрица записывается $x = (x_{i,j})_{1 \le i,j \le n}$.
В данной главе нас больше всего будет волновать след $\sum\limits_{i}x_{i,i}$.

Работая в матрицах, придётся не забывать, что мы используем их с точностью до сопряжения.
С другой стороны, считать что-то в матрицах легче.

\definition[Степень представления]{
    Ранг модуля представления. Обозначают $\deg(\pi)$.
}
Пусть $n = \deg(\pi)$, где $\pi: G \map GL(n, R)$.
Тогда $\pi_g = ((\pi_g)_{i,j})_{1 \le i,j \le n}$.

Коэффициенты матрицы $\pi_g$ обозначают $\pi_{i,j}(g) \in R$, опять же чтобы не плодить скобок.
$\pi_{i,j}$ здесь --- матричный элемент представления $\pi$ в позиции $(i,j)$.


\section{Сплетающие операторы}
Пусть $\pi:G \map GL(U), \rho: G \map GL(V)$ --- два представления.

\definition[Сплетающий оператор (гомоморфизм) $\phi$ между $\pi$ и $\rho$]{
    Для любого $g \in G$ диаграмма коммутативна.
    % https://q.uiver.app/#q=WzAsNSxbMCwxLCJnIFxcaW4gRyJdLFsxLDIsIlxccmhvX2c6IFYiXSxbMSwwLCJcXHBpX2c6IFUiXSxbMiwwLCJVIl0sWzIsMiwiViJdLFswLDFdLFswLDJdLFsyLDEsIlxccGhpIiwwLHsib2Zmc2V0IjotNH1dLFszLDQsIlxccGhpIiwyXSxbMSw0XSxbMiwzXV0=
    \[\begin{tikzcd}[ampersand replacement=\&]
          \& {\pi_g: U} \& U \\
          {g \in G} \\
          \& {\rho_g: V} \& V
          \arrow[from=2-1, to=3-2]
          \arrow[from=2-1, to=1-2]
          \arrow["\phi", shift left=4, from=1-2, to=3-2]
          \arrow["\phi"', from=1-3, to=3-3]
          \arrow[from=3-2, to=3-3]
          \arrow[from=1-2, to=1-3]
    \end{tikzcd}\]
}
Иными словами (на языке действий, а не представлений) $\pi_g(u) \bydef gu, \rho_g(v) \bydef gv$ и коммутативность диаграммы значит $G$-\emph{эквивариантность}
\[\phi(gu) = g\phi(u)\]
Таким образом, сплетающий оператор --- в точности гомоморфизм $R[G]$-модулей:
\[\phi\left(\sum\limits_{g \in G}a_g g\cdot  u\right) = \sum\limits_{g \in G}\phi(a_g g \cdot u) = \sum\limits_{g \in G}a_g\phi(g \cdot u) = \sum\limits_{g \in G}a_g g \phi(u) = \left(\sum\limits_{g \in G}a_g g\right) \phi(u)\]
Мы определили то, что далее будет называться \emph{категорией представлений} --- объекты и морфизмы между ними.

В случае, когда $\phi$ --- \emph{изоморфизм} модулей оно называется \emph{эквивалентностью}.
Далее всюду будем смотреть на представления с точностью до эквивалентности.
\newlection{7 сентября 2016 г.}


\section{Изоморфизм представлений}
Пусть $\pi: G \map GL(U), \rho: G \map GL(V)$ --- два представления.

Элементу $g$ соответствует левый квадрат, но так как $\phi$ обратимо, то его коммутативность равносильна коммутативности правого квадрата
% https://q.uiver.app/#q=WzAsOCxbMCwwLCJVIl0sWzAsMSwiViJdLFsxLDAsIlUiXSxbMSwxLCJWIl0sWzMsMCwiVSJdLFszLDEsIlYiXSxbNCwxLCJWIl0sWzQsMCwiVSJdLFswLDEsIlxccGhpIl0sWzAsMiwiXFxwaV9nIl0sWzEsMywiXFxyaG9fZyJdLFsyLDMsIlxccGhpIiwyXSxbNSw2LCJcXHJob19nIl0sWzQsNywiXFxwaV9nIl0sWzUsNCwiXFxwaGleey0xfSIsMl0sWzcsNiwiXFxwaGkiLDJdXQ==
\[\begin{tikzcd}[ampersand replacement=\&]
      U \& U \&\& U \& U \\
      V \& V \&\& V \& V
      \arrow["\phi", from=1-1, to=2-1]
      \arrow["{\pi_g}", from=1-1, to=1-2]
      \arrow["{\rho_g}", from=2-1, to=2-2]
      \arrow["\phi"', from=1-2, to=2-2]
      \arrow["{\rho_g}", from=2-4, to=2-5]
      \arrow["{\pi_g}", from=1-4, to=1-5]
      \arrow["{\phi^{-1}}"', from=2-4, to=1-4]
      \arrow["\phi"', from=1-5, to=2-5]
\end{tikzcd}\]
Получаем соотношение сопряжения $\rho_g = \phi \circ \pi_g \circ \phi^{-1}$.

Выбрав базисы в $U, V$ получаем два гомоморфизма $G \map GL(n, R)$, таких, что найдётся обратимая матрицы $x \in GL(n, R)$:
\[\forall g \in G: x \pi_g x^{-1} = \rho_g\]
Эти представления эквивалентны.


\section{Подпредставление}
Пусть $\pi: G \map GL(V)$, где $V$ --- $R$-модуль, $U \le V$.
\definition[$U$ --- $G$-подмодуль]{
    $R[G]$-подмодуль в $V$, или же $G$-инвариантное подпространство.
}
Требование об отсутствии $G$-подмодулей в случае кольца $R$ не выполняется практически никогда --- в кольце много идеалов.
Далее предполагаем, что $R = K$ --- поле.

\definition[Неприводимое представление $\pi: G \map GL(V)$]{
    $V \ne \{0\}$ и в $V$ нет нетривиальных $G$-инвариантных подпространств.
    Иначе представление называется \emph{приводимым}.
}
Если представление приводимо ($U \le V$ --- $G$-инвариантное подпространство), то в $U$ найдётся базис $e_1, \dots, e_m$, он дополняется до базиса $V$.

В этом базисе для любого $g$:
\[\pi_g = \vect{\arr{c|c}{* & * \\\hline 0 & *}}\]
Матрицы такого вида образуют \emph{стандартную параболическую подгруппу}.
\definition[$m$-я стандартная параболическая подгруппа $P_m \le GL(n, K)$]{
    \[P_m \bydef \defset{\vect{\arr{c|c}{a & b \\\hline 0 & c}}}{a \in GL(m, K), c \in GL(n - m, K), M \in M(m, n - m, K)},\text{ где }1\le m\le n\]
}
\note{
    Неприводимость представления --- свойство не самого представления, а свойство образа $\Image(\pi) = \pi(G) = \defset{\pi_g}{g \in G}$.
}
Пусть $\pi: G \map GL(V)$ --- представление, $U \le V$ --- $G$-подмодуль.
\definition[Подпредставления]{
    \begin{align*}
        \pi_U: G &\map GL(U)\\g &\mapsto (\pi_g)\Big|_U
    \end{align*}
}
\precaution{
    Не путать с \emph{ограничением} представления $\pi$ на подгруппу $H \le G$. Ограничение обозначается $\restricted_H^G(\pi) = \pi\Big|_H: H \map GL(V)$.
}


\section{Лемма Шура}
Пока $G$ --- произвольная группа, $K$ --- любое поле.
\lemma[Лемма Шура --- 1]{
    Пусть $U, V$ --- неприводимые $G$-модули, $\phi: U \map V$ --- гомоморфизм $G$-модулей.
    Тогда $\phi = 0$, либо $\phi: U \cong V$.
    \provehere{
        $\Ker(\phi)$ --- $G$-подмодуль в $U$.
        $\forall u \in U: \phi(u) = 0 \then \forall g \in G: \phi(gu) = g \phi(u) = 0$, то есть $gu \in \Ker(\phi)$.

        Но таких подмодулей только два.
        \bullets{
            \item Если $\Ker(\phi) = \{0\}$, то $\phi$ --- мономорфизм (инъекция).
            \item Если $\Ker(\phi) = U$, то $\phi \equiv 0$.
        }
        $\Image(\phi) \le V$ --- $G$-подмодуль. В самом деле, $v \in \Image(\phi) \then \exists u \in U: \phi(u) = v \then \forall g \in G: g\phi(u) = \phi(gu) \in \Image(\phi)$.

        Но таких подмодулей только два.
        \bullets{
            \item Если $\Image(\phi) = \{0\}$, то $\phi \equiv 0$.
            \item Если $\Image(\phi) = V$, то $\phi$ --- эпиморфизм (сюръекция).
        }
        Если $\phi \ne 0$, то $\phi$ --- одновременно мономорфизм и эпиморфизм, то есть изоморфизм.
    }
}
\corollary[Лемма Шура --- 2]{
    Пусть $K$ --- поле, $U, V$ --- неприводимые $G$-модули над $K$.
    Тогда если $U \ncong V$, то множество сплетающих операторов между $U$ и $V$ $\Hom_G(U, V) = \Hom_{K[G]}(U, V) = 0$.

    Иначе если $U \cong V$, то $\Aut_G(U)$ --- тело (любой автоморфизм либо равен нулю, либо обратим).
}
Теперь дополнительно предположим, что $K$ --- алгебраически замкнутое поле, и что $\dim U, \dim V < \infty$.
\lemma[Лемма Шура --- 3]{
    Если $U, V$ --- неприводимые конечномерные $G$-модули над $K$, а $\phi \in \Hom_{K[G]}(U, V)$, то
    \bullets{
        \item Либо $\phi \equiv 0$.
        \item Либо $\phi: U \cong V$, и тогда $\phi = \lambda \id$ (где $\lambda \in K$) --- гомотетия.
    }
    \provehere{
        Любому скаляру $\lambda \in K$ можно сопоставить сплетающий оператор \begin{align*}
                                                                                 \lambda \id_U: U &\map U\\u & \mapsto \lambda u
        \end{align*}
        Из $G$-линейности $g(\lambda u) = \lambda (gu)$. При $\lambda \ne 0$: $\lambda \id$ --- автоморфизм.

        Если $\phi: U \map U$ --- $G$-эндоморфизм, то условие алгебраической замкнутости значит в точности то, что $\forall \phi: \exists \lambda \in K$ --- собственное число:
        \[\exists u \in U \sm \{0\}: \phi(u) = \lambda u\]
        Отсюда $(\phi - \lambda \id_U)(u) = 0$.
        Но тогда $\phi - \lambda \id_U$ --- $G$-эндоморфизм $U$ с ненулевым ядром. Тогда $\phi - \lambda \id_U = 0$.
    }
}


\section{Факторпредставление}
Пусть $\pi: G \map GL(V)$ --- представление, $U$ --- $G$-подмодуль.
Тогда $\pi_g$  в подходящем базисе имеют вид
\[\vect{\arr{c|c}{\pi_g\Big|_{U} & * \\\hline 0 & \pi_g\Big|_{V/U}}}\]
$\pi_g\Big|_{V/U}: G \map GL(V/U)$.
Фактормодуль $V/U = \defset{v + U}{v \in V}$ состоит из смежных классов, параллельных $U$.
\[g(v + U) = gv + U\text{, так как $U$ --- $G$-подмодуль.}\]
\definition[Факторпредставление $\pi$ по инвариантному подпространству $U \le V$]{
    Выше полученное $\pi\Big|_{V/U}$.
}
\fact{
    Матрица факторпредставления --- в точности правый нижний блок, натянутый на базисные векторы $e_{m + 1}, \dots, e_n$.
}
\ok
Рассмотрим группу $P_m = \defset{\vect{a & b \\ 0 & c}}{a \in GL(m, R), c \in GL(n - m, R)}$.
Это группа:
\[\vect{a_1 & b_1 \\ 0 & c_1}\vect{a_2 & b_2 \\ 0 & c_2} = \vect{a_1 a_2 & a_1 b_2 + b_1 c_2 \\ 0 & c_1 c_2}\qquad\text{и}\qquad\vect{a & b \\ 0 & c}^{-1} = \vect{a^{-1} & -a^{-1}bc^{-1} \\ 0 & c^{-1}}\]
$L_m = \defset{\vect{a & 0 \\ 0 & c}}{a \in GL(m, R), c \in GL(n - m, R)}$ --- подгруппа Л$\acute{\text{е}}$ви.

Здесь ещё полезно вспомнить
$U_m = \defset{\vect{e & b \\ 0 & e}}{b \in M(m, n - m, R)}$.

Отображение \begin{align*}
                P_m &\map L_m\\\vect{a & b \\ 0 & c} &\mapsto \vect{a & 0 \\ 0 & c}
\end{align*}
является гомоморфизмом!


\section{Прямая сумма представлений. Неразложимые представления}
Пусть $\pi: G \map GL(U), \rho: G \map GL(V)$ --- два представления одной и той же группы на разных модулях.

\definition[Прямая сумма представлений]{
    \begin{align*}
        \pi \oplus \rho: G &\map GL(U \oplus V)\\ g &\mapsto (\pi \oplus \rho)_g
    \end{align*}
    где $(\pi \oplus \rho)_g \bydef ((u,v)\mapsto(\pi_g(u), \rho_g(v)))$
}
Если $U, V$ --- свободные модули, то в качестве базиса прямой суммы можно взять объединение базисов $U$ и $V$.
В этом базисе матрица $\pi \oplus \rho$ --- это прямая сумма матриц $\pi$ и $\rho$.

\note{
    Если не только модули разные, но и группы разные, то двум представлениям $\pi: H \map GL(U), \rho: G \map GL(V)$ можно сопоставить наружную прямую сумму --- представление группы
    $H\times G$ --- прямого произведения групп.
    \begin{align*}
        \pi \boxplus \rho: H \times G \map GL(U \oplus V)\\(h,g)\mapsto \underbrace{(\pi \boxplus \rho)_{(h,g)}}_{\pi_h \oplus \rho_g}
    \end{align*}
}
Обычная прямая сумма представлений $\pi \oplus \rho$ --- это ограничение $\restricted_{\Delta(G)}^{G \times G}(\pi \boxplus \rho)$, где $\Delta(G)$ --- диагональ.
\ok
Если $U$ --- $G$-инвариантное подпространство в $V$, то когда $\pi$ раскладывается в прямую сумму?

Если $R = K$ --- поле, то у любого подпространства $U$ найдётся дополняющее (необязательно $G$-инвариантное) подпространство $W: V = U \oplus W$.

Если $W$ тоже $G$-инвариантно, то $\pi = \pi\Big|_U \oplus \pi\Big|_W$.
\newlection{14 сентября 2016 г.}
$G$ --- конечная группа, $K$ --- поле характеристики $p \notdivs |G|$.
Позже даже будем предполагать $p = 0$.

И, конечно, все представления конечномерны.


\section{Усреднение по конечной группе}

\subsection{Усреднение векторов}
Пусть $\pi: G \map GL(V)$, $V = K^n$.
Найдём инвариантные элементы.
\definition[Инвариантные элементы]{
    \[V^G = \defset{v \in V}{\forall g \in G: \pi_g(v) = v} \le V\]
}
Построим (сюръективную) проекцию $V \map V^G$.

Так как группа конечная, то по ней можно усреднять.
Устроим \begin{align*}
            \phi: V & \map V^G \\ v &\mapsto \frac{1}{|G|}\sum\limits_{g \in G} gv
\end{align*}
Тогда $\forall v \in V: \phi(v) \in V^G$:
\[h \phi(v) = h \frac{1}{|G|}\sum\limits_{g \in G}gv = \frac{1}{|G|}\sum\limits_{g \in G}(hg)v = \phi(v)\]
Из-за усреднения, то есть деления на $|G|$, также верно, что $\forall v \in V^G: \phi(v) = v$.

\subsection{Усреднение линейных отображений}
Пусть $\pi: G \map GL(U), \rho: G \map GL(V)$ --- представления.

Тогда утверждается, что $\Hom_K(U, V)$ несёт структуру линейного представления группы $G$.
Иными словами, сопоставим $\pi, \rho \rightsquigarrow \Hom(\pi, \rho)$.
\begin{align*}
    \Hom(\pi, \rho): G & \map GL(\Hom(U, V)) \\ g &\mapsto (\phi \mapsto \rho_g \phi \pi_g^{-1})
\end{align*}
Здесь $\Hom(\pi, \rho)_g$ получается из коммутативного квадрата
% https://q.uiver.app/#q=WzAsNCxbMCwwLCJVIl0sWzAsMSwiViJdLFsxLDAsIlUiXSxbMSwxLCJWIl0sWzAsMiwiXFxwaV9nIl0sWzEsMywiXFxyaG9fZyJdLFswLDEsIlxccGhpIl0sWzIsMywiXFxIb20oXFxwaSwgXFxyaG8pX2coXFxwaGkpIiwwLHsic3R5bGUiOnsiYm9keSI6eyJuYW1lIjoiZGFzaGVkIn19fV1d
\[\begin{tikzcd}[ampersand replacement=\&]
      U \& U \\
      V \& V
      \arrow["{\pi_g}", from=1-1, to=1-2]
      \arrow["{\rho_g}", from=2-1, to=2-2]
      \arrow["\phi", from=1-1, to=2-1]
      \arrow["{\Hom(\pi, \rho)_g(\phi)}", dashed, from=1-2, to=2-2]
\end{tikzcd}\]
Таким образом, два представления дали новое представление, теперь уже на множестве не строк или столбцов, а на множестве матриц.

Так как $\Hom$ по отношению к $U$ контравариантен, то $\pi_g$ возводится в степень $-1$.
По отношению к $V$ же $\Hom$ ковариантен и для $\rho_g$ не берётся обратный.

Теперь мы можем усреднять уже сами линейные отображения.
\definition[Усреднение линейного отображение]{
    \begin{align*}
        \Hom_K(U, V) &\map \Hom_K(U, V)^G \\ \phi & \mapsto \frac{1}{|G|}\sum\limits_{g \in G}\rho_g \phi \pi_g^{-1}
    \end{align*}
}
Образ состоит из элементов $\defset{\phi \in \Hom_K(U, V)}{\forall g \in G: \rho_g \phi \pi_g^{-1} = \phi} = \Hom_{K[G]}(U, V)$.
В дальнейшем вместо $\Hom_K(U, V)$ будем писать $\Hom(U, V)$, вместо $\Hom_{K[G]}(U, V)$ --- $\Hom_G(U, V)$.


\section{Теорема Машке}
$G$ --- конечная группа, $K$ --- поле характеристики $p \notdivs |G|$.
Все представления конечномерны.
\definition[Вполне приводимое представление]{
    Для любого $G$-инвариантного подпространства $U \le V$: $\exists G$-инвариантное дополнение $W$.
    Иными словами, $K[G]$ полупроста (что?).
}
\theorem{
    В данных условиях все представления вполне приводимы.
    \provehere{
        Для $G$-инвариантного подпространства $U \le V: \exists W$ --- какое-то (необязательно $G$-инвариантное) дополняющее подпространство: $U \oplus W = V$.

        Мы не умеем усреднять подпространства, поэтому поступим так.
        Всякое подпространство --- образ или ядро какого-то линейного отображения.
        А линейные отображения усреднять мы только что научились.

        Положим в качестве $\phi: V \map V$ проектор $V$ на $U$ вдоль $W$.
        Усреднив $\phi$:
        \[\phi_0 = \frac{1}{|G|}\sum\limits_{g \in G}\pi_g \phi \pi_g^{-1}\]
        Утверждается, что $\phi_0$ --- проектор на $U$ вдоль $W_0 \coloneqq \Ker(\phi_0)$, причём $W_0$ $G$-инвариантно.

        Проверим, что $\Image(\phi) \le U$. $\forall v \in V$:
        \[\phi_0(v) = \frac{1}{|G|}\sum\limits_{g \in G}\pi_g \underbrace{\phi(\pi_g^{-1}(v))}_{\in U} \in U\]
        Так как $\phi$ $U$-инвариантно, то $\forall u\in U$:
        \[\phi_0(u) = \frac{1}{|G|}\sum\limits_{g \in G}\pi_g \phi(\pi_g^{-1}(u)) = \pi_g \pi_g^{-1}u = u\]
        Таким образом, $\phi$ --- проектор на $U$.

        Осталось проверить, что $W_0 \coloneqq \Ker(\phi_0)$ --- $G$-инвариантное подпространство. $\forall h \in G, v \in W_0$:
        \[\phi_0(\pi_h(v)) = \pi_h\phi_0(v) = \pi_h(0) = 0\text{, то есть $\pi_h(v) \in \Ker(\phi_0)$.}\]
        Применяя теорему о размерности ядра и образа, и тот факт, что $\Ker(\phi)\cap \Image(\phi) = \{0\}$ (используем, что $\phi^2 = \phi$) получаем $V = U \oplus W_0$.
    }
}
\note[Относительно разницы между проекцией и проектором]{
    Если $V = U \oplus W$, то $\phi: V \map U$ --- проекция на $U$ параллельно $W$, определена так: $\phi(u + w) = u$.

    Проектор --- это отображение $\phi: V \map V$, которое также переводит $\phi(u + w) = u$.
    Различие состоит в области значений.
}
\corollary{
    В условиях теоремы Машке имеет место полная приводимость: неприводимые представления совпадают с неразложимыми представлениями.

    Любое конечномерное представление равняется прямой сумме неприводимых.
}
Таким образом,
\bullets{
    \item Задачи теории обыкновенных представлений свелись к классификации неприводимых представлений $G$ над $K$, и
    \item К разложению любого представления в прямую сумму неприводимых.
}
%    \intfact[Теорема Ремака --- Крулля --- Шмидта]{
%        Если $V$ --- артинов и нётеров модуль, то разложение на неразложимые единственно (с точностью до порядка и изоморфизма).
%        \[V = M_1 \oplus \dots \oplus M_s = N_1 \oplus \dots \oplus N_t\]
%        влечёт $s = t$ и $\exists \sigma \in S_t: M_i \cong N_{\sigma(i)}$.
%    }


\section{Унитаризуемость}
Пусть $K = \C, V = \C^n$.
\theorem[Теорема Машке над $\C$]{\label{uni}
Для люого представления конечной группы $G$ над $\C: \exists G$-инвариантное положительно определённое эрмитово скалярное произведение.
\provehere[Доказательство теоремы Машке над $\C$]{
    Вспомним про эрмитово скалярное произведение $B: V \times V \map \C$ --- полуторалинейное и эрмитовски симметричное ($B(u,v) = \overline{B(v, u)}$).

    Дополнительно можно считать, что $\forall v \in V: B(v, v)\ge 0$, причём $B(v,v) = 0 \iff v = 0$.
    Это классическое эрмитово (унитарное) скалярное произведение, превращающее $V$ в гильбертово пространство.
    \[B\left(\vect{u_1 \\ \vdots \\ u_n},\vect{v_1, \\ \vdots \\ v_n}\right) = \overline{u_1}v_1 + \dots + \overline{u_n}v_n\]

    Пусть $\pi: G \map GL(\C^n) = GL(n, \C)$.
    Научимся усреднять скалярное, чтобы действие элементов группы сохраняло скалярное произведение.
    Скалярное произведение $B$ \emph{унитарно}, если $\forall u, v \in V: B(gu, gv) = B(u, v)$.

    Хотим, чтобы $\pi$ било в $U(n, \C) \bydef \defset{x \in GL(n, \C)}{\overline{x}^t x = e = x \overline{x}^t}$ --- классическую унитарную группу.
    Здесь $\overline{x}^t$ обычно обозначается $x^*$ --- эрмитовски сопряжённая матрица к $x$.
    Кстати, $U(n, \C)$ --- компактная группа относительно комплексной топологии.

    Переписав унитарность в терминах матрицы Грама (которая равна $e$), получаем именно, что образ $\pi$ должен лежать в $U(n, \C)$.
    \[B_0(u, v) \coloneqq \frac{1}{|G|}\sum\limits_{g \in G}B(\pi_g (u), \pi_g (v))\]
    $B_0$ --- полуторалинейная эрмитова положительно полуопределённая форма.

    Теперь относительно $B_0$ все операторы $\pi_h$ ($h \in G$) унитарны.
    \[B_0(\pi_h(u),\pi_h(v)) = \frac{1}{|G|}\sum\limits_{g \in G}B(\pi_{gh}(u), \pi_{gh}(v)) = B_0(\pi_h(u), \pi_h(v))\]
}}
\corollary{
    Любое представление конечной группы над $\C$ унитаризуемо, то есть эквивалентно унитарному: $\rho: G \map U(n, \C)$.

    В унитарном представлении ортогональное дополнение к $G$-инвариантному подпространству само $G$-инвариантно.
    В частности, отсюда вытекает теорема Машке предыдущего параграфа над $\C$.
    \provehere{
        Если $U$ --- $G$-инвариантное подпространство в $V$, $B$ --- $G$-инвариантное положительно определённое эрмитово скалярное произведение на $V$, то $U^\perp$ тоже $G$-инвариантно и $U \oplus U^\perp = V$.
        В самом деле $\forall u \in U, v \in U^\perp$:
        \[B(u, \pi_g(v)) = B(\underbrace{\pi_g^{-1}(u)}_{\in U}, v) = 0\]
    }
}
\newlection{14 сентября 2016 г.}
Всё, касающееся усреднения, можно обобщить на компактные группы с усреднением по мере Хаара --- вместо суммирования и взятия среднего берётся интеграл.
Это называется \emph{гармонический анализ}.


\section{Характеры Фробениуса}
В дальнейшем все характеры будут именно характерами Фробениуса.

Пусть $\pi: G \map GL(V)$ --- конечномерное представление конечной группы над полем $K$, которое вскоре будет характеристики 0.

Выберем базис $e_1, \dots, e_n$.
При фиксированном базисе представление на самом деле является матричным.
\[\tr(x) = x_{1,1} + \dots + x_{n,n} = \lambda_1 + \dots + \lambda_n\]
где $\lambda_1, \dots, \lambda_n$ --- собственные числа $x$.
Они, вообще говоря, могут не лежать в базовом поле, но их сумма лежит.

Сопоставим представлению $\pi: G \map GL(n, K)$ \emph{характер Фробениуса представления} $\pi$.
\begin{align*}
    \chi_\pi: G \map K \\ g \mapsto \tr(\pi_g)
\end{align*}
\numbers{
    \item Характер зависит только от класса эквивалентности $\pi$.
    Два эквивалентных представления имеют равные характеры.
    \item Характер не обязательно является гомоморфизмом!
    \item Для двух представлений $\pi: G \map GL(U); \quad \rho: G \map GL(V)$ можно определить $\pi \oplus \rho: G \map GL(U \oplus V)$.
    \[(\pi \oplus \rho)_g(u, v) = (\pi_g(u), \rho_g(v))\]
    $\chi_{\pi \oplus \rho} = \chi_\pi + \chi_\rho$, так как $\tr\vect{\arr{c|c}{x & 0 \\\hline 0 & y}} = \tr(x) + \tr(y)$.
    \item $\chi_{\pi \otimes \rho} = \chi_\pi \cdot \chi_\rho$, так как $\tr(x \otimes y) = \tr(x)\cdot\tr(y)$, об этом см~(\ref{further-constructions}).
    \item Пусть $\pi \equiv 1$ --- главное представление. $\chi(\pi) = \dim(V) = \deg(\pi)$, так как $\tr\vect{1 & & 0 \\ & \ddots & \\ 0 & & 1} = n$.
    \item Характер является \emph{центральной функцией} на $G$.
    \definition[Центральная функция (функция класса)]{
        Функция, постоянная на классах сопряжённых элементов.
    }
    Иными словами, $h \sim_G g \then \chi_{\pi}(h) = \chi_{\pi}(g)$.
    Тогда так как $\exists f \in G: h = f^{-1}gf$, то $\pi_h = \pi_f^{-1}\pi_g \pi_f$.
    \item Пусть $g \in G, |G| = m < \infty$.
    Тогда $(\pi_g)^m = e$. Значит, все собственные числа любой матрицы $\pi_g$ являются корнями $m$-й степени из единицы.
    \[\chi_\pi(g) \in F\left(\!\!\sqrt[m]{1}\right)\text{, где $F$ --- простое подполе в $K$, то есть }\all{\Ff_p,&\chr(K) = p > 0 \\ \Q,&\chr(K) = 0}\]
    Если $\chr(K) = 0$, то $\chi_{\pi}(g) \in \A$, где $\A$ --- целые алгебраические числа (сумма корней из единицы лежит там, так как каждый корень из единицы лежит там, и $\A$ --- кольцо).

    Над полем же комплексных чисел $\frac1\omega = \overline{\omega}$. Если $K \le \C$, то $\chi_\pi(g^{-1}) = \overline{\chi_\pi(g)}$.

    С другой стороны, $\chi_{\pi}(g^{-1})$ --- характер \emph{двойственного представления}.
    \definition[Двойственное к $\pi: G \map GL(V)$ представление]{
        Левое представление $\pi^*: G \map GL(V^*)$. Для $\eta \in V^*, v \in V$:
        \[((\eta)\pi_g^*)(v) = \eta(\pi_g(v))\]
    }
    Чтобы писать операторы слева, то $\pi$ сопоставляем
    \begin{align*}
        \pi^*: G &\map GL(n, K) \\ g & \mapsto \pi_g^{-t}
    \end{align*}
    $\tr(x^t) = \tr(x)$, поэтому $\chi_{\pi^*}(g) = \chi_\pi(g^{-1})$.
    \corollary{
        Над $K \le \C: \chi_{\pi^*} = \overline{\chi_\pi}$.
    }
    Таким образом, если построено над $\C$ представление, у которого не все характеры вещественные, сразу строится сопряжённое --- другое неприводимое (двойственное и обычное представления неприводимы одновременно) --- представление.
    \item Пусть $K \le \C$. Тогда $\forall g \in G: \abs{\chi_\pi(g)} \le n = \chi_\pi\left(1_G\right)$, так как характер --- сумма корней из единицы.
    \item $\chi_{\Hom(\pi,\rho)} = ?$
    \item $\chi_{\bigwedge^m(\pi)} = \dots$. В частности, $\chi_{\bigwedge^2}(\pi) = \frac12(\chi_\pi(g)^2 - \chi_{\pi}(g^2))$. Дискретная теория вероятностей --- применение теории представлений конечных групп, поэтому эта штука похожа на дисперсию.
    \item $\chi_{S^m(\pi)} = ?$. В частности, $\chi_{S^2}(\pi) = \frac12(\chi_\pi(g)^2 + \chi_{\pi}(g^2))$.
    Можно удостовериться, что так как $S^2(\pi) \oplus \bigwedge^2(V) = V \otimes V$, то $\chi_{\bigwedge^m(\pi)} + \chi_{S^m(\pi)} = \chi(\pi\otimes\pi)$.
}
\intfact[Теорема Фробениуса]{
    Пусть $\chr K = 0$. Тогда $\pi \sim \rho \iff \chi_\pi = \chi_\rho$.
    \provehere{Будет доказана с использованием соотношения Шура (соотношения ортогональности).}
}
\counterexample[В теореме Фробениуса важна нулевая характеристика]{
    Пусть $\chr(K) = p > 0$.
    Существует главное представление $1_G: \arr{ccc}{G & \map & K^* \\ g & \mapsto & 1}$.

    Но если взять $\pi = \underbrace{1_G \oplus \dots \oplus 1_G}_{p + 1}$, то $\chi_{\pi} = \chi(1_G)$.
}


\section{Представления абелевых групп. Лемма Шура}
Пусть группа $G$ --- конечная абелева группа ($[G, G] = \{1\}, |G| < \infty$).

Пусть $K$ --- алгебраически замкнутое поле, $\chr(K) = 0$.
Так как все характеры лежат в $\A$, то достаточно считать, например, что $K = \C$.
\lemma[Лемма Шура]{
    Любое неприводимое представление конечной абелевой группы над алгебраически замкнутым полем одномерно.
    \provehere{
        Пусть $h, g \in G$. Тогда $\pi_h \cdot \pi_g = \pi_{gh} = \pi_{hg} = \pi_g \cdot \pi_h$.
        Таким образом, $\forall h \in G: \pi_h$ --- сплетающий оператор для $\pi$.

        Но $\pi$ неприводимо, тогда $\forall h \in G: \pi_h$ --- гомотетия.
        Но тогда все одномерные подпространства $G$-инвариантны, и из неприводимости $\deg(\pi) = 1$.
    }
}
\corollary{
    Если $\pi$ --- неприводимое представление $G$ над $K$, то одномерный характер --- в точности само представление: $\pi = \chi_{\pi}: G \map K^* = GL(1, K)$.
}
\counterexample{
    Если поле не замкнуто, то лемма Шура, конечно, неверна. Не существует одномерного представления $C_4$ над $\R$, так как над $\R$ нет первообразного корня четвёртой степени из 1.
}

\subsection{Классификация циклических групп}
Пусть $C_n = \angles{g} = \{g^0, g^1, \dots, g^{n - 1}\}$.
Построим таблицу, где столбцы отвечают элементам группы, строки --- характерам.

Рассмотрим для примера $C_2, C_3, C_4$.
\[\begin{array}{c|cc}
      C_2    & 1 & -1 \\\hline
      \chi_0 & 1 & 1  \\
      \chi_1 & 1 & -1
\end{array} \qquad
\begin{array}{c|ccc}
    C_3    & 1 & g        & g^2      \\\hline
    \chi_0 & 1 & 1        & 1        \\
    \chi_1 & 1 & \omega   & \omega^2 \\
    \chi_2 & 1 & \omega^2 & \omega   \\
\end{array} \qquad
\begin{array}{c|cccc}
    C_4    & 1 & g  & g^2 & g^3 \\\hline
    \chi_0 & 1 & 1  & 1   & 1   \\
    \chi_1 & 1 & i  & -1  & -i  \\
    \chi_2 & 1 & -1 & 1   & -1  \\
    \chi_3 & 1 & -i & -1  & i   \\
\end{array}\]
Так как в $C_n$: $g^n = 1$, то для всякого представления $\pi$: $(\chi_\pi(g))^n = \pi(g)^n = \pi(g^n) = 1$.
Отсюда сразу восстанавливаются остальные элементы, и получается, что $\chi_i(g^j) = \omega^{ij}$, где $\omega$ --- произвольный фиксированный первообразный корень $n$-й степени из единицы.

Полученная матрица --- \emph{матрицы дискретного преобразования Фурье}.

\subsection{Классификация представлений произвольных конечных абелевых групп}
Расклассифицировав таким образом представления всех циклических абелевых группы, мы, на самом деле, классифицировали вообще представления всех конечных абелевых групп.

Воспользуемся теоремой о классификации всех конечнопорождённых абелевых групп, всякая конечная абелева группа --- прямая сумма циклических групп.

Ссылаясь на
\intfact{
    Групповая алгебра $K[H \times G]$ есть $K[H] \otimes_K K[G]$~(\ref{khtimesg}).
}
мы можем получить следующее.
Пусть \[\arr{c}{\chi: H \map K^* \\ \theta: G \map K^*} \text{ --- два характера. Тогда }\chi\otimes \theta \underset{\text{у нас всё одномерно}}= \chi\theta: \arr{ccc}{H \times G &\map& K^*\\(h,g)&\mapsto&\chi(h)\theta(g)}\]
Таким образом, все характеры абелевой группы получаются перемножением всевозможных характеров циклических слагаемых из прямой суммы.

\example{
    \item Рассмотрим простейшую нециклическую группу $V = \underset{\ni h}{C_2} \oplus \underset{\ni g}{C_2} = \{1, h, g, hg\}$.
    Для неё таблица характеров
    \[\begin{array}{c|c}
          & \arr{cccc}{1 & h & g & hg} \\\hline
          \arr{c}{\chi_0 \\ \chi_1 \\\chi_2 \\\chi_3} & \arr{cc|cc}{1 & 1 & 1 & 1 \\ 1 & -1 & 1 & -1 \\\hline 1 & 1 & -1 & -1 \\ 1 & -1 & -1 & 1}
    \end{array} = \vect{1 & 1 \\ 1 & -1} \otimes \vect{1 & 1 \\ 1 & -1}\]
}

\subsection{Одномерные представления любых конченых групп}
Разумеется, все одномерные представления неприводимы.

Таким образом, описать одномерные представления --- часть задачи.

Мы умеем описывать представления абелевых групп.
Для произвольной группы $G \rightsquigarrow G^{\ab} \bydef G/[G,G]$.

Чтобы для группы получить представление, исходя из факторгруппы, надо воспользоваться \emph{инфляцией}:
пусть $H \normeq G$.
\gather{\pi: G/H \map GL(V) \\ \downarrow \\ \arr{cccc}{\tilde{\pi}: &G &\map& GL(V)\\&g &\mapsto& \pi(g + H)}}
Поскольку приводимость зависит только от образа, то инфляция неприводимого представления неприводима.

Если абелианизация нетривиальна, то таким образом получаются какие-то нетривиальные представления.
\theorem{
    Пусть $K$ --- алгебраически замкнутое поле, $\chr(K) = 0$. Тогда у конечной группы $G$ имеется $\abs{G/[G,G]}$ различных (не эквивалентных) одномерных представлений (и они являются инфляциями неприводимых одномерных представлений $G/[G,G]$).
    \provehere{
        Только что было предъявлено $|G/[G,G]|$ таких представлений.
        Обратно, если $\tilde{\pi}: G \map GL(1, K) = K^*$ --- какое-то представление, то $[G, G] \le \Ker(\tilde{\pi})$ --- это отображение в абелеву группу.

        Тогда $\tilde{\pi}$ соответствует его дефляция $\pi: G/H \map GL(1, K) = K^*$.
    }
}
\newlection{21 сентября 2016 г.}


\section{Формулировка теоремы Бернсайда --- Фробениуса, первые примеры}
$G$ --- конечная группа, $K$ --- алгебраически замкнутое поле, $\chr(K) = 0$.
На самом деле, результаты верны для поля разложения группы $G$, такого, что $\chr(K) \notdivs |G|$).

Пусть $\pi_1, \dots, \pi_s$ --- все различные (неэквивалентные) неприводимые представления $G$ над $K$.
Пусть $n_1, \dots, n_s$ и $\chi_1, \dots, \chi_s$ --- степени и характеры $\pi_1, \dots, \pi_s$ соответственно.
\intfact[Теорема Бернсайда --- Фробениуса]{\down
\numbers{
    \item $s$ --- количество классов сопряжённых элементов группы $G$.
    \item $|G| = n_1^2 + \dots + n_s^2$ (sum of squares formula)
    \item $n_i \divs |G|$, что может быть усилено до $n_i \divs |G : C(G)|$, или даже до $n_i \divs |G : A|$, где $A \normeq G$ --- произвольная нормальная подгруппа.
    Более того, $n_i \le |G : A|$, где $A \le G$ --- произвольная абелева подгруппа (но уже необязательно делит).
}
}

\subsection{Представления неабелевых групп}
Маленькими неабелевыми группами являются $S_3 = D_3$, $D_4$, $Q_8$.

Все представления $D_n$ одномерны или двумерны, а с представлениями $S_n$ всё совсем не так просто, на сегодняшний день про них известно всё, но это очень большая непростая история.

Пусть имеется перестановочное действие $G \curvearrowright X$.
Тогда ему соответствуют линейное действие на функциях $X \map K$: $G \curvearrowright K^X = \left\{\sum\limits_{x \in X}a_x \delta_x\right\}$, или на формальных комбинациях: $G \curvearrowright \left\{\sum\limits_{\substack{x \in X\\\text{почти все $a_x$ нули}}}a_x x\right\}$.

Несмотря на то, что эти записи вылядят идентичными (в случае конечного $X$), они различаются --- различаются действием $G$.

На функции $f: X \map K$ элементы $g \in G$ действуют так: $(fg)(x) = f(gx)$.
Но так как речь о левых действиях, то $(g^{-1}f)(x) = f(gx)$.

Обозначим $\bigoplus\limits_{x \in X}Kx = K[X]$ за множество формальных комбинаций $X$ с коэффициентами из $K$.

Действие $G$ переставляет базис данного векторного пространства над $K$, и перестановке базиса отвечает линейное действие на $\bigoplus\limits_{x \in X}Kx$.

Оказывается, из примера действия группы самой на себе сдвигами (трансляциями) получаются все представления групп.

Рассмотрим левое регулярное представление $G \curvearrowright G$:
\begin{align*}
    G \times G &\map G \\ g,x &\mapsto gx
\end{align*}
Ему соответствует линейное действие $G \curvearrowright K[G]$.

Действие $S_n$, переставляющее базисные элементы $V$ ($\dim V = n$) не является неприводимым.
\examples[Представления неабелевых групп]{
    \item $S_n \curvearrowright [n]$. Если $(e_1, \dots, e_n)$ --- базис $K^n$, то имеется естественное линейное действие $S_n \curvearrowright K^n$, $\sigma e_i = e_{\sigma(i)}$.

    Действие не неприводимо: здесь есть одномерное инвариантное подпространство $U = K(e_1 + \dots + e_n)$.
    Согласно теореме Машке у данного подпространства есть инвариантное дополнение $W$.
    Если подумать, то окажется, что $W = K(e_1 - e_2) + \dots + K(e_{n - 1} - e_n) = \defset{\sum\limits a_i e_i}{\sum\limits a_i = 0}$.

    $S_n$ действует на $W$, это \emph{стандартное представление} $\sigma$.
    \intfact{В характеристике нуль $\sigma$ неприводимо.
    }
    \item Конкретизируем: рассмотрим $S_3$.
    $[S_3, S_3] = A_3$. $|S_3 / [S_3, S_3]| = 2$. Таким образом, у $S_3$ два одномерных представления --- главное (единица $1_{S_3}$) и ещё одно (знак $\sgn: S_3 \map K^*$).

    Сопряжённых классов у $S_3$ три --- тип единицы (1,1,1), тип транспозиции (1,2), тип 3-цикла (3).
    Порядка классов оттуда --- 1, 3, 2 соответственно.

    Неприводимых представлений будет столько же, сколько и классов --- три. Используя sum of squares theorem, можно узнать степень третьего представления. $6 = n_1^2 + n_2^2 + n_3^2 = 1^2 + 1^2 + x^2 \then x = 2$ (ещё можно использовать, что $x \in \N$ --- число, делящее индекс центра).

    $K(e_1 - e_2) + K(e_2 - e_3)$ под действием $S_3 = \angles{(12), (23)}$ преобразуется в себя под действием матриц $(12) \mapsto \vect{-1 & 1 \\ 0 & 1}; (23) \mapsto \vect{1 & 0 \\ 1 & -1}$.
    Это ещё одно неприводимое представление $S_3$.
    При нём 3-цикл $(123)$ переходит в $\vect{0 & -1 \\ 1 & -1}$, и считая следы этих матриц, мы можем построить таблицу характеров.
    \[\begin{array}{c|ccc}
          & 1 & (13) & (123) \\\hline 1 & 1 & 1 & 1 \\ \sgn & 1 & -1 & 1 \\ \chi_{\sigma} & 2 & 0 & 1
    \end{array}\]
    У этой таблицы есть множество замечательных свойств, но они будут выведены позднее.
    \item Группа $Q_8$ задаётся копредставлением $\angles{i,j,k\middle| i^2 = j^2 = k^2 = ijk = -1} = \{\pm 1, \pm i, \pm j, \pm k\}$ (при условии $(-1)^2 = 1$).
    $|Q| = 8 = 2^3$, то есть $Q$ --- 2-группа. $C(Q) = \{\pm 1\}$.
    Таким образом, у неё четыре одномерных представления.
    Классов сопряжённых элементов в данной группе пять: $Q_8 = \{1\} \sqcup \{-1\} \sqcup \{\pm i\} \sqcup \{\pm j\} \sqcup \{\pm k\}$.
    \[\begin{array}{c|ccccc}
          & 1 ~(1) & -1 ~(1) & \pm i ~(2) & \pm j ~(2) & \pm k ~(2) \\\hline 1 & 1      & 1       & 1         & 1         & 1          \\
          \chi_1 & 1      & 1       & -1         & -1          & 1         \\
          \chi_2 & 1      & 1       & -1          & 1         & -1         \\
          \chi_3 & 1      & 1      & 1          & -1          & -1          \\
          \chi_4 & 2      & -2      & 0          & 0          & 0
    \end{array}\]
    В скобках в первой строке пишется количество элементов в соответствующем классе сопряжённых.

    Последнему представлению соответствуют матрицы Паули, которые построил Кэли: $\H = \defset{\vect{z & w \\ -\overline{w} & \overline{z}}}{z, w \in \C}$ при выборе базиса $\{1, i\}$ в $\C$ получает базис
    \[\vect{1 & 0 \\ 0 & 1} \quad \vect{i & 0 \\ 0 & -i} \quad \vect{0 & 1 \\ -1 & 0} \quad \vect{0 & i \\ i & 0}\]
    (Данное представление точное, поэтому неприводимое --- у всех одномерных представлений $-1$ лежит в ядре).
    Это и есть образы $1, i, j, k$ при неком двумерном представлении $Q_8$.
    \definition[Точное представление (faithful representation) $\pi: G \map GL(n, K)$]{$\Ker(\pi) = 1$.}
    \item Теперь посмотрим на диэдральную группу $D_n= \angles{x, y\middle|x^2 = y^2 = (xy)^n = 1}$ при $n = 4$.

    $D_4 = \{1, x, y, xy, yx, xyx, yxy, xyxy = yxyx\}$.
    Образующие отвечают симметриям квадрата относительно диагонали и серединного перпендикуляра к стороне.

    $C(D_4) = \{1, xyxy = yxyx\}$. Снова $D_4/C(D_4) = V$. Здесь таблица характеров такая
    \[\begin{array}{c|ccccc}
          & 1 ~(1) & xyxy = yxyx ~(1) & \{x, yxy\} ~(2) & \{y, xyx\} ~(2) & \{xy, yx\} ~(2) \\\hline
          1      & 1      & 1                & 1               & 1               & 1               \\
          \chi_1 & 1      & 1                & -1              & -1              & 1               \\
          \chi_2 & 1      & 1                & -1              & 1               & -1              \\
          \chi_3 & 1      & 1                & 1               & -1              & -1              \\
          \chi_4 & 2      & -2               & 0               & 0               & 0
    \end{array}\]
    Последняя строка получена, как точное представление --- симметрии квадрата в $\R^2$.
    \[x \mapsto \vect{1 & 0 \\ 0 & -1} \quad y \mapsto \vect{0 & 1 \\ 1 & 0}\]
    И хотя таблицы характеров $D_4$ и $Q_8$ одинаковы, но группы неизоморфны, и различие заключается в том, что у $D_4$ есть двумерное представление над $\R$, а у $Q_8$ нет.
}


\section{Соотношения ортогональности Шура (лемма Шура в матричной форме)}
Пусть $\pi: G \map GL(U), \rho: G \map GL(V)$ --- два неприводимых представления группы $G$ над одним и тем же полем $K$.

Пусть $\phi: U \map V$ --- произвольное $K$-линейное отображение.
Сопоставим ему усреднение \[\phi_0 = \frac{1}{|G|}\sum\limits_{g \in G}\rho_g \phi \pi_g^{-1}\] это уже $K[G]$-линейное отображение, или сплетающий оператор.

Пусть теперь $K$ алгебраически замкнуто, $\chr(K) = 0$ (на самом деле достаточно, чтобы характеристика не делила порядок группы).
\lemma[Лемма Шура]{\down
\bullets{
    \item Если $U \ncong V$, то $\phi_0 = 0$.
    \item Если же $\pi = \rho$ (в частности, $U = V$), то тогда $\phi_0$ --- гомотетия с коэффициентом $\frac{\tr(\phi_0)}{\dim V} = \frac{\tr(\phi)}{\dim V}$.}
\provehere{
    $\phi_0 = \vect{\lambda & \cdots & 0 \\ \vdots&\ddots &\vdots \\ 0 & \cdots & \lambda}$, а при $\pi = \rho$ матрица сопрягается (и след не меняется).
}
}
Выберем в $U$ базис $u_1, \dots, u_m$ и в $V$ базис $v_1, \dots, v_n$.
Базисом линейных отображений $U \map V$ являются $\phi_{i,j}: \arr{ccc}{U &\map& V\\u_h &\mapsto& \all{v_i,&h = j \\ 0,&h \ne j}}$.
Матрица $\phi_{i,j}$ в данных базисах равна $e_{i,j}$.
\newlection{21 сентября 2016 г.}
Подставим в качестве $\phi = e_{i,j} \in M(n, m, K)$.

Для представлений $\pi$ и $\rho$ в данных базисах определены \emph{матричные элементы}
\begin{align*}
    \pi_{i,j}: G \map K \\ g \mapsto \pi(g)_{i,j}
\end{align*}
и аналогично для $\rho$:
\begin{align*}
    \rho_{k,l}: G \map K \\ g \mapsto \rho(g)_{k,l}
\end{align*}

Усредним $\phi = e_{i,j}$, получится некое $\phi_0$.
Ранее записанная лемма Шура говорит о том, что либо $\phi_0 = 0$, либо там почти все элементы равны нулю, а остальные равные $\frac{\tr(\phi)}{\dim V}$.

Получается следующая теорема (для неприводимых представлений)
\theorem[Соотношения ортогональности Шура]{\label{schur-orthogonality}\down
\numbers{
    \item Если $\pi \nsim \rho$, то $\forall i,j,k,l: \frac{1}{|G|}\sum\limits_{g \in G}\pi_{i,j}(g)\rho_{k,l}(g^{-1}) = 0$.
    \item Если $\pi = \rho$, то $\frac{1}{|G|}\sum\limits_{g \in G}\pi_{i,j}(g)\pi_{k,l}(g^{-1}) = \frac{1}{\deg(\pi)}\delta_{i,l}\delta_{j,k}$.
}
Для поля $\C$ можно устроить эрмитово скалярное произведение $C: K^G \times K^G \map \C$
    \[C(\chi, \theta) = \frac{1}{|G|}\sum\limits_{g \in G}\chi(g)\overline{\theta(g)}\]
    Чаще всего мы будем вычислять скалярное произведение от характеров.

    Тогда теорема говорит о том, что все матричные элементы $\pi_{i,j}$ для всех неприводимых представлений $\pi$ образуют ортогональный базис пространства $\C^G$ относительно скалярного произведения $C$.
    Для компактных групп это называется теоремой Петера --- Вейля.
    \provehere{Написано выше.}
}


\section{Первое соотношение ортогональности}
Пусть $\chi, \theta \in K^G$.
Определим уже не эрмитовское, а симметрическое скалярное произведение
\[B(\chi, \theta) = \frac{1}{|G|}\sum\limits_{g \in G}\chi(g)\theta(g^{-1})\]
\lemma{Оказывается, что если $\chi, \theta$ --- \textbf{характеры} конечномерных представлений над $\C$, то $C(\chi, \theta) = B(\chi, \theta)$.
Тем не менее, это разные скалярные произведения (одно эрмитово, другое симметрическое).

\provehere{
    В самом деле, если $\pi$ --- унитарное представление, то $\pi_g^{-1} = \overline{\pi_g}^t$.
    Любое представление эквивалентно унитарному (теорема Машке над $\C$).

    Пусть $\Image(\pi) \in U(n, \C)$.
    Тогда $B(\pi_{i,j}, \pi_{k,l}) = \frac{1}{|G|}\sum\limits_{g \in G}\pi_{i,j}(g)\pi_{k,l}(g^{-1}) = \frac{1}{|G|}\sum\limits_{g \in G}\pi_{i,j}(g)\overline{\pi_{l,k}(g)} = C(\pi_{i,j}, \pi_{l,k})$.
    В частности, пусть $(\pi^{(m)})_{1 \le m \le s}$ --- все неприводимые представления.
    Тогда набор \[\defset{\pi_{i,j}^{(m)} \cdot \sqrt{\deg (\pi^{(m)})}}{1 \le m \le s, 1 \le i, j \le \deg \pi^{(m)}}\] ортонормирован.

    Осталось заметить, что характер --- сумма собственных чисел (которые корни из единицы), стоящих на диагонали, а для корней из единицы $\omega: \overline{\omega} = \omega^{-1}$.
}}
\theorem[Первое соотношение ортогональности]{
    Если $\chi, \theta$ --- характеры неприводимых представлений, то $B(\chi, \theta) = \delta_{\chi,\theta}$.
    \provehere{
        Используем~(\ref{schur-orthogonality}). Положим $\chi = \pi_{1,1} + \dots + \pi_{n,n}$ и $\theta = \rho_{1,1} + \dots + \rho_{n,n}$.

        Если $\pi \nsim \rho$, то всегда $B(\pi_{i,i}, \rho_{j,j}) = 0$.

        Если же $\pi \sim \rho$, то можно сопрячь матрицу не меняя след, считаем, что $\pi = \rho$.
        Тогда $B(\chi, \theta) = \underbrace{\frac{1}{\deg(\pi)} + \cdots + \frac{1}{\deg(\pi)}}_{\deg(\pi)} = 1$.
    }
}
Вспомним, что характеры постоянны на классах сопряжённости, иначе говоря, \emph{центральные функции} на $G$.
\corollary{
    Характеры неприводимых представлений линейно независимы.

    \provehere{
        Напишем линейную зависимость $\lambda_1 \chi_1 + \cdots + \lambda_t \chi_t$.
        По очереди скалярно перемножая с $\chi_i$, получаем $\lambda_iB(\chi_i, \chi_i) = 0$, откуда все коэффициенты в зависимости нулевые.
    }
}
\corollary{
    Количество различных неприводимых характеров не превосходит количество классов сопряжённых элементов.
    \provehere{
        Прямо следует из линейной независимости.
    }
}

Дальше уже $\chr(K) = 0$, в последющих следствиях не подойдёт не делящая порядок группы.

\corollary{
    Пусть $\pi$ --- любое представление $G$.
    Тогда кратность вхождения неприводимого $\pi_i$ в $\pi$ равна $B(\chi_{\pi}, \chi_i)$ (где $\chi_i \bydef \chi_{\pi_i}$).
    \provehere{
        По теореме Машке $\pi$ есть сумма неприводимых представлений:
        \[\pi = \pi_1^{\oplus m_1} \oplus \dots \oplus \pi_t^{\oplus m_t}\]
        Следовательно, $\chi_\pi = m_1 \chi_1 + \dots + m_t \chi_t$.
    }
}
\corollary[Теорема Ремака --- Крулля --- Шмидта]{\label{Krull–Schmidt-Remak-theorem}
Разложение представления на неприводимые определено однозначно с точностью до изоморфизма.
}
\note{
    Теорема доказывается в гораздо меньшей общности, чем она верна, и используются сильно более сильными средствами, чем те, которые нужны, но что поделать.
}
\corollary[Теорема Фробениуса]{
    $\pi \sim \rho$, если $\chi_\pi = \chi_\rho$. Ещё раз отметим, что это верно \textbf{только в характеристике нуль}.
    \provehere{Всякое неприводимое представление входит в $\pi$ и $\rho$ с равной кратностью.}
}
\corollary{
    $\forall$ представления $\pi: B(\chi_\pi, \chi_\pi) = 1 \iff \pi$ неприводимо.
    \provehere{
        Пусть $\pi = \pi_1^{\oplus m_1} \oplus \dots \oplus \pi_t^{\oplus m_t}$.
        Отсюда следует, что $B(\chi_\pi, \chi_\pi) = m_1^2 + \dots + m_t^2$.
    }
}
\corollary[Ортогональность первой строке таблицы характеров]{
    Если $\chi \ne 1$ --- характер неприводимого представления, то $\sum\limits_{g \in G}\chi_g = 0$.
}

\precaution{Основной ошибкой начинающих является то, что при подсчёте $B(\chi, \theta) = \sum\limits_{g \in G}\chi(g)\theta(g^{-1})$ взятие обратного забывается.
Всякому классу сопряжённых элементов $C \subset G$ можно сопоставить другой класс $C^{-1} \subset G$, и в общем случае совсем необязательно, что $C = C^{-1}$.
}


\section{Разложение представление на неприводимые. Sum of squares formula}
\label{sum-of-squares-theorem}
Рассмотрим регулярное представление $G$ --- действие $G$ слева на групповой алгебре.
\begin{align*}
    \text{reg}: G \curvearrowright K[G] &\map K[G]\\ g, \sum\limits_{h \in G}a_h h &\mapsto a_h gh
\end{align*}
Посчитаем характер данного представления.

Пусть $G \curvearrowright X$ --- действие.
С ним связано линейное представление на пространстве с базисом $X$ $\pi: G \curvearrowright K[X]$.
\theorem[Fixed points formula]{
    $\chi_\pi(g) = |\Fix_X g|$ (где $\Fix_X g = \defset{x \in X}{gx = x}$).
    \provehere{
        Матрицы, в которые отправляются элементы $G$ --- матрицы-перестановки.
        След такой матрицы равен количеству единичек на диагонали, то есть количеству неподвижных точек.
    }
}
\corollary{
    $\chi_{\text{reg}} = |G| \cdot \delta_{g,1_G}$.
    \provehere{
        Только единица оставляет какие-то элементы на месте, и для неё все точки неподвижны.
    }
}
Пусть $K$ --- алгебраически замкнутое поле, $\chr(K) = 0$.
\theorem{
    Каждое неприводимое представление $\pi_i$ группы $G$ входит в разложение регулярного с кратностью $n_i = \deg(\pi_i)$.
    \[\text{reg} = \pi_1^{\oplus n_1} \oplus \dots \oplus \pi_s^{n_s}\]
    \provehere{
        Вычислим $B(\chi_{\text{reg}, \chi_i}) = \sum\limits_{g \in G}\chi_{\text{reg}}(g)\chi_i(g^{-1}) = \frac{1}{|G|}\cdot |G| \cdot \chi_i(1) = n_i$.
    }
}
\note{
    Это же следует и из теоремы Веддербарна --- Артина, причём даже не в характеристике нуль.
    В матричном кольце $M(n, K)$ ровно $n$ неприводимых подмодулей --- столбцы $K^n$ (?)
}
\corollary{
    $|G| = n_1^2 + \cdots + n_t^2$.
    \provehere{
        Регулярное представление раскладывается в сумму $n_1$ неприводимых степени $n_1$, $n_2$ неприводимых степени $n_2$, \ldots
    }
}
\corollary{
    Функции $\pi^{(i)}_{j,k}$ образуют базис пространства $K^G$.
    \provehere{
        Они линейно независимы, и их количество равно $|G| = \dim_K(K^G)$.
    }
}


\section{Второе соотношение ортогональности (для столбцов)}
Будем рассматривать представления $G$ над $\C$.

Пусть $\chi_1, \dots, \chi_t$ --- характеры неприводимых представлений.

В первой соотношении были фиксированы две строки таблицы характеров, и суммирование было по столбцам.
Сейчас сделаем наоборот.

\theorem{
    Пусть $h, g \in G$, предположим, что нам уже известно, что $s = t$. \[\sum\limits_{i = 1}^{t}\chi_i(h)\overline{\chi_i(g)} = \all{|C_G(g)|,&g \sim h \\ 0,& g \nsim h}\]
    \provehere{
        Пусть $1, g_2, \dots, g_s$ --- представители классов сопряжённых элементов, $\chi_1, \dots, \chi_s$ -- различные неприводимые характеры.

        Составим матрицу $A = (a_{i,j})_{1 \le i,j \le s}$, где $a_{i,j} = \sqrt{\left|g_j^G\right|}\cdot \chi_i(g_j)$.

        Обозначим $m_h \coloneqq |g_h^G|$.
        Первое соотношение ортогональности выглядит так: $\frac{1}{|G|}\sum\limits_{h = 1}^{s}m_h\chi_i(g_h)\overline{\chi_j(g_h)} = \delta_{i,j}$

        Если для матрицы $A \in M(n, \C)$ обозначить за $A^*$ её эрмитовски сопряжённую $A^* \bydef \overline{A}^t$, то видно, что из первого соотношения ортогональности $\frac{1}{|G|}A \cdot A^* = E$.
        Отсюда сразу получается $\frac{1}{|G|}A A^* = e$.
        \[\frac{1}{|G|}\sum\limits_{h = 1}^{s}\sqrt{\abs{g_i^G}\cdot\abs{g_j^G}}\cdot \overline{\chi_h(g_i)}\chi_h(g_j) = \frac{1}{|G|}\sqrt{\abs{g_i^G}\cdot\abs{g_j^G}}\sum\limits_{h = 1}^{s}\overline{\chi_h(g_i)}\chi_h(g_j) = \delta_{i,j}\]
        При $i \ne j$ получаем $\sum\limits_{h = 1}^{s}\overline{\chi_h(g_i)}\chi_h(g_j) = 0$, иначе $i = j$, и так как $|g_i^G| = |G : C_G(g_i)|$, то $\frac{|g_i^G|}{|G|} = \frac{1}{|C_G(g^G)|}$, то получается искомая формула.
    }
}
\newlection{28 сентября 2016 г.}


\section{Усреднение с весом (averaging with weight)}
Раньше все усреднения использовали меру Хаара --- все элементы группы имели одинаковый вес $\forall g \in G: \mu(g) = \frac{1}{|G|}$.

Оказывается, можно проделать то же самое для гораздо более широкого класса мер.

Пусть $\phi \in \Hom(U, V)$.
Усредним его с помощью $f$, получив $G$-инвариантное $\phi_f$.
В качестве $f$ здесь могут выступать центральные функции, поясним это ниже.

Рассмотрим пространство центральных функций на $G$: $cf_K(G) \bydef \defset{f \in K^G}{ \forall g, h \in G: f(h^g) = g(h)}$.

Также рассмотрим представление $\pi: G \map GL(V)$.
Пусть $f \in cf_K(G)$.

Определим усреднение $\pi_{(f)} = \frac{1}{|G|}\sum\limits_{g \in G}f(g)\pi_g$.
\note{
    Если $f$ --- необязательно центральная функция, то при отождествлении $G^K$ и $K[G]$ ($\delta_g \leftrightarrow g$) получаем, что $\pi_{(g)} = \pi_g$.
}
\lemma{Утверждается, что $\pi_{(f)}\in \End_{K[G]}(V)$, то есть $\pi_{(f)}$ коммутирует со всеми $\pi_g$.
\provehere{
    Проверим, что $\forall g \in G: \pi_g \pi_{(f)} = \pi_{(f)}\pi_g$.

    \[\forall x \in G, u \in V: \pi_{(f)}(xu) = \frac{1}{|G|}\sum\limits_{g \in G}f(g)\underbrace{}_{xx^{-1}}g(xu) = x\frac{1}{|G|}\sum\limits_{g \in G}f(xgx^{-1})(x^{-1}gx)u = x \pi_{(f)}(u)\]
}}
\note{
    Обратное тоже верно в таком смысле: если для любого представления усреднение по функции $f$ --- $G$-инвариантно, то усреднение производится по центральной функции.
}
\lemma{
    Пусть $f \in cf(G)$, пусть $\pi$ --- неприводимое представление $G$ над алгебраически замкнутым полем $K$, $\chr K = 0$.
    Положим $\deg \pi = n$.

    Тогда $\pi_{(f)}$ --- гомотетия с коэффициентом $\frac{1}{n}B(f, \chi_{\pi^*})$.
    \provehere{
        $\pi_{(f)}$ --- гомотетия по лемме Шура. Посчитаем коэффициент $\lambda = \frac{\tr\left(\pi_{(f)}\right)}{n}$ гомотетии.
        \[\lambda = \frac{1}{n}\frac{1}{|G|}\sum\limits_{g \in G}f(g)\underbrace{\tr(\pi_g)}_{\chi_{\pi}(g)} = \frac{1}{n}\frac{1}{|G|}\sum\limits_{g \in G}f(g)\cdot\chi_{\pi^*}(g^{-1})\]
    }
}


\section{Количество неприводимых представлений конечной группы}
Как обычно, $K$ алгебраически замкнуто, $\chr(K) = 0$ (на самом деле верно и для поля разложения, $\chr(K)\notdivs|G|$).
\theorem{
    Количество неприводимых различных (неэквивалентных) представлений $G$ над $K$ равно количеству классов сопряжённых элементов в $G$.
    \provehere{
        Пусть $c_1, \dots, c_s$ --- классы сопряжённых элементов, $\pi_1, \dots, \pi_t, \chi_1, \dots, \chi_t$ --- неприводимые представления и их характеры соответственно.
        Характеры --- ортонормированная система функций, по отношению к билинейной форме $B$, и они являются центральными функциями.

        Базис $cf_K(G)$ --- характеристические функции классов сопряжённости $\delta_{c_i}: x \mapsto \all{1,&x \in c_i \\ 0,&x \notin c_i}$.

        Так как $\chi_1, \dots, \chi_t$ линейно независимы, то $t \le s$.

        Чтобы доказать, что $t = s$, надо убедиться, что $\chi_1, \dots, \chi_t$ --- полная ортонормированная система, то есть нет никакой ненулевой центральной функции, ортогональной всем $\chi_t$.
        \[\forall f \in cf_K(G): B(f, \chi_i) = 0 \overset{?}\then f = 0\]
        Пусть $\forall i: B(f, \chi_i) = 0$. Тогда при усреднении получаем $\forall i: (\pi_i)_{(f)} = 0$.

        Так как каждое представление раскладывается в прямую сумму неприводимых, то вообще любое представление усредняется в ноль, например, регулярное $\text{reg}: G \map GL(K[G])$.
        Пусть $(e_g)_{g \in G}$ --- базис $K[G]$.
        Таким образом, $\text{reg}_{(f)}(e_1) = \frac{1}{|G|}\sum\limits_{g \in G}f(g)\text{reg}_g(e_1) = \frac{1}{|G|}\sum\limits_{g \in G}f(g)e_g$.
        Но так как $e_g$ --- базис, то все коэффициенты равны нулю.

        Действительно, $f$ оказалась равна нулю, откуда $\chi_i$ --- полная система, и $s = t$.
    }
}
\newlection{19 октября 2016 г.}


\section{Дальнейшие конструкции над представлениями}
\bullets{\label{further-constructions}
\item Тензорное произведение представлений

Пусть $\pi: G \map GL(U), \rho: G \map GL(V)$ --- два представления одной группы.
    \[(\pi, \rho)\rightsquigarrow\pi \otimes\rho: \arr{ccc}{G &\map& GL(U \otimes V)\\g&\mapsto&\pi_g \otimes_{K} \rho_g}\]
    \note{
        Это совсем не сермяжная истина, это эквивалентно тому, что диагональ определена так: $\Delta: \arr{ccc}{K[G]&\map&K[G]\\g &\mapsto& g \otimes g}$.
        Для алгебр Ли будет всё совсем иначе.
    }
    Если ввести базис, то окажется, что $(\pi \otimes \rho)_g = \pi_g \otimes \rho_g$, где $\otimes$ --- кронекеровское произведение матриц.
    А мы знаем, что $\tr(x \otimes y) = \tr(x) \cdot \tr(y)$, то есть если перемножить характеры двух каких-то представлений, то получится тоже характер какого-то представления: $\chi_\pi \cdot \chi_\rho = \chi_{\pi \otimes \rho}$.

    Конечно, $\pi \otimes \rho$ необязательно неприводимо, но если одно из $\pi$ и $\rho$ одномерно, а другое неприводимо, то результат --- тоже неприводимое произведение (какое-то другое).

    Именно так в общем случае строятся представления --- берутся все представления, которые можно построить, дальше их тензорные произведения (например, степени), они снова раскладываются на неприводимые, и так теоретически может найтись всё, но это надо доказывать.
    \item Наружное тензорное произведение (outward tensor product).

    Пусть $\pi: H \map GL(U), \rho: G \map GL(V)$ --- два представления \textbf{разных групп}.

    Сопоставим им представление $\pi \boxtimes \rho: \arr{ccc}{H \times G &\map& GL(U \otimes V)\\(h,g)&\mapsto&\pi_h \otimes \rho_g}$.
    В матрицах это опять же кронекеровское произведение матриц.
    \note{
        В сравнении с предыдущим пунктом получаем $\pi \otimes \rho = \restricted^{G \times G}_{G}(\pi \boxtimes \rho)$.

        Их не надо путать, произведение неприводимых непременно неприводимо только если произведение наружное, а иначе --- как правило приводимо.
    }
    \item $n$-я внешняя степень.

    Пусть $\pi: G \map GL(V)$. Сопоставим ему $\bigwedge^m(\pi): \arr{ccc}{G&\map&GL(\bigwedge^m(V))\\g&\mapsto&\bigwedge^m(\pi_g)}$.
    По линейности определяется \[\bigwedge\!\!{}^m(\pi)(u_1 \wedge \dots \wedge u_m) = \pi_g(u_1) \wedge \cdots \wedge \pi_g(u_m)\]
    то есть $\bigwedge^m(\pi)_g = \bigwedge\!\!{}^m(\pi_g)$.
    Можно посчитать $\chi_{\bigwedge^2(\pi)}(g) = \frac{1}{2}(\chi_\pi(g)^2 - \chi_\pi(g^2))$.
    \item Точно так же определяется $n$-я симметрическая степень.

    Пусть $\pi: G \map GL(V)$. Сопоставим ему $S^m(\pi): \arr{ccc}{G&\map&GL(S^m(V))\\g&\mapsto&S^m(\pi_g)}$.
    По линейности определяется \[S^m(\pi)(u_1 \wedge \dots \wedge u_m) = \pi_g(u_1) \proddots \pi_g(u_m)\]
    то есть $S^m(\pi)_g = S^m(\pi_g)$.

    Полезно помнить формулу для симметрического квадрата $\chi_{\bigwedge^2(\pi)}(g) = \frac{1}{2}(\chi_\pi(g)^2 + \chi_\pi(g^2))$.

    Видно, что $\bigwedge^2(\pi)\oplus S^2(\pi) = \pi \otimes \pi$, и действительно $\chi_{\bigwedge^2(\pi)} + \chi_{S^2(\pi)} = \chi_{\pi \otimes \pi}$.

}


\section{Представления прямого произведения групп}
\label{khtimesg}
На самом деле говорится, что $K[H \times G] = K[H] \otimes_K K[G]$.
\note{На модуле $A \otimes B$ умножение достаточно вводить на разложимых тензорах, и обычно его определеяют по формуле $(a_1 \otimes b_1)(a_2 \otimes b_2) = (a_1a_2 \otimes b_1b_2)$. Таким образом, тензорное произведение алгебр --- алгебра.
Но мы докажем не это.}
\theorem{
    Любое неприводимое представление группы $H \times G$ над алгебраически замкнутым полем характеристики нуль $K$ имеет вид $\pi \boxtimes \rho$, где $\pi, \rho$ --- неприводимые представления $H$ и $G$ соответственно. Обратно, любое такое $\pi \boxtimes \rho$ неприводимо.
    \provehere{
        Сначала докажем, что $\pi \boxtimes \rho$ неприводимо для неприводимых $\pi,\rho$.

        Была определена билинейная форма $B_G: K^G \times K^G \map K$, $B(\chi,\theta) = \frac{1}{|G|}\sum\limits_{g \in G}\chi(g)\theta(g^{-1})$.
        Была доказана лемма, что характер $\chi$ --- характер неприводимого представления, если $B(\chi,\chi) = 1$.

        Посчитаем $B_{H \times G}(\chi_{\pi \boxtimes \rho},\chi_{\pi \boxtimes \rho})$.
        Так как $\chi_{\pi \boxtimes \rho}(h, g) = \chi_\pi(h)\otimes \chi_\rho(g)$, то
        \multline{B_{H \times G}(\chi_{\pi \boxtimes \rho},\chi_{\pi \boxtimes \rho}) = \frac{1}{|H \times G|}\sum\limits_{h \in H, g \in G}\chi_\pi(h)\chi_\rho(g)\chi_\pi(h^{-1})\chi_\rho(g^{-1}) =\\= \frac{1}{|H|}\sum\limits_{h \in H}\chi_\pi(h)\chi_\pi(h^{-1})\cdot\frac{1}{|G|}\sum\limits_{g \in G}\chi_\rho(g)\chi_\rho(g^{-1}) = B_H(\chi_\pi,\chi_\pi)B_G(\chi_\rho,\chi_\rho) = 1}

        Но так как получено ровно столько представлений, сколько и есть классов сопряжённых элементов, то из соображений количества больше представлений нет.
        Та же выкладка показывает, что получены неэквивалентные представления:
        \multline{B_{H \times G}(\chi_{\pi \boxtimes \rho},\chi_{\pi' \boxtimes \rho'}) = \frac{1}{|H \times G|}\sum\limits_{h \in H, g \in G}\chi_\pi(h)\chi_{\rho'}(g)\chi_{\pi'}(h^{-1})\chi_\rho(g^{-1}) =\\= \frac{1}{|H|}\sum\limits_{h \in H}\chi_\pi(h)\chi_{\pi'}(h^{-1})\cdot\frac{1}{|G|}\sum\limits_{g \in G}\chi_\rho(g)\chi_{\rho'}(g^{-1}) = B_H(\chi_\pi,\chi_{\pi'})B_G(\chi_\rho,\chi_{\rho'}) = \delta_{\pi,\pi'}\cdot\delta_{\rho,\rho'}}
    }
}
\note{Теорема верна и при $K$ --- поле разложения, $\chr K \notdivs |G|$.}
\newlection{24 октября 2016 г.}


\section{Свойства целочисленности представлений}
Пусть $\pi: G \map GL(n, \C)$ --- неприводимое представление ($\deg \pi = n$).
Для любого $g \in G: g^m = 1 \then \pi_g^m = \pi_{g^m} = \pi_1 = \id$, то есть все собственные числа $\pi_g$ --- корни степени $m$ из едиинцы.
В частности, $\chi_\pi(g) \in \Z\left[\sqrt[m]{1}\right] \le \A$.

Пусть $C \subset G$ --- класс сопряжённых элементов.
\lemma{
    $\sum\limits_{g \in C}\frac{\chi_\pi(g)}{\deg(\pi)} \in \A$.
    \provehere{
        $\sum\limits_{g \in C}\pi_g$ является центральным элементом:
        \[\sum\limits_{g \in C}\pi_g \cdot \pi_x = \pi_x \cdot \sum\limits_{g \in C}\pi_g\]
        Тем самым (лемма Шура), $\exists \lambda \in \C: \sum\limits_{g \in C}\pi_g = \lambda \id$.
        Посчитаем след: $\tr\left(\sum\limits_{g \in C}\pi_g\right) = n \lambda$, но на самом деле $\lambda \in \A$, например, так как это корень характеристического многочлена ${\sum\limits_{g \in C}\pi_g}$.
    }
}

\corollary{
    Над $\C$ степень любого неприводимого представления делит порядок группы.
    \provehere{
        Пусть $\pi$ --- неприводимое представление. Пусть $\chi = \chi_\pi$. Тогда $B(\chi,\chi) = 1\iff\sum\limits_{g \in G}\chi(g)\chi(G^{-1}) = |G|$.
        Теперь поделим обе части равенства на $n$.
        \[\underbrace{\sum\limits_{C \subset G}\chi(C)}_{\in \A}\underbrace{\sum\limits_{h \in C}\frac{\chi(h)}{n}}_{\in \A} = \frac{|G|}{n}\]
        Отсюда $\frac{|G|}{n} \in \A$, но по лемме Гаусса $\A \cap \Q = \Z$.
    }
}

\theorem[Следствие из предыдущей]{
    Над $\C$ степень $n$ любого неприводимого представления $\pi: G \map GL(V)$ делит индекс центра.
    \provehere{
        Рассмотрим $\pi^{\boxtimes m}: \underbrace{G \times \dots \times G}_{m} \map GL(V^{\otimes m})$.
        Оно неприводимо, но точным не является.
        Пусть $H \coloneqq \defset{(h_1, \dots, h_m)}{h_1 \proddots h_m = 1}\le C(G) \times \dots \times C(G)$.

        Можно рассмотреть дефляцию представления $\pi^{\boxtimes m}: G/H \map GL(V ^{\otimes m})$.
        Она всё ещё неприводима --- образ остался прежним.

        Порядок факторгруппы равен $\frac{|G|^m}{|C(G)|^{m-1}} = |G| \cdot |G : C(G)|^{m-1}$, степень представления равна $\deg(\pi)^{m}$.
        Тогда \[(\deg(\pi))^m \bigdivs |G| \cdot |G : C(G)|^{m-1}\]и так как это верно для любого $m \in \N$, то $\deg(\pi)\bigdivs |G : C(G)|$.
    }
}

\theorem{
    Пусть $K = \C$, $A \le G$ --- абелева подгруппа. Тогда степень неприводимого представления не больше $|G : A|$, а если $A \normeq G$, то $\deg(\pi)\bigdivs |G : A|$.
}


\section{Индуцированные представления}
Пусть $H \normeq G$.
Тогда можно построить инфляцию представления $\pi: G/H \map GL(V)$, это будет представление \begin{align*}
                                                                                                \tilde{\pi}: G &\map GL(V)g \mapsto \pi_{gH}
\end{align*}
Обратно, если $\pi: G \map GL(V)$ и $H \le \Ker(\pi)$, то есть дефляция \begin{align*}
                                                                            \tilde{\pi}: G/H &\map GL(V)gH \mapsto \pi_{g}
\end{align*}

Теперь пусть $H \le G$ --- просто подгруппа.
По представлению $\pi: G \map GL(V)$ можно построить ограничение $\restricted^G_H(\pi)$.
Однако совсем необязательно ограничение неприводимого представления неприводимо.

Чтобы пройти в обратную сторону, построим по представлению подгруппы представление группы. $\pi: H \map GL(V)$.
Построим \emph{индуцированное представление} $\induced_H^G(\pi): G \map ?$
В конструкции индуцированного представления участвует не только само представление $\pi$, но и децствие $G \curvearrowright G/H$.

\subsection{Компактная индукция (compact induction)}
Если дано представление $\pi: H \map V$, то $V$ является $K[H]$ модулем.
Чтобы превратить его в $K[G]$ модуль, можно взять тензорное произведение
\[K[G] \otimes_{K[H]} V\]
Тут стоит остановиться и пояснить, что есть тензорное произведение над некоммутативным кольцом.

Пусть $R$ --- необязательно коммутативное кольцо, $U$ --- правый $R$-модуль, $V$ --- левый $R$-модуль.
Тогда $U \otimes_R V$ --- абелева группа со сбалансированным биаддитивным отображением $\phi: U \times V \map U \otimes_R V$, удовлетворяющая следующему универсальному свойству:

Для любой абелевой группы $A$, любого сбалансированного биаддитивного $\psi: U \otimes_R V \map A$: $\exists! \eta : U \otimes_R V \map A$ --- гомоморфизм абелевых групп.
% https://q.uiver.app/#q=WzAsMyxbMCwwLCJVIFxcdGltZXMgViJdLFsxLDEsIkEiXSxbMiwwLCJVIFxcb3RpbWVzX1IgViJdLFswLDEsIlxccHNpIl0sWzAsMiwiXFxwaGkiXSxbMiwxLCJcXHRoZXRhIiwyLHsic3R5bGUiOnsiYm9keSI6eyJuYW1lIjoiZGFzaGVkIn19fV1d
\[\begin{tikzcd}[ampersand replacement=\&]
{U \times V}
      \&\& {U \otimes_R V} \\
      \& A
      \arrow["\psi", from=1-1, to=2-2]
      \arrow["\phi", from=1-1, to=1-3]
      \arrow["\theta"', dashed, from=1-3, to=2-2]
\end{tikzcd}\]
\definition[Сбалансированное отображение $\phi$]{$\phi(u\lambda, v) = \phi(u, \lambda v)$.}
\definition[Биадддитивное отображение $\phi$]{$\phi(u, v_1 + v_2) = \phi(u, v_1) + \phi(u, v_2)$ и $\phi(u_1 + u_2, v) = \phi(u_1, v) + \phi(u_2, v)$.}
Далее сюда надо перенести конструкцию.

Если один из модулей был бимодулем, то тензорное произведение останется модулем.

\definition[Индуцированное представление]{
    $\induced_H^G(V) = K[G] \otimes_{K[H]} V$.
}
Заметим, что $K[G] \otimes_{K[H]} V$ --- фактормодуль модуля $K[G] \otimes_K V$.

Если $v_1, \dots, v_n$ --- базис $V$ над $K$, то над $K$ базисом $K[G] \otimes_K V$ является система $g \otimes v_i$.
При появлении дополнительных соотношений стало возможным переносить элементы из $H$ по другую сторону $\otimes$: $gh \otimes v = g \otimes hv$.

Пусть $T = \{x_1, \dots, x_t\}$ --- правая трансверсаль к $H$ в $G$, то есть $\forall g \in G: \exists! i = 1..t, h \in H: g = x_i h$.
\[G = x_1 H \sqcup \dots \sqcup x_t H\]
\lemma{
    Базисом $K[G] \otimes_{K[H]} V$ при фиксированной трансверсали $T = (x_1, \dots, x_t)$ и базисе $(v_1, \dots, v_n)$ являются $x_i \otimes v_j$.

    В частности, $\dim_K(\induced^G_H(V)) = |G : H| \dim_K(V)$.
}
\newlection{26 октября 2016 г.}
Построение индуцированного модуля выглядело так: строится $G$-модуль $V = \induced_H^G(U)$, такой, что
\numbers{
    \item $U \le V$.
    \item $V = x_1 U\oplus \dots \oplus x_t U$.
}
Без упоминания тензорного произведения весьма утомительно проверять, что данная конструкция не зависит от выбора представителей.

\subsection{Полная индукция (complete induction)}
В то время как компактная индукция является аналогом прямой суммы, полная индукция является аналогом прямого произведения, то есть для бесконечных групп конструкция --- куда больше.
Для конечных групп же конструкции изоморфны, что вскоре будет показано.

Рассмотрим множество $H$-инвариантных функций, покамест обозначим его с большой буквы
\[\text{Ind}_H^G(U) = \defset{f: G \map U}{\forall h \in H, x \in G: f(hx) = \pi_h(f(x))}\]
Теперь устроим действие $G \curvearrowright \text{Ind}^G_H(U)$ следующим образом:
\[\forall x, g \in G, f \in \text{Ind}_H^G(U): (gf)(x) = f\left(xg\right)\]
Так как действия на $x$ слева и справа независимы, то определение корректно.
Это действительно действие:
\[g_1(g_2f(x)) = g_2f(xg_1) = f((x g_1)g_2) = f(x (g_1 g_2))\]

$\text{Ind}_H^G(U)$ можно рассматривать, как $K[G]$-модуль, найдём базис данного пространства функций.

В отличие от компактной индукции, где элементы $H$ действовали на $G$ справа, здесь всё наоборот, поэтому нам пригодится левая трансверсаль к $H$ в $G$.
Чтобы всё было согласовано, построим её по правой трансверсали, взяв обратные: $T^{-1} = \{x_1^{-1}, \dots, x_t^{-1}\}$.
\[G = H x_1^{-1} \sqcup \dots \sqcup H x_t^{-1}\]
Введём функции \begin{align*}
                   f_{x_i, u}: G &\map U \\g &\mapsto \all{\pi_h(u),&g = hx_i^{-1} \in Hx_i^{-1} \\ 0,& g \notin Hx_i^{-1}}
\end{align*}
Если $(u_1, \dots, u_n)$ --- базис $U$, то набор функций $f_{x_i,u_j}$ является базисом $\text{Ind}_H^G(U)$.

Видим, что базисы $\induced_H^G$ и $\text{Ind}_H^G$ равномощны, и пишутся очень похоже.
Векторные пространства $\induced_H^G(U)$ и $\text{Ind}_H^G(U)$ изоморфны, но \textbf{для конечных групп} они ещё и являются эквивалентными представлениями.

\theorem{
    Если $|G| < \infty, H \le G$, $U$ --- $H$-модуль, то $\induced_H^G(U) \cong \text{Ind}_H^G(U)$, как $G$-модули.
    \provehere{
        Изоморфизм можно устроить на базисе так: $x_i \otimes u_j \leftrightarrow f_{x_i,u_j}$.
        $G$ действует на $x_i \otimes u$ так: пусть $gx_i = x_j h$ для $x_j \in T, h \in H$. Тогда
        \[g(x_i \otimes u) = (gx_i) \otimes u = (x_j h) \otimes u = x_j \otimes\pi_h(u)\]
        С другой стороны, если подействовать тем же элементом $g$ на $f_{x_i,u}$, то получится вот что:
        \[gf_{x_i,u}(x) = f_{x_i, u}(x g) = \all{\pi_{b}(u), &xg = b x_i^{-1} \in H x_i^{-1}\\0,&\text{иначе}}\]
        Таким образом, $g f_{x_i, u}$ отправляет не в ноль элементы $x$, такие, что $x \in H x_i^{-1} g^{-1}$.
        Мы уже ранее сказали, что $x_i^{-1}g^{-1} = h^{-1}x_j^{-1}$. Видим, что
        \[gf_{x_i,u}(x) = \all{\pi_{b}(u) = \pi_{bh^{-1}}(\pi_{h}(u)), &x = bh^{-1}x_j^{-1} \in H x_j^{-1}\\0,&\text{иначе}}\]
        Действительно, это совпадает с определением $f_{x_j, \pi_{h}(u)}$, действия на базисах сошлись.
    }
}
Матрица индуцированного представления выглядит так
\[\pi_g = \vect{\pi_{x_1^{-1}gx_1} & \cdots & \pi_{x_1^{-1}gx_t} \\ \vdots & \ddots & \vdots \\ \pi_{x_t^{-1}g x_1} & \cdots & \pi_{x_t^{-1}gx_t}}\]
где запись означает ненулевое значение $\pi_{x_i^{-1} g x_j}$ только если $x_i^{-1}g x_j \in H$.

Таким образом, матрица индуцированного представления --- блочно-мономиальная, и $\induced^G_H(U) = x_1 U \oplus \dots \oplus x_t U$.
Операторы $\pi_g$ сначала переставляют эти слагаемые, а потом на каждом действуют оператором $\pi_h$ для некоторого $h$ (для каждого $U$ --- $h$ --- своё).


\section{Индуцированные характеры}
Пусть $\pi: H \map GL(U)$ --- представление, $\chi = \chi_\pi$.
Мы построили $\induced_H^G(\pi) \eqqcolon \pi^G: G \map GL(V = U^G)$.

Посчитаем характер $\chi^G: G \map K$.
Для этого сначала продолжим характер $\chi$ до функции на $G$:
\begin{align*}
    \chi^0(G): G &\map K\\g &\mapsto \all{\chi(g),&g \in H \\ 0,& g \notin H}
\end{align*}
$\chi^0$ --- совсем необязательно центральная функция на $G$.
\theorem{
    $\chi^G(g) = \sum\limits_{\substack{x_i \in T \\ x_i^{-1}g x_i \in H}}\chi(x_i^{-1}gx_i)$.
    \provehere{
        Характер --- сумма диагональных элементов, и так как матрица $\pi^G$ блочно-мономиальная, то суммировать надо характеры ровно тех $\pi_{x_i^{-1}g x_i}$, где $x_i^{-1}g x_i \in H$.
    }
}
\corollary{
    $\chi^G(g) = \sum\limits_{x_i \in T}\chi^0(x_i^{-1}gx_i)$.
}
\corollary{
    Чтобы суммировать не по $x_i \in T$, а по $y \in G$, надо просто заменить $y = x_i h$. $\chi$ --- центральная на $H$ функция, поэтому всё сойдётся:
    \[\chi^G(g) = \frac{1}{|H|}\sum\limits_{\substack{y \in G\\y^{-1}gy \in H}}\chi(y^{-1}gy)\]
}
\corollary{
    Объединяя предыдущие два, получаем
    \[\chi^G(g) = \frac{1}{|H|}\sum\limits_{y \in G}\chi^0(y^{-1}gy)\]
}


\section{Формула слияния (fusion formula)}
Пусть $H \le G, g \in G$.
Посмотрим на $g^G \cap H$, это объединение некоторых классов сопряжённости в $H$:
\[g^G \cap H = h_1^H \sqcup \dots \sqcup h_m^H\]
Эти элементы $h_1, \dots, h_m$ называются \emph{представителями классов сопряжённых с $g$ в $H$}.

\theorem{
    Если $\chi$ --- характер $H \le G$, $g \in G, h_1, \dots, h_m$ --- представители классов сопряжённых с $g$ в $H$, то
    \[\chi^G(g) = |C_G(g)| \sum\limits_{i = 1}^{m}\frac{\chi(h_i)}{|C_H(h_i)|}\]
    Поскольку $g \sim_G h_i$, то $C_G(g) \sim C_G(h_i)$ и, значит, формулу можно переписать в виде
    \[\chi^G(g) = \sum\limits_{i = 1}^{m}|C_G(h_i) : C_H(h_i)|\cdot \chi(h_i)\]
    \provehere{
        Введём $Y_i \coloneqq \defset{y \in G}{y^{-1}g y \sim_H h_i}$, причём $Y = Y_1 \sqcup \dots \sqcup Y_m$.
        Запишем найденную ранее формулу и будем её преобразовывать:
        \[\chi^G(g) = \frac{1}{|H|}\sum\limits_{\substack{y \in G\\y^{-1}gy \in H}}\chi(y^{-1}gy) = \frac{1}{|H|}\sum\limits_{y \in Y}\chi(y^{-1}gy) = \frac{1}{|H|}\sum\limits_{i = 1}^{m}\sum\limits_{y \in Y_i}\chi(y^{-1}g y) = \frac{1}{|H|}\sum\limits_{i = 1}^{m}|Y_i|\chi(h_i)\]
        Осталось доказать, что $\frac{|Y_i|}{|H|}= \frac{|C_G(g)|}{|C_H(h_i)|}$.

        Для этого заметим, что $Y_i = C_G(g) y_i H$: $\subset$ очевидно, $\supset$ показывается так: \multline{y^{-1} g y \sim_H h_i \then \exists h \in H: hy^{-1}gyh^{-1} = h_i,\quad\text{ и далее }\\ \all{y_i^{-1}g y_i = h_i \\ h y^{-1}g y h^{-1} = h_i} \then y_i h y^{-1} \in C_G(g) \then y \in C_G(g)y_i h}
        Далее для подсчёта количества элементов в двойном смежном классе можно воспользоваться формулой Фробениуса, учитывая, что $C_G(g) \cap H = C_H(h_i)$.
    }
}
В нормальной подгруппе $H \normeq G$ видно, что сумма пустая, то есть характеры, индуцированные с нормальной подгруппы, сконцентрированы на $H$.

\newlection{26 октября 2023 г.}
\intfact{
    Индуцирование транзитивно:
    пусть $F \le H \le G$. Тогда $\induced^G_F(\chi) = \induced_H^G(\induced_F^H(\chi))$.
}


\section{Закон взаимности Фробениуса}
Пусть $H \le G$, $\pi: H \map GL(U). \rho: G \map GL(V)$ --- неприводимые представления.

Рассмотрим $\pi^G \coloneqq \induced^G_H(\pi)$ и $\rho_H - \restricted_H^G(\rho)$.
Закон взаимности Фробениуса говорит, что $\pi^G$ содержит $\rho$ с той же кратностью, что $\rho_H$ содержит $\pi$.
\theorem{
    Пусть $\chi \in cf_K(H), \rho \in cf_K(G)$.
    Тогда $B_G(\chi^G, \rho) = B_H(\chi, \rho_H)$. Здесь $\chi^G$ --- индуцирование центральной функции по формулам, полученным ранее (они все дадут одинаковый результат, так как выведены одна из другой), $\rho_H$ --- ограничение $\rho\Big|_H$.
    \provehere{
        \[B_G(\chi^G, \rho) = \frac{1}{|G|}\sum\limits_{g \in G}\chi^G(g)\rho(g^{-1}) = \frac{1}{|G|}\frac{1}{|H|}\sum\limits_{g \in G}\sum\limits_{y \in G}\chi(y^{-1}gy)\rho(g^{-1})\]
        Заменим порядок суммирования: пусть $h = y^{-1}gy$.
        \[\frac{1}{|G|}\frac{1}{|H|}\sum\limits_{g \in G}\sum\limits_{y \in G}\chi(y^{-1}gy)\rho(g^{-1}) = \frac{1}{|G|}\frac{1}{|H|}\sum\limits_{h \in H}\sum\limits_{y \in G}\chi(h)\rho(y h^{-1}y^{-1}) = \frac{1}{|G|}\sum\limits_{y \in G}\underbrace{\sum\limits_{h \in H}\chi(h)\rho(h^{-1})}_{B_H(\chi, \rho_H)}\]
    }
}
\corollary{
    Если $A \le G$ --- абелева подгруппа, $\rho: G \map GL(U)$ неприводимо, то $\deg(\rho) \le |G : A|$.
    \provehere{
        Все неприводимые представления $A$ одномерны.
        $\rho\Big|_A$ --- прямая сумма неприводимых представлений $A$.

        Пусть $\pi$ --- какое-то из них. Тогда $\rho$ входит в $\induced^G_A(\pi)$, но $\deg(\induced^G_A(\pi)) = |G : A|$.
    }
}


\chapter{Необработанные куски из теории представлений}
\newlection{19 октября 2023 г.}
\lemma[Лемма Шура над замкнутым полем]{
    Пусть $R$ --- $F$-алгебра, $F$ --- замкнутое поле. Рассмотрим простой $R$-модуль $M$, пусть $\dim_F{M} < \infty$.

    Рассмотрим $\phi \in \End_R(M)$. Тогда $\phi$ --- умножение на скаляр $\lambda \in F$.
    \provehere{
        Нам известно, что $\End_R(M)$ --- тело, причём $\End_R(M) \subset \End_F(M)$.
        $\End_R(M)$ --- $F$-подпространство;\ так как $\End_R(M)$ конечномерная алгебра с делением над замкнутым полем $F$, то $\End_R(M) \cong F$.
    }
}
Пусть $\pi: G \map GL(V)$.
Тогда \emph{степень представления} $\deg(\pi) = \dim V$.
\corollary{
    Всякое неприводимое представление абелевой группы над замкнутым полем одномерно (имеет степень 1).
    \provehere{
        $F[G]$ --- коммутативная алгебра. Тогда $\forall g \in G, h \in F[G], v \in V: g \cdot hv = h \cdot gv$, то есть умножение на $g$ --- автоморфизм $F[G]$-модуля $V$.

        Используя лемму Шура, получаем, что $\exists \alpha_g \in F: \forall v \in V: g \cdot v = \alpha_g v$.

        Любое одномерное $F$-подпространство $F[G]$-инвариантно. Так как представление неприводимо, то представление одномерно.
    }
}


\section{Операции над представлениями}
\definition[Характер представления]{
    След матрицы представления.
}
\bullets{
    \item % Можно рассмотреть $U \otimes_{F[G]} V$.
    Можно рассмотреть действие $G \curvearrowright U \otimes_F V$.
    $g(u \otimes v) = (gu) \otimes (gv)$.

    Иными словами (так удобно писать, когда $U = V$, чтобы различать представления) \[(\rho \otimes \pi)_g(u \otimes v) = \rho_g(u) \otimes \pi_g(v)\]
    Пусть $u_1, \dots, u_n$ --- базис $U$, $v_1,\dots,v_m$ --- базис $V$.
    Выберем в $U \otimes_F V$ базис $u_i \otimes v_j$.
    Тогда \[(\rho \otimes \pi)_g(u_i \otimes v_j) = \rho_g(u_i) \otimes \pi_g(v_j) = \sum\limits_{k = 1}^{n}u_k([\rho_g]_u)_{k,i} \otimes \sum\limits_{l = 1}^{m}v_l ([\pi_g]_v)_{l,j} = \sum\limits_{m,l}u_m \otimes v_l \cdot (\rho_{m,i}(g)\cdot\pi_{l,j}(g))\]
    След представления $\rho \otimes \pi$ равен произведению характеров: $\sum\limits_{i,i}\rho_{i,i}(g)\pi_{j,j}(g) = \tr(\rho_g)\cdot \tr(\pi_g)$.

    Определитель матрицы $(\pi \otimes \rho)_g = \prod\limits_{i,j}\lambda_i \mu_j = (\det \rho_g)^k \cdot (\det \pi_g)^n$.
}
\lemma{
    Предположим, что $M, N$ --- $R$-модули, и $M \overset{\psi}\map N \overset{\phi}\map M$ --- тождественная композиция.
    Тогда $N \cong M \oplus \Ker \phi$ (в случае групп полупрямое произведение).
    \provehere{
        Докажем, что $N = \Image \psi \oplus \Ker \phi$. Так как $\psi$ инъективно, то $\Image \psi \cong M$.
        Рассмотрим $x \in \Image \psi \cap \Ker \phi$. $x = \psi(y)$, значит, $\phi(\psi(y)) = 0$, откуда $x = 0$.

        Если $z \in N$, то $\psi(\phi(z)) \eqqcolon t$ лежит в $\Image(\psi)$, причём $z - t \in \Ker(\phi)$: $\phi(z - t) = \phi(z) - \phi(\psi(\phi(z))) = 0$.
    }
}
Данная лемма позволяет выделять подмодуль прямым слагаемым, если найдена ретракция $\phi$.
\theorem[Машке]{
    Пусть $F$ --- необязательно замкнутое поле, $G$ --- конечная группа, причём $\chr F \notdivs |G|$.
    Тогда любое конечномерное представление $G$ над $F$ вполне приводимо.
    \provehere{
        Пусть $\{0\} \ne N \lneq M$, где $M$ --- конечномерный. $M = N \oplus N'$, где $N'$ --- $F$-подпространство (не $F[G]$ подмодуль).
        Заведём $p: M \map N, p(n + n') = n$, где $n \in N, n' \in N'$.

        $p$ --- ретракция, подправим его так, чтобы оно стало $G$-эквивариантным отображением.
        \[\tilde{p}(v) = \frac{1}{|G|}\sum\limits_{g \in G}gp(g^{-1}v)\]
        $\tilde{p}$ по-прежнему $F$-линейна, рассмотрим для $h \in G$: \[\tilde{p}(hv) = \frac{1}{|G|}\sum\limits_{g \in G}gp(g^{-1}hv) = \frac{1}{|G|}\sum\limits_{h^{-1}g \in G}h\cdot(h^{-1}g)\cdot p(g^{-1}hv) = h\tilde{p}(v)\]
        Таким образом, $\tilde{p}$ $G$-эквивариантно (и является ретракцией: $\forall n \in N: \tilde{p}(n) = g p(g^{-1}n) = gg^{-1}n= n$), и $N$ выделяется прямым слагаемым.}
}
\corollary{
    По-прежнему $\chr F \notdivs |G|$.
    Если все неприводимые представления одномерны, то группа абелева.
    \provehere{
        Рассмотрим регулярное представление $F[G]$ как $F[G]$-модуля. $F[G] \cong \bigoplus\limits_{k = 1}^{|G|}V_k$, где $V_k$ неприводимы: $\dim_F(V_k) = 1$.
        В матричном виде $\Image \text{reg} \subset D(|G|, F)$.
    }
}


\newlection{24 октября 2023 г.}
Далее мы везде работаем над алгебраически замкнутым полем характеристики нуль.
В частности, всякое такое поле содержит алгебраическое замыкание $\Q$ --- алгебраические числа $\A$.

\intfact{Любое представление над полем характеристики нуль эквивалентно представлению над $\C$.}

Будем считать, что базовое поле --- $\C$.

Вспомним про эрмитовскую полуторалинейную форму
\[B(x, y\lambda) = B(x, y)\lambda \qquad B(x, y) = \overline{B(y, x)} \qquad \then \qquad B(x\lambda, y) = \overline{\lambda}B(x, y)\]
Будем считать, что форма положительно определена: $\forall x \ne 0: B(x, x) > 0$.
\definition[Унитарная группа]{
    $U(B) \bydef \defset{a \in GL(n, \C)}{\forall x, y: B(ax, ay) = B(x, y)}$
}
\theorem[Унитаризуемость]{
    Для любого представления $\pi: G \map GL(n, \C): \exists$ положительно определённая эрмитова форма $B$: $\Image(\pi) \subset U(B)$.
    \provehere{~(\ref{uni})}
}
Дальше здесь идёт почти буквально лемма Шура в матричной форме.

Устроим билинейную форму на множестве функций $K^G$:
\[B(\eta, \theta) = \frac{1}{|G|}\sum\limits_{g \in G}\eta(g)\theta(g^{-1})\]
Чаще всего мы будем вычислять скалярное произведение от характеров.

Пусть $\pi: G \map GL(n, K)$ --- матричное представление, $\chr(K) \notdivs |G|$ и $K$ алгебраически замкнуто.

\theorem[Соотношения ортогональности Шура]{
    Пусть $\pi: G \map GL(n, K), \rho: G \map GL(m, K)$ --- два матричных представления.
    \numbers{
        \item Если $\pi \nsim \rho$, то $\forall i,j,k,l: B(\pi_{i,j}, \rho_{k,l}) = 0$.
        \item Если $\pi = \rho$, то $B(\pi_{i,j}, \pi_{k,l}) = \frac{1}{\deg(\pi)}\delta_{i,l}\delta_{j,k}$.
    }
    \provehere{
        Подставим в качестве $\phi$ матричную единицу $e_{i,j}$. Её симметризация $\phi_0 = \frac{1}{|G|}\sum\limits_{g \in G}\rho_g e_{i,j} \pi_g^{-1}$.
        Посчитаем элемент $\phi_0$ в позиции $(k,l)$. Он равен $\frac{1}{|G|}\sum\limits_{g \in G}\rho_{k,i}(g)\cdot\pi_{j,l}(g^{-1})$.

        В случае $\pi \nsim \rho$ $\phi_0$ равно нулю, откуда $B(\pi, \rho) = 0$. Иначе, если $\pi = \rho$, то $\exists \lambda \in K: (\phi_0)_{k,l} = \lambda \delta_{k,l}$.
        Посчитаем $\lambda$: $\tr(\phi_0) = \frac{1}{|G|}\sum\limits_{g \in G}\tr(\pi_g e_{i,j} \pi_g^{-1}) = \frac{1}{|G|}\sum\limits_{g \in G}\tr(e_{i,j}) = \delta_{i,j}$.

        Отсюда $\deg(\pi)\cdot \lambda = 1$.
    }
}
\corollary{
    $B$ невырождена на подпространстве функций $K^G$.
    При данной форме пространство --- ортогональная прямая сумма $n$ одномерных подпространств и $\frac{n^2-n}2$ гиперболических плоскостей.
}
\corollary{
    Пусть $\pi^{(1)}, \dots, \pi^{(m)}$ --- все неэквивалентные неприводимые представления $G$.
    Тогда набор функций $\defset{\pi^{(i)}_{k,l}}{1 \le i \le m, 1 \le k, l \le \deg(\pi^{(i)})}$ линейно независим.
    \provehere{
        Рассмотрим матрицу $B(\pi^{(i)}_{k,l}, \pi^{(j)}_{k',l'})$, она невырождена.
    }
}
Заметим, что $\dim F^G = |G|$.
В этом пространстве нашлись $n_1^2 + \cdots + n_s^2$ линейно независимых функций (здесь $n_i = \deg(\pi^{(i)})$.

Из теоремы Веддербарна --- Артина сразу следует, что они ещё и являются системой образующих (для замкнутого поля хорошей характеристики).

Мы же это докажем используя технику характеров для замкнутого поля характеристики нуль.
\newlection{26 октября 2023 г.}
Пусть $\phi \in F^G$.
Устроим отображение \begin{align*}
                        F^G &\map F[G] \\ \phi &\mapsto \sum\limits_{g \in G}\phi(g) \cdot g
\end{align*}
Заметим, что $h^{-1}(\sum\limits_{g \in G}\phi(g)\cdot g)h = \sum\limits_{g \in G}\phi(g) \cdot \underbrace{h^{-1}gh}_{f} = \sum\limits_{f \in G}\phi\left(hfh^{-1}\right)f \overset{?}= \sum\limits_{f \in G}\phi(f)$.

Если это верно для любого $h$, то $\phi$ --- центральная функция, функция класса.

Обозначим $C_1, \dots, C_m$ --- классы сопряжённости $G = C_1 \sqcup \dots \sqcup C_m$.

Пусть $c \in \Cent(K[G])$. $c = \sum\limits_{i = 1}\phi(C_i)\underbrace{\sum\limits_{g \in C_i}g}_{c_i} = \sum\limits_{i = 1}^{m}\phi(C_i)c_i$.

Пусть $\pi$ неприводимо.
Продлим по линейности $\pi: F[G] \map \End(V)$.
Если $c \in \Cent(F[G])$, то $\pi_c \cdot \pi_g = \pi_{c \cdot g} = \pi_{g \cdot c} = \pi_g \cdot \pi_c$.

Таким образом, $\pi_c \in \End_{F[G]}(V)$.
Согласно лемме Шура, $\pi_c$ --- гомотетия, $\lambda \id$.

Чтобы посчитать $\lambda$, предположим, что $c = \sum\limits_{i = 1}^{m}\phi(C_i)c$. $\tr(\pi_c) = \sum\limits_{i = 1}^{m}\phi(C_i)\tr(c_i) = \sum\limits_{i = 1}^{m}\phi(C_i)\chi_{\pi}(C_i) \cdot |C_i|$.

\newlection{31 октября 2023 г.}
Если модифицировать $c = \frac{1}{|G|}\sum\limits_{g \in G}\phi(g)g^{-1}$, то получится $\pi_c = \lambda \id$, где $\lambda$ находится при подсчёте следа:
\[\tr(\pi(c)) = \frac{1}{|G|}\sum\limits_{g \in G}\phi(g)\tr(\pi(g^{-1})) = \frac{1}{|G|}\sum\limits_{g \in G}\phi(g)\chi_\pi(g^{-1}) = B(\phi,\chi_{\pi})\]

\theorem{
    Характеры неприводимых представлений образуют базис пространства функций классов
    \provehere{
        Пусть $X$ --- подпространство, порождённое неприводимыми характерами, тогда $cf_K(G) = X \oplus X^{\perp_B}$, так как $B$ невырождена на $X$.
        Дальше аналогично проверяем, что если функция класса ортогональна всем характерам, то она равна нулю.
    }
}


\section{Таблицы характеров}
Пусть $\chi_1, \dots, \chi_t$ --- хаарктеры неприводимых представлений, $C_1, \dots, C_s$ --- классы сопряжённых.
Таблица характеров --- таблица следующего вида, где в ячейке $\chi_i, C_j$ стоит значение $\chi_i(C_j)$.
\[\begin{array}{c|ccc}
      & C_1 = \{1\} & \cdots & C_s         \\\hline
      \chi_1 = 1 & 1           & \cdots & 1           \\
      \chi_2     & \deg \pi_2  & \cdots & \chi_2(C_s) \\
      \vdots     & \vdots      & \ddots & \vdots      \\
      \chi_s     & \deg \pi_s  & \cdots & \chi_s(C_s) \\
\end{array}\]
Пусть $\Z \subset F$.
\definition[Целые алгебраические числа над $F$]{
    $\A_F \bydef \Int_F(\Z) \bydef \defset{a \in F}{\exists p \in \Z[t]: \lc(p) = 1, \ev_a(p) = 0}$.
}
\ok
Пусть $R \subset A$ --- коммутативные кольца с единицей.
\definition[$A$ целое над $R$]{
    $\forall a \in A: a$ --- целый над $R$.
}
\definition[$A$ --- конечное над $R$]{
    $A$ --- конечно порождено, как модуль над $R$.
}
\lemma{
    Пусть $A \subset B \subset R$ --- цепочка конечных расширений. Тогда $A$ конечно над $R$.
}
\lemma{
    Пусть $A = R[a_1, \dots, a_n]$, где $a_i$ --- целые над $R$.
    Тогда $A$ конечно над $R$.
    \provehere{
        $a_i^k \in \angles{a_i^0, \dots, a_i^{k-1}}$, если $k$ --- степень $a_i$. Индукция по $n$ с применением предыдущей леммы.
    }
}
\theorem{
    Если $R \subset A$ --- конечное расширение, и $a \in A$, то $a$ --- целый над $R$.
    \provehere{
        Пусть $m_a: A \map A$ --- гомоморфизм ($R$-модулей) умножения на $a$. $m_a \in \End_R(A)$.

        Пусть $x_1, \dots, x_n$ порождают $A$ над $R$. $x_i \cdot a\sum\limits_{j = 1}^{n}c_{i,j}x_j$
        Обозначим $c = (c_{i,j})$, тогда $xa = Cx$.

        $\chi_C(C) = 0$ по теореме Гамильтона --- Кэли (это, кстати, полиномиальное равенство, можно использовать, что оно было доказано над алгебраически замкнутым полем).

        $0 = \chi_c(C)x = \chi_c(a\id)x$. Так как $1 \in \Lin(x_1, \dots, x_n)$, то $\chi_c(a) \cdot 1 = 0$, откуда $\chi_c(a) = 0, a \in \A_R$
    }
}
\corollary{
    $R \subset A$ --- конечно $\iff A$ порождено как $R$-алгебра конечным числом целых элементов.
}
Возвращаясь к таблице характеров, получаем, что $\chi(g) \in \A$.
\newlection{7 ноября 2023 г.}
\bullets{
    \item Наружное произведение.
}
Пусть $K/F$ --- расширение полей ($F \le K$).
Пусть $V$ --- векторное пространство над $K$, его можно рассматривать, как векторное пространство над $F$ --- надо забыть про умножение на элементы $K$, не лежащие в $F$.

Можно сделать наоборот.
Если $U$ --- векторное пространство над $F$, то можно сделать векторное пространство над $K$:
о $K \otimes_F U$ можно мыслить, как о векторном пространстве над $K$: действие устроено так:
\[\alpha(\beta \otimes x) = (\alpha \beta)\otimes x \text{, при }\alpha,\beta \in K, x \in U\]
$\dim_K(K \otimes_F U) = \dim_F(U)$, конструкция называется \emph{расширением скаляров}.
\ok
Пусть $M$ --- правый модуль над $R$, $N$ --- левый модуль над $R$, $R$ --- необязательно коммутативное кольцо.
Тогда $M \otimes_R N$ --- абелева группа, которую можно представить в виде
\[M \otimes_R N = M \otimes_\Z N/ \angles{m \otimes \alpha n - m \alpha \otimes n \middle|m \in M, n \in N, \alpha \in R}\]
Если $M$ является $A-R$-бимодулем, то $M \otimes_R N$ является левым $A$-модулем (структура вводится естественным образом).
Заметим, что надо всё-таки проверить корректность: например, модуль, по которому происходит факторизация, должен быть $A$-инвариантным.

Пусть $\phi: R \map A$ --- гомоморфизм колец.
Ему сопоставляется функтор $\phi^{\#}: A-\text{mod} \map R-\text{mod}$.
$A$-модуль $M$ превращается в $R$-модуль так: $r \cdot m = \phi(r) \cdot m$.
Если $\phi$ --- вложение колец, то это соответствует расширению полей выше.

Также можно сопоставить $\phi_{\#}: \Lmod{R} \map \Lmod{A}$ --- расширение скаляров.
Получаем левый $A$-модуль $\phi_{\#}(M) = A \otimes_R M$, где $A$ рассматривается, как $A-R$-бимодуль.

Теперь пусть $\phi: H \map G$ --- гомоморфизм групп.
Тогда его можно продолжить до гомоморфизма групповых алгебр $\phi: F[H] \map F[G]$.

Если $U$ --- $F[H]$-модуль, соответствующий представлению $\pi$, то $\phi_{\#}(U)$ --- $F[G]$-модуль.
Если $H \le G$ и $\phi$ --- вложение, то соответствующее представление --- \emph{индуцированное с представления} $\pi$, обозначается $\induced_H^G(\pi)$.

Наоборот, если $\rho$ --- представление группы $G$, $V$ --- соответствующий $F[G]$-модуль, то $\phi^{\#}(V)$ --- $F[H]$-модуль.
Если $H \le G$ и $\phi$ --- вложение, то соответствующее представление --- \emph{сужение представления} $\pi$, обозначается $\restricted_H^G(\pi)$.

В частности, если $G$ --- абелианизация $H$, то $\phi: H \twoheadrightarrow G$ сюръективно.
Его можно продолжить до $\phi: F[G] \twoheadrightarrow F[G^{\ab}]$.
Тогда ограничение представления $\phi^{\#}$ --- инфляция, индуцированное представление $\phi_{\#}$ --- дефляция.
\newlection{9 ноября 2023 г.}
В абстрактной ситуации пусть $\phi: R \map A$ -- гомоморфизм необязательно коммутативных колец.
\begin{align*}
    \phi_{\#}: \Lmod{R} &\map \Lmod{A}\\M &\mapsto A \otimes_R M\text{ --- задали на $M$ структуру $A$-модуля}
\end{align*}
\begin{align*}
    \phi^{\#}: \Lmod{A} &\map \Lmod{R}\\M &\mapsto M\text{ --- забывающий функтор}
\end{align*}

Пусть $M$ --- $R$-модуль, $N$ --- $A$-модуль.
\proposal{Имеет место следующее универсальное свойство:

Для любого гомоморфизма $R$-модулей $\phi$: $\exists!$ гомоморфизм $A$-модулей $\psi: A \otimes_R M \map N$.
% https://q.uiver.app/#q=WzAsMyxbMCwwLCJNIl0sWzIsMCwiQSBcXG90aW1lc19SIE0iXSxbMSwxLCJOIl0sWzAsMiwiXFxwaGkiXSxbMCwxLCJpIl0sWzEsMiwiXFxwc2kiLDJdXQ==
    \[\begin{tikzcd}[ampersand replacement=\&]
          M \&\& {A \otimes_R M} \\
          \& N
          \arrow["\phi", from=1-1, to=2-2]
          \arrow["i", from=1-1, to=1-3]
          \arrow["\psi"', from=1-3, to=2-2]
    \end{tikzcd}\]
    \provehere{
        Элементы $\defset{1 \otimes m}{m \in M}$ порождают $A \otimes M$. Чтобы диаграмма была коммутативной, необходимо равенство $\psi(i(m)) = \psi(1 \otimes m) = \phi(m)$.
        Значит, $\psi$ единственно, если уж существует.

        Зададим на разложимых тензорах $\psi(a \otimes m) = a \cdot \phi(m)$.
        Оно билинейно и распространяется по линейности на $A \otimes_R M$.
    }
}
\corollary{
    Имеет место естественный изоморфизм $\Hom_R(M, \phi^{\#}(N)) \cong \Hom_A(\phi_{\#}(M), N)$.

    В теории категорий такие два функтора $\phi^{\#}$ и $\phi_{\#}$ называются сопряжёнными.
}

\theorem[Закон взаимности Фробениуса]{
    Предположим, что $H \le G$, $\chr(F) = 0$ (на самом деле достаточно $\chr(F)\notdivs|G|$), $F$ алгебраически замкнуто.
    Пусть даны два представления $\pi: H \map GL(V)$, $\rho: G \map GL(U)$.

    Тогда $\text{Mor}(\induced^G_H(\pi), \rho) \cong \text{Mor}(\pi, \restricted^G_H\rho)$.

    Отсюда следует $B_H(\chi_\pi, \chi_{\restricted_H^G\rho}) = B_G(\chi_{\induced_H^G \pi}, \chi_\rho)$. Обычно пишут $B_H(\chi_\pi, {\restricted_H^G\chi_\rho}) = B_G(\induced_H^G \pi_\chi, \chi_\rho)$
    \provehere[Доказательство закона взаимности Фробениуса]{
        Рассмотрим следующую ситуацию.

        Пусть $\phi: H \hookrightarrow G$ --- вложение.
        Пусть $V$ --- $F[H]$-модуль, $U$ --- $F[G]$-модуль.

        Тогда $\Hom_{F[G]}(\phi_{\#}(V), U) \cong \Hom_{F[H]}(V, \phi^{\#}(U))$.

        Это $F$-линейный изоморфизм (проверить), то есть изоморфизм векторных пространств над $F$.

        Пусть $U, V$ --- простые модули, поле алгебраически замкнуто и характеристики, не делящей порядок группы.
        Пусть $\phi^{\#}(U) = \bigoplus\limits_{i = 1}^{k}V_i$ --- разложение в сумму простых $F[H]$-модулей.
        \[\dim_F\Hom\left(V, \bigoplus\limits_{i = 1}^{k}V_i\right) = \text{количество $V_i \cong V$ (лемма Шура)}\]
        Аналогично пусть $\phi_{\#}(V) = \bigoplus\limits_{i = 1}^{k}U_i$ --- разложение в сумму простых $F[G]$-модулей.
        \[\dim_F\Hom\left(\phi_{\#}(V), U\right) = \text{количество $U_i \cong U$}\]
        Но тогда получается, что количество вхождений $V$ в $\phi^{\#}(U)$ равно количеству вхождений $U$ в $\phi_{\#}(V)$

        Теперь пусть $\pi = \bigoplus \pi_i$, $\rho = \bigoplus \rho_j$ --- разложение в прямую сумму неприводимых.
        $\induced^G_H(\pi) = \bigoplus\induced^G_H(\pi_i)$ так как тензорное произведение дистрибутивно относительно прямого произведения:
        $F[G] \otimes (\bigoplus V_i) = \bigoplus (F[G] \otimes V_i)$.

        Также $\restricted^G_H(\rho) = \bigoplus \restricted\rho_i$, так как $\restricted$ --- просто сужение.

        Из билинейности $B$ и того, что $\chi_\pi = \sum\induced^G_H\chi_{\pi_i}, \chi_\rho = \sum\restricted^G_H\chi_{\rho_j}$ следует равенство скалярных произведений в общем случае.
    }
}

\subsection{Закон Фробениуса без теории категорий}
Пусть $H \le G$, $\pi: H \map GL(V)$ --- представление, найдём матрицу $\induced^G_H(\pi)$.

Рассмотрим $F[G] \otimes_{F[H]} V$.
Пусть $(v_1, \dots, v_n)$ --- базис $V$.
Зафиксируем трансверсаль $T = \{1, g_2, \dots, g_m\}$ --- представители левых смежных классов $G/H$:
\[G = \bigsqcup\limits_{i = 1}^{m}g_i H\]
Всякий разложимый тензор является линейной комбинацией векторов $g_i \otimes v_j$:
\[F[G] \otimes_{F[H]} V \cong (F[H])^{\oplus m} \otimes_{F[H]} V \cong V^{\oplus m}\text{ --- изоморфизм $F[H]$-модулей}\]
Тогда как $F$-модули они изоморфны и подавно, $\dim_F(FG\otimes_{F[H]}V) = m \cdot \dim V$, откуда $g_i \otimes v_j$ --- базис.

Пусть $gg_i = g_k h$ для некоторого $g_k \in T, h \in H$.
\[g \cdot (g_i \otimes v_j) = g_k h \otimes v_j = g_k \otimes hv_j = g_k \otimes \pi_h(v_j) = \sum\limits_{l = 1}^{n}(g_k \otimes v_l)\pi_{l,j}(h)\]
Отсюда можно выцепить матрицу $\induced^G_H\pi_h$.
Матрица блочная, блоки соответствуют $g_i \otimes v_j$ для фиксированного $j$.

\[\induced^G_H\chi_\pi(g) = \sum\limits_{i: g_i^{-1}g g_i \in H}\chi_\pi(g_i^{-1}g g_{i})\]
Характер --- центральная функция на $H$, а не на $G$, поэтому избавиться от сопряжения легко не получится.
Пусть $f = g_i h \in g_i H$.
Тогда $\chi_\pi(f^{-1}g f) = \chi_\pi(h^{-1}g_i^{-1}g g_{i}h) = \chi_\pi(g_i^{-1}gg_i)$.
\[\induced^G_H\chi_\pi(g) = \sum\limits_{i: g_i^{-1}g g_i \in H}\chi_\pi(g_i^{-1}g g_{i}) = \frac{1}{|H|}\sum\limits_{f: f^{-1}gf \in H}\chi_{\pi}(f^{-1}gf)\]

\multline{B(\induced\chi_\pi, \rho) = \frac{1}{|G|}\sum\limits_{g \in G}\induced\chi_\pi(g) \cdot \chi_\rho(g^{-1}) = \frac{1}{|G| \cdot |H|}\sum\limits_{g, f \in G: f^{-1}gf \in H}\chi_\pi(f^{-1}gf)\chi_\rho(g^{-1}) = \\ \left\|f^{-1}gf = h, g^{-1} = fhf^{-1}\right\| \\ = \frac{1}{|G|\cdot|H|}\sum\limits_{f \in G, h \in H}\chi_\pi(h)\cdot \chi_\rho(fh^{-1}f^{-1}) = \frac{1}{|H|}\sum\limits_{h \in H}\chi_\pi(h)\chi_\rho(h^{-1})}
В данном доказательстве мы не пользуемся тем, что характеристика нуль.

