\documentclass[a4paper]{report}

\usepackage{../mathstemplate}
\usepackage{wasysym}

\date{I семестр, осень 2022 г.}
\title{Геометрия и топология. Неофициальный конспект}
\author{Лектор: Лебедева Нина Дмитриевна \\ Конспектировал Леонид Данилевич}


\begin{document}
    \maketitle
    \tableofcontents
    \newpage
    \setcounter{lection}{0}


    \chapter{База топологии}
    \newlection{8 ноября 2022 г.}


    \section{Метрические пространства}
    Рассмотрим произвольное множество $X$.
    Введём на нём метрику $d: X \times X \map \R_{\ge 0}$, удовлетворяющую некоторым тождествам:

    \begin{equation}
        \forall x, y \in X: ad(x, y) = 0 \iff x = y
    \end{equation}
    Симметричность:
    \begin{equation}
        \forall x, y \in X: d(x, y) = d(y, x)
    \end{equation}
    Неравенство треугольника:
    \begin{equation}
        \forall x, y, z \in X: d(x, y) + d(y, z) \ge d(x, z)
    \end{equation}
    \definition[Метрическое пространство]{Пара $(X, d)$, где $d$ удовлетворяет трём вышеперечисленным аксиомам.}

    При проверке, что некая функция действительно является метрикой, сложности чаще всего вызывает проверка третьей аксиомы, неравенства треугольника.
    Скорее всего, проверки остальных двух аксиом я буду опускать.

    \subsection{Примеры метрических пространств}
    \bullets{
        \item \emph{Дискретная метрика} может быть введена на любом множестве $X: d(x, y) = \all{0,&x = y\\1,&x\ne y}$.
        \item Для $X = \R^n$ \emph{манхеттенская метрика} $d(x, y) = \sum\limits_{i = 1}^n|x_i - y_i|$.
        \item Для $X = \R^n$ \emph{шахматная метрика (метрика Чебышёва)} $d(x, y) = \max\limits_{i = 1}^n|x_i - y_i|$.
        \item Для $X= C[0; 1]$ --- множества непрерывных функций $[0; 1] \map \R$ --- можно задать метрику $d(x, y) = \max\limits_{t \in [0; 1]}|x(t)-y(t)|$.
        Данная метрика вместе с $C[0; 1]$ образуют \emph{пространство непрерывных функций} $(X, d)$.
        \item Для $X = \R^n$ \emph{евклидова метрика} $d(x, y) = \sqrt{\sum\limits_{i = 1}^n (x_i - y_i)^2}$.
        Проверим, что для евклидовой метрики выполняется неравенство треугольника:
        \theorem[Прямое произведение метрических пространств]{
            Пусть $(X_1, d_1)$ и $(X_2, d_2)$ --- метрические пространства. Тогда функция $d: (X_1 \times X_2) \times (X_1 \times X_2) \map \R_{\ge 0}$, определённая $d((A_1, A_2), (B_1, B_2)) = \sqrt{d_1(A_1, B_1), d_2(A_2, B_2)}$, задаёт метрику на $X_1 \times X_2$.
            \prove{
                Проверим неравенство треугольника: рассмотрим $(A_1, A_2), (B_1, B_2), (C_1, C_2) \in X_1 \times X_2$.

                Обозначим $a_i = d_i(B_i, C_i), b_i = d_i(A_i, C_i), c_i = d_i(A_i, B_i)$.

                Используя свойства неравенств треугольника для $d_1$ и $d_2$, получаем \[\sqrt{c_1^2 + c_2^2} \le \sqrt{(a_1+b_1)^2 + (a_2 + b_2)^2}\]

                Рассмотрим на плоскости треугольник с координатами вершин ${(0, 0), (b_1, b_2), (a_1 + a_2, b_1 + b_2)}$.
                Неравенство треугольника для него выглядит \[\sqrt{(a_1+b_1)^2 + (a_2 + b_2)^2} \le \sqrt{a_1^2 + a_2^2} + \sqrt{b_1^2 + b_2^2}\]

                Дальше, по транзитивности, получаем $\sqrt{c_1^2 + c_2^2} \le \sqrt{a_1^2 + a_2^2} + \sqrt{b_1^2 + b_2^2}$, откуда в самом деле $d$ является метрикой.
            }

        }
        \corollary{На произведении $n$ пространств $X_1 \times \cdots \times X_n$ аналогичная функция \[d((a_1, \dots, a_n), (b_1, \dots, b_n)) = \sqrt{\sum\limits_{i = 1}^{n}d_i(a_i, b_i)}\] также является метрикой.}
        \note{
            Также иногда рассматривают метрику на прямом произведении пространств $d((a_1, a_2), (b_1, b_2)) = \max\{d_1(a_1, b_1), d_2(a_2, b_2)\}$.
            Проверить, что данная функция тоже является метрикой, довольно просто.
        }
        \item \up \definition[Сужение метрического пространства]{ Метрическое пространство $(X, d)$ можно сузить на $Y \subset X$, метрикой будет $d\big|_{Y \times Y}$.}
    }
    \definition[Открытый шар в метрическом пространстве $(X, d)$ с центром в $a \in X$ и радиусом $r > 0$]{
        $B_r(a) = \defset{x \in X}{d(a, x) < r}$.
    }
    \definition[Замкнутый шар в пространстве $(X, d)$ с центром в $a \in X$ и радиусом $r > 0$]{
        $\overline{B_r}(a) = \defset{x \in X}{d(a, x) \le r}$.
    }
    По умолчанию все шары открыты.
    \definition[Сфера в метрическом пространстве $(X, d)$ с центром в $a \in X$ и радиусом $r > 0$]{
        $S_r(a) = \defset{x \in X}{d(a, x) = r}$.
    }
    \definition[Расстояние от точки $a \in X$ до подмножества $A \subset X$]{
        $\inf\defset{d(x, a)}{x \in A}$
    }
    \definition[Окрестность множества $A \subset X$ с радиусом $r$]{
        $U_r(A) = \defset{x \in X}{d(x, A) < r}$.
    }
    \definition[Диаметр множества $A \subset X$]{
        $\diam(A) = \sup\defset{d(x, y)}{x, y \in X}$.

        Если $\diam(A) < \infty$, то множество называют \emph{ограниченным}.
    }
    Несложно проверить, что условие ограниченности эквивалентно тому, что множество лежит в некотором (открытом) шаре.

    \definition[Множество $A \subset X$ открыто]{
        Любая точка $a \in A$ содержится в $A$ вместе с некоторым своим шаром: \[\forall a \in A: \exists r > 0: B_r(a) \subset A\]
    }
    \fact{
        Множество $A$ открыто, если оно представимо, как объединение множества открытых шаров.
        $A = \bigcup\limits_{\alpha \in \varLambda}B_{r_\alpha}(x_\alpha)$.
        \provetwhen {
            Возьмём для каждой точки шар, с которым она содержится в множестве, и объединим их всех.
        }{
            Для каждой точки $x$ из шара $S$ подойдёт шар радиусом $r(S) - d(x, c(S))$, проверяется неравенством треугольника.
        }
    }
    \corollary{Открытый шар открыт.}
    \note{
        В метрике $(X, d)$ $X$ и $\o$ открыты.

        В дискретной метрике (все расстояния целые) $(X, d)$ любое одноэлементное множество открыто.
        Достаточно рассмотреть шар радиусом $\nicefrac{1}{2}$.
    }
    \theorem{
        Объединение открытых множеств открыто.
        Пересечение \textbf{конечного} числа открытых множеств открыто.
        \provebullets{
            \item Очевидно из определения через объединение шаров
            \item Всякая точка $a \in A$ лежит вместе с шаром радиуса $\min(r_1, \dots, r_n)$, где $r_i$ --- радиус открытого шара с центром в $a$, содержащегося в $A_i$.
        }
    }
    \note{
        $[0; 1] = \bigcap\left(-\frac{1}{n}; 1 + \frac{1}{n}\right)$ --- пересечение бесконечного числа открытых множеств может не быть открыто.
    }
    \proposal{
        Все открытые множества на прямой --- дизъюнктные объединения интервалов.
        \prove{
            Заметим, что $B_r(x) = (x - r; x + r)$.

            Для каждой точки можно найти максимальный по включению интервал, содержащийся в множестве, и содержащий данную точку.

            Любые два таких интервала либо уж не пересекаются, либо уж совпадают.
        }
    }


    \section{Топологические пространства}
    Пусть $X$ --- произвольное множество.
    Рассмотрим $\Omega \subset 2^X$, такое, что
    \begin{gather*}
        \o \in \Omega; \quad X \in \Omega\\
        \forall U \subset \Omega: \bigcup U \in \Omega\\
        \forall U \subset \Omega: (|U| < \infty \then \bigcap U \in \Omega)
    \end{gather*}
    Тогда будем говорить, что $\Omega$ --- \emph{топологическая структура (топология)} на множестве $X$.
    \definition[Топологическое пространство $(X, \Omega)$]{
        Множество $X$ с заданной на нём топологией $\Omega$.
    }
    В топологических пространствах элементы $\Omega$ называют открытыми множествами.

    \subsection{Примеры топологических пространств}
    \bullets{
        \item Для метрического пространства $(X, d)$ определяют \emph{индуцированное метрикой $d$ топологическое пространство} $(X, \Omega_d)$ где $\Omega_d$ --- множество подмножеств $X$, метрически открытых в $X$.

        Так, на прямой $\R$ при рассмотрении дискретной метрики $d(x, y) = \all{1,& x \ne y\\0,& x = y}$, или стандартной метрики $d(x, y) = |x - y|$, получаются различные топологические пространства.

        Дискретную метрику можно определить на любом множестве, породится дискретная топология $\Omega = 2^X$.

        \item Антидискретная топология $\Omega = \{\o, X\}$.

        \item \up \definition[Топология стрелки]{\label{subsec:arrow}: $(\R, \Omega)$, где $\Omega = \bigdefset{(a; +\infty)}{a \in \R} \cup \{\o, \R\}$.}

        \note{Пространство при $\Omega = \bigdefset{[a; +\infty)}{a \in \R} \cup \{\o, \R\}$ не удовлетворяет второй аксиоме.}
        \item Топология конечных дополнений: для произвольного $X: \Omega = \bigdefset{A \subset X}{|X \bs A| < \infty}$.
        \item На двухточечном множестве $\{0, 1\}$ есть 4 различные топологии, из них интересна (может быть?) топология $\Omega = \{\o, \{0\}, \{0, 1\}\}$ (или $\{\o, \{1\}, \{0, 1\}\}$, неважно).
    }

    \definition[Замнкутое множество]{Множество с открытым дополнением}

    \theorem{
        Для топологического пространства $(X, \Omega)$
        \begin{gather*}
            \o, X \text{ замкнуты}\\
            \text{пересечение замкнутых множеств замкнуто}\\
            \text{объединение конечного числа замкнутых множеств замкнуто}\\
        \end{gather*}
        \provehere{Следует из формулы двойственности де Моргана.}
    }

    \subsection{Примеры замкнутых множеств}
    \bullets{
        \item В дискретной метрике все множества замкнуты, так как все (их дополнения) открыты.
        \item На прямой $\R$ со стандартной метрикой отрезки $[a, b]$ и точки $\{a\}$ замкнуты.
        \item Хорошего вида у замкнутых множеств нет: так, Канторово множество замкнуто.
        \definition[Канторов множество]{
            Строится итеративно:
            \numbers{
                \item Берём отрезок $[0; 1]$.
                \item Вырезаем из него средний интервал, равный трети длины $(\nicefrac{1}{3}; \nicefrac{2}{3})$.
                \item Осталось два отрезка, $[0; \nicefrac{1}{3}]$ и $[\nicefrac{2}{3}; 1]$.
                Опять вырезаем из каждого средний интервал, равные трети длины.
                \item И так далее: можно доказать по индукции, что на очередном шагу будет некоторое конечное множество непересекающихся отрезков.
            }
        }
        Канторово множество замкнуто, так как можно взять последовательность надмножеств Канторова множества, появляющихся в определении (каждое замкнуто) и взять их пересечение.
        \item Точка и вообще замкнутый шар замкнуты в любом метрическом пространстве.
        \provehere{Несложно проверить.}
    }
    \fact{Для топологического пространства $(x, \Omega)$ $U \bs F = U \cap \overline{F}$ открыто, где $U$ --- открыто, $F$ --- замкнуто.}
    \fact{Для топологического пространства $(x, \Omega)$ $F \bs U = F \cap \overline{U}$ замкнуто, где $U$ --- открыто, $F$ --- замкнуто.}


    \section{Метрики и топологии}
    \definition[Метризуемое топологическое пространство $(X, \Omega$)]{
        Существует метрика ${d: X \times X \map \R}$, такая, что $\Omega = \Omega_d$.
    }
    Дискретная топология метризуема, порождается дискретной метрикой.

    \fact{Для $X: |X| > 1$ антидискретная топология не является метризуемой.
    \provehere{Так как $|X| > 1$, то $\exist x, y \in X: B_{\frac{d(x,y)}2}(x)$ содержит $x$, но не содержит $y$.}}

    \fact{Стрелка~(\cref{subsec:arrow}) тоже не метризуема.
    \provehere{Найдётся два непустых непересекающихся шара, но в стрелке любе два непустых открытых множества пересекаются.}
    }
    \fact{
        Топология конечных дополнений метризуема $\iff$ множество $X$ конечно.
        \provehere{
            $|X| < \infty$ --- топология дискретна.

            $|X| \ge \infty$ --- неметризуема по той же причине, что и стрелка, любые два открытых пересекаются
        }
    }


    \section{Сравнение метрик и топологий}
    Пусть на множестве $X$ заданы две различные топологии $\Omega_1$ и $\Omega_2$, причём $\Omega_1 \subseteq \Omega_2$.
    Говорят, что $\Omega_1$ \emph{слабее (грубее)} $\Omega_2$, или же $\Omega_2$ \emph{сильнее (точнее)} $\Omega_1$.

    Из определения видно, что дискретная топология --- самая сильная, а антидискретная --- самая слабая.

    \theorem{
        Для множества $X$ с двумя метриками $d_1$ и $d_2$ топология $\Omega_{d_1}$ слабее $\Omega_{d_2}$, если и только если
        \[\forall B_{r_1}^{d_1}(a): \exists r_2: B_{r_1}^{d_1}(a) \supseteq B_{r_2}^{d_2}(a)\]
        \provetwhen{
            Рассмотрим шар $B_{r_1}^{d_1}(a)$. Он открыт в первой топологии, но первая --- слабее, значит, открыт во второй.
            Значит, $\exists r_2: B_{r_1}^{d_1}(a) \supseteq B_{r_2}^{d_2}(a)$
        }{
            Для множества $U$, открытого в первой топологии, найдётся объединению шаров в первой топологии, равное ему.

            Из условия теоремы каждый такой шар содержит шар во второй метрике.
            Объединив их все, получим, что $U$ открыто во второй топологии.
        }
    }
    \newlection{15 ноября 2022 г.}
    \corollary{
        Для двух метрик $d_1$ и $d_2$, определённых на одном и том же множестве $X$, условие $\forall x, y \in X: d_1(x, y) \le d_2(x, y)$ влечёт условие:
        топология $(X, d_1)$ грубее топологии $(X, d_2)$.
    }
    \definition[(Топологически) эквивалентные метрики]{
        Метрики, порождающие одно и то же топологическое пространство.
    }
    \note{
        Для $C \in \R_{>0}$ и метрики $d_1 : X \times X \map \R_{\ge 0}$ функция $C d_1$ --- тоже метрика, причём эквивалентная $d_1$.
        \provehere{Рассмотреть шары.}
    }
    \note{
        Метрика $d_1$ грубее метрики $d_2$, если $\exists C > 0: \forall x, y: d_1(x, y) \le C \cdot d_2(x, y)$.
        \provehere{$Cd_2$ эквивалентна $d_2$ и точнее $d_1$.}
    }
    \definition[Липшицево эквивалентные метрики $d_1$ и $d_2$]{\label{lipshitz_equivalence}
    Такие метрики, что $\exists c, C > 0: \forall x, y \in X: c \cdot d_2(x, y) \le d_1(x, y) \le C \cdot d_2(x, y)$.
    }
    \note{
        Липшицева эквивалентность влечёт топологическую эквивалентность.
        Обратное в общем случае неверно: метрики $|x - y|$ и $\sqrt{|x - y|}$ на $\R$ эквивалентны лишь топологически.
    }

    Так, на прямом произведении множеств метрики
    \begin{gather*}
        d((A_1, A_2), (B_1, B_2)) = \sqrt{d_1(A_1, B_1)^2 + d_2(A_2, B_2)^2}.
        \tilde{d}((A_1, A_2), (B_1, B_2)) = \max{d_1(A_1, B_1)^2, d_2(A_2, B_2)^2}.
    \end{gather*}
    билипшицево эквивалентны (что такое билипшицево эквивалентны?
    Тем не менее, просто липшицева эквивалентность понятна).
    Более точно, $\forall x, y: \tilde{d}(x, y) \le d(x, y) \le \sqrt{2}\tilde{d}(x, y)$.

    На обычной плоскости $\R^2$ метрики
    \begin{gather*}
        \sqrt{(x_1 - x_2)^2 + (y_1 - y_2)^2}\\
        \max\{|x_1 - x_2|, |y_1 - y_2|\}\\
        |x_1 - x_2|+ |y_1 - y_2|
    \end{gather*} липшицево эквивалентны.
    Коэффициенты не превосходят $2$.


    \section{Специальные точки множеств в топологии}
    Рассмотрим произвольную топологию $(X, \Omega)$.

    \subsection{Внутренность множества. Внутренние точки}
    \definition[Внутренность множества $A$]{Наибольшее по включению открытое множество, содержащееся в $A$, как подмножество.}
    \note{Существование следует из того, что объединение любого количества открытых множеств открыто.}

    \subsubsection{Свойства внутренности}
    \bullets{
        \item $\Int A \subset A$
        \item Для открытого $B: B \subset A \then B \subset \Int A$.
        \item $A = \Int A \iff A$ открыто.
        \item $\Int(\Int A) = \Int A$.
        \item $A \subset B \then \Int A \subset \Int B$.
        \item $\Int (A \cap B) = \Int A \cap \Int B$.
        \provebullets {
            \item[$\subseteq$] $A \cap B \subseteq A \then \Int (A \cap B) \subseteq \Int A$.
            \item[$\supseteq$] $\left.\begin{aligned}
                                          \Int A \subseteq A& \\ \Int B \subseteq B&\\
            \end{aligned}\right\} \then \Int A \cap \Int B \subseteq A \cap B \overset{\Int \cap \Int\text{ открыто}}\then \Int A \cap \Int B \subseteq \Int (A \cap B)$.
        }
        \item $\Int (A \cup B) \supset \Int A \cup \Int B$
        \singlepage{
            \provebullets{
                \item[$\supseteq$] $\left.\begin{aligned}
                                              \Int A \subseteq A& \\ \Int B \subseteq B&\\
                \end{aligned}\right\} \then \Int A \cup \Int B \subseteq A \cup B \overset{\Int \cup \Int\text{ открыто}}\then \Int A \cup \Int B \subseteq \Int (A \cup B)$.
                \item [$\ne$] в общем случае: Для $A = \Q$ и $B = \I$ $\Int (A \cup B) = \R$, но $\Int A \cup \Int B = \o$.
            }}
    }
    \definition[Окрестность точки $x \in (X, \Omega)$]{Любое открытое множество, содержащее $x$.}
    \definition[Внутренняя точка]{Содержится с некой своей окрестностью.}
    Теорема: внутренность множества --- множество его внутренних точек.
    Доказательство: Докажем два включения. $\forall b \in B$ рассмотрим окрестность $U(b)$, как внутренней точки $A$.
    Это открытое подмножество $A$, значит, $U(b) \subset \Int A$.
    Отсюда $B \subset \Int A$.
    С другой стороны, для любой точки из $\Int A$ верно, что она внутренняя --- подходящей окрестностью является сама $\Int A$.

    Следствие: А открыто -- все его точки внутренние.

    \subsection{Замыкание множества. Точки прикосновения}
    \definition[Замыкание множества $A$]{
        Пересечение всех замкнутых множеств, его содержащих.

        Обозначается $\cl A$ или $\clos A$.
    }
    \note{Замыкание --- наименьшее замкнутое множество, содержащее данное. $\cl A = \bigcap\limits_{X \bs \Fc \in \Omega \land \Fc \supset A}\Fc$}

    \subsubsection{Свойства замыкания}
    \bullets{
        \item Замыкание замкнуто
        \item $A \subset \cl A$.
        \item Для замкнутого $B: B \supset A \then B \supset \cl A$.
        \item $A = \cl A \iff A$ --- замкнуто.
        \item $A \subset B \then \cl A \subset \cl B$.

        \item $\Cl (A \cup B) = \cl A \cup \cl B$
        \item $\cl (A \cap B) \subset \cl A \cap \cl B$
        \item $\cl A = \Int \overline{A}$
    }
    \definition[Точка $x$ --- точка прикосновения $A$] {$\forall U(x): U(x) \cap A \ne \o$.}
    \theorem{
        Для произвольного $A: \cl A = \defset{x \in X}{x\text{ --- точка прикосновения } A}$.
        \provehere{
            $\Cl A = X \bs \Int(X \bs A)$, откуда, видно, что $\Cl A$ --- действительно, множество точек, не содержащих окрестности, которая не пересекается с $A$.
        }
    }
    \corollary{Множество замкнуто $\iff$ оно совпадает со множеством точек прикосновения.}

    \subsection{Граница множества, граничные точки}
    \definition[Граница множества]{Точки, лежащие в замыкании, но не во внутренности: $\Fr A = \Cl A \bs \Int A$.}
    \note{Точка $x$ --- граничная для $A$, если любая окрестность точки $x$ пересекается и с $A$, и с $\overline{A}$.}
    \theorem{
        Граница множества совпадает со множеством граничных точек.
        \provehere{
            \[(x \in \Fr A) \iff (x \in \Cl A \land x \notin \Int A) \iff (\forall U(b): U(b) \cap A \ne \o \land U(b) \cap \overline{A} \ne \o) \iff x \in \Fr A\]
        }
    }

    \subsubsection{Свойства}
    \bullets{
        \item Граница --- замкнутое множество (как пересечение $\cl A$ и $\overline{\Int A}$).
        \item $\Fr A = \Fr (X \bs A)$
        \item $A$ --- замкнуто $\iff A \supset \Fr A$.
        \item $A$ --- открыто $\iff A \cap \Fr A \ne \o$.
    }

    \subsection{Предельные, изолированные точки}
    \definition[$x$ --- предельная точка $A$]{
        Любая проколотая окрестность пересекается с $A$. \[\forall U(x) \ni x: (U(x) \bs x) \cap A \ne \o\]
    }
    \definition[$x$ --- изолированная точка $A$]{
        Существует проколотая окрестность, не пересекающаяся с $A$. \[\exists U(x) \ni x: (U(x) \bs x) \cap A = \o\]
    }

    \subsubsection{Свойства}
    \bullets{
        \item Предельные точки включают точки прикосновения.
        \item $\Cl A = \Int A \sqcup \Fr A = \{\text{предельные точки }A\} \sqcup \{\text{изолированные точки }A\}$.
    }


    \section{База топологии}
    \definition[Для $(X, \Omega)$ $\Sigma \subset \Omega$ --- база топологии]{
        $\forall U \in \Omega: \exists \Gamma_U \subset U: U = \bigcup\limits_{w \in \Gamma_U}w = \bigcup\Gamma_U$.
    }
    \precaution{База не единственна; в качестве $\Sigma$ всегда можно рассмотреть $\Omega$, но хочется поменьше.}
    Так, для метризуемой топологии в качестве базы можно рассмотреть множество всех открытых шаров;
    Для топологии на прямой можно рассмотреть множество всех шаров с рациональным радиусом, или даже радиусом $\nicefrac{1}{n}$.

    \definition[$\Gamma \subset 2^X$ --- покрытие $X$]{
        $\bigcup\Gamma = X$.
    }
    В частности, любая база топологии --- покрытие $X$, так как $X$ --- открыто в любой топологии.

    \theorem{
        Для $(X, \Omega)$ $\Sigma$ --- база топологии $\Omega \iff \forall U \in \Omega, \forall a \in U: \exists w \in \Sigma: a \in w \subset U$.
        \provetwhen{
            Пусть $U = \bigcup\limits_{w \in \Gamma}{w}$.

            Тогда для любого $a \in U: a$ содержится в каком-то $w \in \Gamma$, и, действительно, $a \in w \subset U$.
        }{
            Построим для открытого $U \in \Omega$ множество $\Gamma_U$ из определения базы.
            Согласно посылке теоремы $\forall a \in U: \exists w_a \in \Sigma: a \in w_a \subset U$.

            Рассмотрим в качестве $\Gamma = \{w_a\}_{a \in U}$.
        }
    }
    \definition[$\Sigma_a$ --- база топологии в точке $a \in X$ (база окрестностей)]{
        $\forall w \in \Sigma_a: w \ni a$ и $\forall U(a): \exists w \in \Sigma_a: w \in U(a)$.
    }
    \note{
        Для $\Sigma$ --- базы $\Omega$: $\Sigma_a = \defset{w \in \Sigma}{a \in w}$ --- база топологии в точке $a$.
    }
    \note{
        Обратно: $\bigcup\{\Sigma_a\}$ --- база топологии.
    }
    В метризуемой топологии базой точки является, например, совокупность шаров с центром в данной точке.
    \theorem[Критерий базы]{
        Рассмотрим $\Sigma = \{B_\alpha\}_{\alpha \in \Lambda}$ --- некоторое покрытие.

        $\Sigma$ --- база некоторой топологии $\iff \forall \alpha_1, \alpha_2: B_{\alpha_1} \cap B_{\alpha_2}$ представимо, как объединение некоторого подмножества $B$.
        \provetwhen{
            По определению базы топологии (пересечение открытых множеств открыто)
        }{
            Построим топологию $\Omega$ над базой $\Sigma$ и проверим, что это --- топология.

            $\Omega \coloneqq \defset{\bigcup\limits_{\alpha \in S}B_\alpha}{S \subset \Lambda}$.

            \numbers{
                \item$\o, X$ принадлежат $\Omega$ (последнее --- так как $\Sigma$ --- покрытие $X$).
                \item Объединение двух множеств из $\Omega$ очевидно принадлежит $\Omega$.
                \item Проверим, что пересечение двух множеств из $\Omega$ принадлежит $\Omega$.

                Пусть $U_1 = \bigcup\limits_{\alpha \in S_1}B_\alpha$, $U_2= \bigcup\limits_{\alpha \in S_2}B_\alpha$.
                Несложно видеть, что $U_1 \cap U_2 = \bigcup\limits_{\alpha_1 \in S_1, \alpha_2 \in S_2}\left(B_{\alpha_1} \cap B_{\alpha_2}\right)$.

                Но согласно свойству, что пересечение элементов $\Sigma$ является объединением некоторого его подмножества, автоматически получаем, что $U_1 \cap U_2$ --- объединение некоторого подмножества $\Sigma$.
            }

        }
    }
    Для покрытия $\Sigma$, удовлетворяющему условию теоремы, обозначим топологию, задаваемую $\Sigma$ следующим образом: $\Omega(\Sigma)$.
    \note{
        $\Omega(\Sigma)$ --- наименьшая по включению топология, содержащая $\Sigma$.
        (Наименьшая по включению топология существует, так как пересечение топологий --- топология).
    }
    \ok
    Построим топологию из произвольного набора подмножеств $\Delta \subset 2^X$.
    Тогда построим $\Sigma(\Delta)$, как все конечные пересечения элементов $\Delta$ (и само множество $X$).
    \[\Sigma(\Delta) = \{X\} \cup \defset{\bigcap\limits_{i = 1}^{k}}{k_i \in \N, w_i \in \Delta}\]

    Такая база топологии $\Sigma(\Delta)$ удовлетворяет критерию базы, на ней можно построить топологию $\Omega(\Sigma(\Delta))$.

    Такая $\Delta$ называется \emph{предбаза} --- множество подмножеств такое, что база --- объединение конечных пересечений его элементов.
    \note{$\Delta$ --- предбаза топологии $\Omega$, если $\Omega$ --- наименьшая по включению топология, содержащая $\Delta$. }


    \section{Подпространства}
    Рассмотрим $A \subset X$ для топологического пространства $(X, \Omega)$.
    Обозначим $\Omega_A = \defset{U \cap A}{U \in \Omega}$.
    Такая $\Omega_A$ --- \emph{топология, индуцированная на подпространстве $A$}.

    Несложно убедиться, проверив три аксиомы, что $\Omega_A$ --- топология на множестве $A$.
    \newlection{22 ноября 2022 г.}

    \subsection{Свойства подпространства}
    \bullets{
        \item Множество, открытое в подпространстве, необязательно открыто.
        Так, подпространство всегда открыто в себе.

        \item Тем не менее, в открытом подпространстве открытые подмножества исходно открыты.
        \item Для $\Sigma$ --- базы исходной топологии --- можно определить базу для подпространства $A$, как $\Sigma_A = \defset{A \cap U}{U \in \Sigma}$.

        \item Пусть $B \subset A \subset X$. Тогда топологии $\Omega_B$ и $(\Omega_A)_B$ совпадают.
    }
    Пусть $A \subset X$, где на $X$ определена метрика $d$.

    На $A$ можно ввести топологию двумя способами:
    \numbers{
        \item Топология, индуцированная на метрике-сужении $d\big|_A$, называется $\Omega_{d_A}$
        \item Подпространство топологии $(X, \Omega_d)$. Для данной теоремы назовём её $\Omega^X_A$.
    }
    \theorem{
        Эти топологии совпадают.
        \provehere{
            Проверим, что для всякой пары \{точка; открытое множество, её содержащее\} из первой топологии, точка содержится в меньшем по включении открытом множестве из второй топологии.
            И наоборот.

            Так как про обе конструкции понятно, что они являются топологиями, то этой проверки будет достаточно.
            \bullets{
                \item $\Omega_{d_A} \subseteq \Omega_A^X$.

                Зафиксируем точку $a \in A$ и открытое множество $U \in \Omega_{d_A}~(a \in U)$.
                Так как $\Omega_{d_A}$ --- индуцирована на метрике, то $a$ содержится с неким открытым шаром радиуса $r$ внутри $U$.

                Этот же шар содержится в исходной топологии $\Omega$

                \item $\Omega_{d_A} \subseteq \Omega_A^X$.

                Зафиксируем точку $a \in A$ и открытое множество $U \in \Omega_A^X~(a \in U)$.

                $U = V \cap A$ для некоего $V \in \Omega$.
                В множестве $V: a$ содержится вместе с неким шаром радиуса $r$.

                В топологии $\Omega_A^X$ есть как раз пересечение этого шара и множества $A$ (так как этот шар был в $\Omega$).

                \emph{
                    Вроде бы доказательство правильное, на лекции было что-то странное, я, к сожалению, как раз отвлёкся, поэтому вышенаписанное --- частично моя импровизация.
                    Тем не менее, не понимаю, почему на доске была разность каких-то радиусов.
                }
            }
        }
    }


    \section{Произведение метрических пространств}
    Пусть даны метрически пространства $(X, \Omega_X)$ и $(Y, \Omega_Y)$.

    Положим $\Sigma \coloneqq \defset{U \times V}{U \in \Omega_X, V \in \Omega_Y}$.

    \theorem{
        $\Sigma$ --- база некой топологии.
        \provehere{
            Проверим критерий базы.

            $\forall (u_1 \times v_1), (u_2 \times v_2) \in \Sigma: (u_1\times v_1) \cap (u_2 \times v_2) = (u_1 \cap u_2) \times (v_1 \cap v_2)$.

            Данная формула красиво обосновывается через пересечение прямоугольников.

            Из свойства топологии $u1 \cap u_2 \in \Omega_X$ и $v_1 \cap v_2 \in \Omega_Y$.
        }
    }
    \definition[Стандартная топология на произведении пространств]  {
        Топология, построенная на базе $\Sigma$, определённой выше.
    }
    \example{
        Пусть $\Omega$ --- множество метрически открытых множеств в $\R$.

        Тогда стандартная топология на $(R, \Omega) \times (\R, \Omega)$ --- стандартная топология плоскости.
        \provehere{
            В базе $\Sigma$ содержится <<открытый квадрат>> с данным центром и сколь угодно малым радиусом
        }
    }
    \note{
        Перемножать можно не сами топологии, а их базы $\Sigma_X$ и $\Sigma_Y$, всё равно будет база стандартной топологии произведения $\Sigma = \defset{u \times v}{u \in \Sigma_X, v \in \Sigma_Y}$.
    }
    \ok
    Рассмотрим два метрических пространства $(X, d_x)$ и $(Y, d_y)$.
    На их произведении $X \times Y$ топологию можно ввести двумя способами:
    \numbers{
        \item Индуцировать топологию на стандартной метрике произведения пространств $d((x_1, y_1), (x_2, y_2)) = \sqrt{d_x(x_1, x_2)^2 + d_y(y_1, y_2)^2}$.
        \item Перемножить, как топологические пространства $(X, \Omega_{d_x})$ и $(Y, \Omega_{d_y})$.
    }
    \theorem{
        Эти топологии совпадают.
        \prove{
            Вместо $d = \sqrt{d_x^2 + d_y^2}$ будем рассматривать ей липшицево эквивалентную~(\cref{lipshitz_equivalence}) метрику $\tilde{d} = \max\{d_1, d_2\}$.
            Она индуцирует ту же топологию.

            Заметим, что эта топология порождаются базой $\Sigma = \defset{B_r^{\tilde{d}}((x, y))}{x \in X, y \in Y, r \in \R_{>0}}$.

            Но $\Sigma = \defset{B_r^{\tilde{d}}((x, y))}{x \in X, y \in Y, r \in \R_{>0}} = \defset{B_r^{d_x}(x) \times B_r^{d_y}(y)}{x \in X, y \in Y, r \in \R_{>0}}$.

            Теперь видно, что $\Sigma$ является базой $(X, \Omega_{d_x}) \times (Y, \Omega_{d_y})$ тоже.
            В самом деле, если перемножить базы --- открытые шары в $d_x$ и $d_y$, то получится база $\Sigma' = \defset{B_{r_1}^{d_x}(x) \times B_{r_2}^{d_y}(y)}{x \in  X, y \in Y, r_1, r_2 \in \R_{>0}}$.
            Но --- так как можно взять из радиусов наименьший --- мы поймём, что $\Omega_{\Sigma'} = \Omega_{\Sigma}$.
        }
    }

    \subsection{Тихоновская топология прямого произведения бесконечного числа пространств}
    Рассмотрим множество пространств, проиндексированное $\Lambda$: $\{(X_\alpha, \Omega_\alpha)\}_{\alpha \in \Lambda}$.

    \definition[Прямое произведение множеств $\{X_\alpha\}_{\alpha \in \Lambda}$]{\down
    Множество функций $X = \defset{x \in \left(\bigcup\limits_{\alpha \in \Lambda}X_\alpha\right)^\Lambda}{\forall \alpha \in \Lambda: x(\alpha) \in X_\alpha}$.
    }
    \definition[Координатная проекция]{\label{projection_def}
    Функция $p_\alpha: X \map X_\alpha$: $p_\alpha(x) = x(\alpha)$.
    }
    \definition[Цилиндр]{
        Подмножество произведения, открытое в одном сомножителе, и совпадающее с другими.
        Формально, $p^{-1}_\alpha(U)$ для некоего $\alpha \in \Lambda$ и $U \in \Omega_\alpha$.
    }
    \definition[Тихоновская топология произведения пространств $\{(X_\alpha, \Omega_\alpha)\}_{\alpha \in \Lambda}$]{
        Топология строится с помощью предбазы $\Delta \coloneqq \defset{p^{-1}_\alpha(U)}{\alpha \in \Lambda, U \in \Omega_\alpha}$.
    }
    \note{
        Для $|\Lambda| < \infty$ топология совпадает с ранее определённой.
    }


    \chapter{Непрерывные отображения}


    \section{Свойства образа и прообраза}
    Задана функция $f: X \map Y$.
    \definition[Образ $A \subset X$]{
        $f(A) \bydef \defset{f(a)}{a \in A}$.
    }
    \definition[Прообраз $B \subset Y$]{
        $f^{-1}(B) \bydef \defset{a \in A}{f(a) \in B}$.
    }
    \numbers{
        \item Прообраз объединения --- объединение прообразов $f^{-1}(A \cup B) = f^{-1}(A) \cup f^{-1}(B)$.
        \item  Прообраз пересечения --- пересечение прообразов $f^{-1}(A \cap B) = f^{-1}(A) \cap f^{-1}(B)$.
        \item Прообраз дополнения --- дополнение прообраза $f^{-1}(Y \sm U) = X \sm f^{-1}(U)$.
        \item Образ объединения --- объединение образов $f(A \cup B) = f(A) \cup f(B)$.
        \item Образ пересечения \textbf{содержится в} пересечении образов $f(A \cap B) \subseteq f(A) \cap f(B)$.
        \counterexample{
            $f: x \mapsto x \pmod{2}$.
            \[\{1\} = f(\{0, 1\} \cap \{1, 2\}) \subseteq f(\{0, 1\}) \cap f(\{1, 2\}) = \{0, 1\}\]
        }
    }
    \definition[Тождественное отображение]{
        $\id_X: X \map X; \quad x \mapsto x$.
    }
    \definition[Вложение $A \subset X$ в $X$]{
        $\text{in}_{A\map X}: A \map X; \quad x \mapsto x$.
    }


    \section{Непрерывность отображения}
    Пусть $X, Y$ --- топологические пространства.
    \definition[Непрерывное отображение $f: X \map Y$]{
        Отображение, в открытые множества переводящее только открытые множества. $\forall U \in \Omega_Y: f^{-1}(U) \in \Omega_X$.
    }
    \note{
        Применив дополнение, очевидно, что альтернативным определением является то же про замкнутые множества: $\forall V \notin \Omega_Y: f^{-1}(V) \notin \Omega_X$
    }
    \example{
        $\id_X$ непрерывно: прообраз всякое открытого множества открыт.
    }
    \example{
        $f(x) = c$ --- постоянное отображение --- непрерывно: прообраз всякое множества открыт (либо $X$, либо $\o$).
    }
    \example{
        Если в $X$ много открытых множеств, или в $Y$ мало, то $f: X \map Y$ непрерывна. Так, непрерывны функции, определённые на дискретном $X$ и/или действующие в антидискретное $Y$.
    }
    \note{
        Если $X_2$ тоньше $X_1$, а $Y_2$ грубее $Y_1$, то $f: X_1 \map Y_1$ непрерывна $\then f: X_2 \map Y_2$ непрерывна.
    }
    \theorem{
        Композиция непрерывных функций непрерывна.

        Так, пусть $f: X \map Y$ и $g: W \map X$.
        \provehere{$\forall U \in \Omega_W: g^{-1}(U) \in \Omega_X \then (f\circ g)^{-1}(U) = f^{-1}(g^{-1}(U)) \in \Omega_Y$.}
    }
    \corollary{Если $f$ непрерывна, то её сужение $f\big|_W $ непрерывно.
    \provehere{Рассмотреть $g = \text{in}_{W \map X}$.}
    }
    Пусть $f: X \map Y$, а множество $Z$ таково, что $f(x) \subset Z \subset Y$.

    Положим $\tilde{f}: X \map Z$, $\tilde{f}(x) = f(x)$.
    \theorem{
        $f$ непрерывна $\iff$ $\tilde{f}$ непрерывна.
        \provetwhen{
            $f = \text{in}_Z \circ \tilde{f}$
        }{
            Всякое открытое множество в $Z$ имеет вид $w = u \cap Z$ для некоего $U \in \Omega_Y$.
            \[f^{-1}(w) = f^{-1}(u) \cap f^{-1}(Z) = f^{-1}(u)\]
        }
    }


    \section{Локальная непрерывность}
    \definition[$f: X \map Y$ непрерывна в $a \in X$]{
        $\forall U \ni f(a): \exists V \ni a: f(V) \subseteq U$.
    }
    \theorem{
        Функция $f$ непрерывна $\iff$ во всякой точки области определения функция непрерывна.
        \provetwhen{
            Очевидно из определений.
        }{
            Рассмотрим $U \in \Omega_Y$.
            Проверим, что $f^{-1}(U)$ открыто. $f^{-1}(U) = \bigcup\limits_{x \in f^{-1}(U)}V(x)$, где $V(x)$ --- окрестность точки $x$, содержащая образ в $U$. Объединение открытых --- открыто. значит, $f^{-1}(U)$ в самом деле открыто.
        }
    }
    \note{
        Условие локальной непрерывности можно проверять не на всех открытых множествах, а только на базах окрестностей $\Sigma_a$ и $\Sigma_{f(a)}$.
    }
    \corollary{
        Для метрических пространств $X, Y$ удобно рассмотреть в качестве базы множество открытых шаров.
        Определение непрерывности в таком случае переписывается так:

        $f$ непрерывна в точке $a$: $\iff \forall \eps > 0: \exists \delta > 0: f(B^{d_X}_\delta(a)) \subset B^{d_Y}_{\eps}(f(a))$.

        $f$ непрерывна в точке $a$: $\iff \forall \eps > 0: \exists \delta > 0: d_X(x - a) < \delta \then d_Y(f(x) - f(a)) < \eps$.
    }
    \emph{Узнали? Согласны?}
    \definition[Липшицево отображение между метрическими пространствами]{
        Такое отображение $f: X \map Y$, что $\exists C \in \R_{>0}: \forall a, b \in X: d_Y(f(a), f(b)) \le C \cdot d_X(a, b)$.

        Константа $C$ из определения --- константа Липшица.
        Отображение, липшицевое с константой $c$ называется $c$-липшицевым.
    }
    \theorem{
        Липшицево отображение непрерывно.
        \provehere{Легко проверить, что оно непрерывно в любой точке.}
    }
    \examples{
        \item Пусть $x_0 \in X$. Положим $d_{x_0}(y) \bydef d(x_0, y)$. Утверждается, что $d_{x_0}$ --- $1$-липшицево.
        \item Более общий случай:
        пусть $A \subset X$. Положим $d_{A}(y) \bydef d(A, y) \bydef \inf\limits_{a \in A}d(a, y)$. Утверждается, что $d_{A}$ --- $1$-липшицево.
        \provehere{
            По определению инфимума $\forall \tau > 0: \exists a_y \in A: d(y, a_y) < d(y, A) + \tau$.

            Тогда $d(x, A) \le d(x, a_y) \le d(x, y) + d(y, a_y) \le d(x, y) + d(y, A) + \tau$ --- дважды применили неравенство треугольника.

            Используя $\forall \tau > 0: d(x, A) \le d(x, y) + d(y, A) + \tau$, получаем $d(x, A) \le d(x, y) + d(y, A)$.

            Аналогично-симметрично $d(y, A) \le d(x, y) + d(x, A)$, то есть $|d_A(x) - d_A(y)| \le d(x, y)$.
        }
        \item Пусть $d$ --- произвольная метрика на $X$. $d: X \times X \map R$.
        Утверждается, что $d$ --- липшицево отображение.

        Коэффициент зависит от того, как определена метрика на произведении. Для $d = \sqrt{d_X^2 + d_Y^2}$ этот коэффициент равен $\sqrt{2}$.

        \prove{
            Рассмотрим две произвольные точки из области определения: $(a, b), (x, y) \in X \times X$.

            Без потери общности предположим, что $d(x, y) \ge d(a, b)$. В таком случае $|d(x, y) - d(a, b)| = d(x, y) - d(a, b)$.

            $d(x, y) - d(a, b) \le d(x, a) + d(a, y) - d(a, b) \le d(x, a) + d(y, b) \le \sqrt{2}\sqrt{d(x, a)^2 + d(y, b)^2}$, что по определению равно $\sqrt{2}\cdot d((x, y), (a, b))$.

            Здесь мы воспользовались двумя неравенствами треугольника, а также тем, что $s + t \le \sqrt{2}\sqrt{s^2 + t^2}$, что очевидно после возведения в квадрат.
        }
    }


    \newlection{26 ноября 2022 г.}


    \section{Гомеоморфизмы}
    \definition[Гомеоморфизм]{
        Непрерывное отображение $f: (X, \Omega_X) \map (Y, \Omega_Y)$, такое, что $f$ --- биекция, причём $f^{-1}$ --- тоже непрерывно.

        Если между $(X, \Omega_X)$ и $(Y, \Omega_Y)$ существует гомеоморфизм, то говорят, что $(X, \Omega_X)$ гомеоморфно $(Y, \Omega_Y)$, пишут $(X, \Omega_X) \sim (Y, \Omega_Y)$.
    }
    \theorem{
        Гомеоморфность --- отношение эквивалентности на множестве топологических пространств.
        \provebullets{
            \item $\id$ --- гомеоморфизм.
            \item Если $f$ --- гомеоморфизм, то $f^{-1}$ --- гомеоморфизм.
            \item Композиция гомеоморфизмов --- гомеоморфизм.
        }
    }
    \examples{
        \item $X = \{a, b\}$. Для топологий $\Omega_1 = \{\o, X, \{a\}\}$ и $\Omega_2 = \{\o, X, \{b\}\}$ $(X, \Omega_1) \sim (X, \Omega_2)$.
        Гомеоморфизм --- функция $f(x) = \all{a,& x = b \\ b,& x = a}$.

        \item Всякие два отрезка с одинаковым типом концом гомеоморфны: $[a, b] \sim [c, d]$. Можно построить непрерывное линейное отображение.
        \item $(-\frac{\pi}{2}; \frac{\pi}{2}) \sim \R$. В качестве непрерывного отображения может выступать функция $y = \tan(x)$.
        \item На плоскости $\R^2$ всякие два шара (два открытых, или два замкнутых) гомеоморфны.

        \item Квадрат гомеоморфен кругу: можно рассмотреть отображение, линейно переводящее <<радиусы>> в радиусы.

        \item \up \intfact{Более того, всякие два выпуклых непустых замкнутых множества гомеоморфны друг другу.}

        \item $S^n \sm \{\text{точка}\} \sim \R^n$. $S^n$ --- стандартная сфера в пространстве $\R^{n + 1}$, так, $S^1$ --- окружность.

        \provehere{
            Рассмотрим сферу, а на ней --- два полюса $A$ и $B$.
            Проведём касательную плоскость $\alpha$ через точку $B$; всякой точке $C \in S^n$ сопоставим пересечение луча $AC$ и плоскости $\alpha$.

            Проверить, что это гомеоморфизм, можно с помощью инверсии с центром в точке $O$ (центр сферы) и радиусом $R$ (радиус сферы).

            Применив инверсию к плоскости $\alpha$, получим сферу, построенную на $BO$, как на диаметре.

            Таким образом, после сужения инверсии, получается отображение из плоскости в сферу без точки.

            Доказательство того, что инверсия непрерывна, будет чуть позже.
        }

        \item Круг без точки гомеморфен кольцу --- кругу без круга. Опять же, отображение линейно переводит радиусы в радиусы.

        \item Если из пространства выкинуть окружность, то это будет то же самое, что и выкниуть прямую и точку.
        \[\R^3 \sm S^1 \sim \R^3 \sm (R^1 \cup \{\text{точка вне прямой}\})\]

        \item Пример непрерывной биекции, не являющейся гомеоморфизма:
        $f: [0; 2 \pi) \map S_1$, такая, что $f(x) = (\cos(x), \sin(x))$.

        Обратное отображение не является непрерывным, так как $[0; 1)$ открыто в $[0; 2 \pi)$, но $f([0; 1))$ --- отнюдь не открытое подмножество окружности.
    }


    \section{Фундаментальные покрытия}
    \definition[Фундаментальное покрытие пространства $X$]{\label{fundamental_covering_def}
    Такое покрытие $\Gamma = \{A_\alpha\}_{\alpha \in \Lambda}$ топологического пространства $(X, \Omega)$, что \[\forall u \subset X: \left(u \in \Omega \iff \forall \alpha \in \Lambda: u \cap A_\alpha \in \Omega_{A_\alpha}\right)\]
    }
    \note{
        Аналогично можно рассмотреть не открытые, а замкнутые множества: $F$ замкнуто в $X \iff \forall \alpha: F \cap A_\alpha$ замкнуто в $A_\alpha$.
    }

    \theorem{
        Пусть $\{A_\alpha\}_{\alpha \in \Lambda}$ --- фундаментальное покрытие $X$.

        Если $f: X \map Y$ таково, что $\forall \alpha: f\big|_{A_\alpha}$ --- непрерывно, то само $f$ --- непрерывно.

        \provehere{
            Рассмотрим произвольное открытое множество $u \in \Omega_Y$. Докажем, что $f^{-1}(u) \in \Omega_X$.

            Для произвольного $\alpha \in \Lambda:$ \[\left(f\big|_{A_\alpha}\right)^{-1}(u) \in \Omega_{A_\alpha} \then f^{-1}(u) \cap A_\alpha \in \Omega_{A_\alpha}\]
            откуда по определению фундаментального покрытия $f^{-1}(u) \in \Omega_X$.
        }
    }
    \counterexample{
        Для $X = [0; 1)$ можно рассмотреть покрытие $[0; 1) = \left[0; \frac{1}{2}\right) \cup \left[\frac12; 1\right)$.
        Оно не является фундаментальным, так как $f(x) = \{x\}$ --- взятие дробной части --- не непрерывно, хотя непрерывно на каждом полуинтервале из покрытия.

        В то же время покрытие $[0; 1) = \left[0; \frac12\right] \cup \left[\frac12; 1\right)$ --- фундаментально (например, см.~(\cref{fundamental_coverings_detection})).
    }
    \singlepage{\definition[Открытое покрытие]{
        Такое покрытие, что все его элементы открыты в $X$.
    }
    \definition[Замкнутое покрытие]{
        Такое покрытие, что все его элементы замкнуты в $X$.
    }
    \definition[Локально конечное покрытие]{
        $\forall x \in X: \exists V_x \in \Omega_X: x \in V_x$ такая, что она пересекает конечное число элементов покрытия: $\defset{\alpha \in \Lambda}{V_x \cap A_\alpha \ne \o}$ конечно.
    }
    }
    \theorem{
        \label{fundamental_coverings_detection}\down
        \numbers{
            \item Открытое покрытие фундаментально.
            \item Конечное замкнутое покрытие фундаментально.
            \item Локально конечное замкнутое покрытие фундаментально.
        }
        \provebullets{
            \item[1.] Рассмотрим произвольное множество $U \subset X$.
            Если $\forall \alpha \in \Lambda: A_\alpha \cap U \in \Omega_{A_\alpha}$, то --- по тривиальному свойству топологии, индуцированной на открытом множестве --- $\forall \alpha \in \Lambda: A_\alpha \cap U \in \Omega_{X}$.
            В таком случае $U$ можно представить, как объединение всех таких частей.
            \item[2.] Рассмотрим произвольное множество $U \subset X$.
            Если $\forall \alpha \in \Lambda: A_\alpha \cap U \in \Omega_{A_\alpha}$, то дополнение ${A_\alpha \cap (X \sm U)}$ замкнуто в $A_\alpha$.
            По тривиальному свойству замкнутой индуцированной топологии получаем, что $X \sm U$ замкнуто во всех $A_\alpha$.

            В таком случае $X \sm U$ можно представить, как объединение всех таких частей (объединение конечного числа замкнутых --- замкнуто).
            Значит, $U$ открыто.

            \item[3.] Зафиксируем для каждой точки окрестность $V_x$, пересекающую конечное число элементов покрытия --- из определения локальной конечности.

            Рассмотрим произвольное множество $U \subset X$. Несложно видеть, что $U = \bigcup\limits_{x \in U}(U \cap V_x)$.

            Если все эти части открыты в $V_x$, где --- после сужения --- конечные замкнутые покрытия, то, объединив их все, получаем, что и само $U$ открыто.
        }
    }


    \section{Непрерывность и произведение пространств}
    Рассмотрим $X = \prod\limits_{\alpha \in \Lambda}X_\alpha$ --- произведение топологических пространств.
    \theorem{
        Координатные проекции $p_\alpha: X \map X_\alpha$~(\cref{projection_def}) непрерывны.
        \provehere{
            Всякое множество $U \in \Omega_{X_\alpha}$ такого, что $p^{-1}(U)$ открыто --- по определению, как элемент предбазы $\prod X_\alpha$.
        }
    }
    \definition[Координатная компонента $f$] {
        Пусть $f: Z \map \prod\limits_{\alpha \in \Lambda}X_\alpha$.
        Тогда компонентой $f$ по координате $\alpha$ называется $f_\alpha = p_\alpha \circ f$.
    }
    \theorem{
        $f: Z \map \prod\limits_{\alpha \in \Lambda}X_\alpha$ непрерывно $\iff \forall \alpha: f_\alpha$ непрерывно.
        \provetwhen {
            Композиция $p_\alpha \circ f$ непрерывна.
        }{
            Проверим, что $f$ непрерывна на элементах предбазы: прообразы $\forall U$ --- открытом в произведении --- $: p^{-1}(U) \in \Omega_{X_\alpha}$.

            Воспользуемся тем, что $p^{-1}_\alpha(U)$ открыт (по определению топологии произведения).
            Кроме этого, $f^{-1}\left(p^{-1}_\alpha\left(U\right)\right)$ тоже открыто, как прообраз открытого в $f$.

            Значит, $f^{-1}\left(p^{-1}_\alpha\left(U\right)\right) = (p_\alpha \circ f)^{-1}(U)$ --- открыто, откуда $p_\alpha \circ f$ непрерывно.
        }
    }
    \counterexample{
        <<Обратное>> неверно: не факт, что если  $f: \prod\limits_{\alpha \in \Lambda}X_\alpha \map Y$ непрерывно на всех проекциях, то оно непрерывно.

        Так, можно рассмотреть $f: \R^2 \map \R; \quad (x, y) \mapsto \all{\dfrac{xy^2}{x^2 y^4},&(x, y) \ne (0, 0)\\0,&x = y = 0}$.

        Такая функция непрерывна в сужении на любую прямую (в том числе и координатную), но не непрерывна:
        \provebullets{
            \item
            \bullets{
                \item Для всякой прямой, не проходящей через 0 ($y = kx + b$ или $x = b$, где $b \ne 0$) сужение функции на эту прямую имеет определённую формулу --- частное многочленов, где знаменатель строго положителен.
                Она непрерывна.

                \item Сужение на прямую $x = 0$ даёт $f(0, x) = 0$.

                \item Наконец, для прямых $y = kx$ сужение даёт функцию $f(x, kx) = \dfrac{k^2 x^3}{x^2 + k^4 x^4} = \dfrac{k^2 x}{1 + k^4 x^2}$ при $x \ne 0$.
                Не в нуле функция понятно, что непрерывна; в нуле $\dfrac{kx}{1 + k^4 x^2} \underset{x \to 0}\Map 0$
            }
            \item Несмотря на всё это, если сузить функцию на параболу $x = y^2$, то окажется, что $f(y^2, y) = \nicefrac{1}{2}$ при $y \ne 0$, однако $y = 0$ при $y = 0$.
            Эта функция непрерывной уже не является.
        }
    }


    \section{Арифметические операции над непрерывными функциями}
    \theorem{
        Функции $f: \R^2 \map \R$ непрерывны, где $f(x, y) = x + y$ или $f(x, y) = x - y$, или может быть $f(x, y) = x \cdot y$.
        (Доказываем для трёх функций)
        \provebullets{
            \item $f(x, y) = x + y$. Проверим непрерывность в точке: рассмотрим открытый шар $B_{\eps}(x_0 + y_0)$.
            Докажем, что в его прообразе есть окрестность $(x_0, y_0)$.

            Для этого возьмём $\delta = \frac{\eps}{2}$. Несложно видеть, что $f\left(B_\delta(x_0), B_\delta(y_0)\right) \subset B_{\eps}(x_0 + y_0)$: \[\forall x \in B_\delta(x_0), y \in B_\delta(y_0): |x_0 + y_0 - x - y| \le |x_0 - x| + |y_0 - y| \le \delta + \delta = \eps\]
            \item Аналогично.
            \item $f(x, y) = x \cdot y$. Проверим непрерывность в точке: рассмотрим открытый шар $B_{\eps}(x_0 \cdot y_0)$.
            Докажем, что в его прообразе есть окрестность $(x_0, y_0)$.

            Пусть $c = \max\{|x_0|, |y_0|\}$.

            Для этого возьмём $\delta = \min\{\frac{\eps}{3c}, \sqrt{\frac{\eps}3}\}$.
            \[\forall x \in B_\delta(x_0), y \in B_\delta(y_0): |x \cdot y - x_0 \cdot y_0| \le |(x \pm \delta)(y \pm \delta) - x_0 \cdot y_0| \le |x_0\delta| + |y_0 \delta| + |\delta^2| \le \eps\]
        }
    }
    \corollary{
        В топологическом пространстве $(X, \Omega)$ для непрерывных функции $f, g: X \map \R$ верно, что $f + g, f - g$ и $f \cdot g$ --- тоже непрерывны.
        \prove{
            Пусть функция $(f, g): X \map \R \times \R$ определена так: $(f, g)(x) = (f(x), g(x))$. Она непрерывна, так как проекции непрерывны;

            тогда \begin{gather*}
                      f + g = (x + y) \circ (f, g)\\
                      f - g = (x - y) \circ (f, g)\\
                      f \cdot g = (x \cdot y) \circ (f, g)
            \end{gather*} непрерывны, как композиции.
        }
    }
    \corollary{
        В топологическом пространстве $(X, \Omega)$ для непрерывных функции $f, g: X \map \R$ верно, что $\frac{f}{g}$ --- тоже непрерывна на своей области определения (где $g$ не обращается в 0).
        \prove{
            Рассмотрим $h: \R\sm \{0\} \map \R \sm \{0\}; \quad h(x) = \frac{1}{x}$. Как известно из матанализа, она непрерывна.

            Тогда $\frac{1}{g} = h \circ g$ и $\frac{f}{g} = f \cdot (h \circ g)$.
        }
    }


    \section{Топологические свойства}
    Как доказать, что два топологических пространства не являются гомеоморфными?

    При гомеоморфизме сохраняются некоторые свойства.
    Если эти свойства различны, то пространства заведомо не гомеоморфны.

    \definition[Топологическое свойство]{
        Свойство, которое пространства сохраняют при гомеоморфизме.

        Пространство может обладать или не обладать некоторым свойством.
    }
    \definition[Топологический инвариант]{
        Характеристика, которую пространства сохраняют при гомеоморфизме.

        Какое-то число, например.
    }

    \subsection{Аксиомы счётности}
    Ниже для краткости будем называть счётными \emph{не более, чем счётные} множества.
    \definition[Первая аксиома счётности, AC1]{
        Топологическое пространство $X$ удовлетворяет первой аксиоме счётности, если у любой точки существует счётная база.
    }
    Любое метрическое пространство удовлетворяет первой аксиоме счётности: можно взять у всякой точки открытые шары с центром в ней и радиусом $\frac{1}{n}$.
    \definition[Вторая аксиома счётности, AC2]{
        Топологическое пространство $X$ удовлетворяет второй аксиоме счётности, если у него существует счётная база.
    }
    \theorem{
        Из второй аксиомы счётности следует первая: в качестве базы точки можно взять все открытые множества, содержащие её: $\Sigma_a = \defset{U \in \Sigma}{a \in U}$.
    }
    \definition[Наследственное топологическое свойство]{
        Если всё пространство $X$ обладает свойством, то всегда любое его подпространство обладает этим же свойством.
    }
    \definition[Наследование свойства для произведения]{
        Если два пространства $X$ и $Y$ обладают свойством, то всегда $X \times Y$ тоже этим свойством обладает.
    }
    \fact{
        Вторая аксиома счётности --- наследственное свойство.
        Вторая аксиома счётности наследуется подпространством: если $\Sigma$ --- счётная база, то \[\Sigma_A \bydef \defset{U \cap A}{U \in \Sigma}\] тоже счётна.

        Вторая аксиома счётности наследуется и для произведения: для базы в точке $(x, y)$ достаточно взять базу $\Sigma_{(x, y)} = \Sigma_x \times \Sigma_Y$

    }
    \newlection{29 ноября 2022 г.}
    \note{
        $\R$ удовлетворяет второй аксиоме счётности: можно рассмотреть в качестве базы шары радиусом $\frac{1}{n}$ с рациональными центрами.

        Отсюда следует, что второй аксиоме счётности удовлетворяют и все подпространства $\R^n$.
    }
    \theorem[Линделёф]{\label{Lindeloef_theorem}
    Если $X$ удовлетворяет второй аксиоме счётности, то из любого покрытия можно выделить счётное подпокрытие.
    \prove{
        Рассмотрим покрытие $U: \bigcup\limits_{\alpha \in \Lambda}U_\alpha = X$.

        Обозначим в качестве $S \coloneqq \defset{v \in \Sigma}{\exists \alpha \in \Lambda: v \subset U_\alpha}$.

        $S$ --- покрытие, так как всякая точка лежит в $U_\alpha$ вместе с неким множеством из базы.

        Теперь для каждого $s \in S$ сопоставим один любой элемент из $U$, содержащий $s$. Таким образом мы выделим счётное подпокрытие.
    }
    }

    \subsection{Сепарабельные пространства}
    \definition[$A \subset X$ всюду плотно]{$\cl A = X$}
    \note{Это значит, что всякая точка пространства --- точка прикосновения $A$, то есть $\forall U \in \Omega_X \sm \{\o\}: U \cap A \ne \o$.}
    \definition[Сепарабельное пространство]{
        Пространство, в котором есть всюду плотное счётное множество.
    }
    \example{$\R$ сепарабельно: $\Q$ --- всюду плотное счётное подмножество.}
    \theorem{
        \numbers{\item Из второй аксиомы счётности следует сепарабельность.

        \item В метрических пространствах из сепарабельности следует вторая аксиома счётности.
        }
        \provebullets{
            \item[1.] Сопоставим всякой $v \in \Sigma$ одну из её точек. Это всюду плотное множество.
            \item[2.] Рассмотрим $\Sigma = \defset{B_{\frac{1}{n}}(x)}{x \in A; n \in \N}$, где $A$ --- всюду плотное счётное множество.

            Проверим, что $\Sigma$ --- база. Для этого рассмотрим любую точку и любое открытое множество, её содержащее $b \in U \in \Omega$.

            Проверим, что существует шар $B$ из базы, такой, что $b \in B \subset U$.
            Пространство метрическое, есть достаточно большое $k \in \N:$ шар $B_{\frac{1}{k}}(b)$, содержащийся в $U$.

            Тогда $\exists a \in B_{\frac{1}{k}} \cap A: d(a, b) < \frac{1}{2k}$, так как $A$ --- всюду плотно.

            Теперь понятно, что шар $b \in B_{\frac{1}{2k}}(a)$, и что этот шар --- из базы.
        }
    }

    \subsection{Аксиомы отделимости}

    \subsubsection{Первая аксиома отделимости}
    \definition[Пространство удовлетворяет первой аксиоме отделимости, T1]{
        $\forall x, y \in X: {\exists U_x \in \Omega: x \in U_x \not\ni y}$.
    }
    \theorem{
        Пространство удовлетворяет первой аксиоме отделимости $\iff$ все одноточечные множества замкнуты.
        \provetwhen{
            Рассмотрим $x \in X$. Проверим, что $X \sm \{x\}$ открыто. Рассмотрим $\forall y \in X \sm \{x\}$.
            Для всякого $y$ он лежит в $X \sm \{x\}$ вместе с некоторой окрестностью по отделимости от $x$.
            Значит, $\{x\}$ замкнуто.
        }{
            В качестве отделяющего множества для $x$ и $y$ можно взять $X \sm \{y\}$.
        }
    }

    \subsubsection{Вторая аксиома отделимости}
    \definition[Пространство удовлетворяет второй аксиоме отделимости, T2]{
        $\forall x, y \in X: \exists U_x \ni x, U_u \ni y: U_x \cap U_y = \o$. Разумеется, $U_x, U_y \in \Omega$.

        По-другому такие пространства называются хаусдорфовыми.
    }
    \fact{\label{Hausdorfability_check}
    Любое метрическое пространство Хаусдорфово --- для точек $x$ и $y$ можно рассмотреть шары радиусом $d(x, y) / 2$.
    }
    \theorem{
        Пространство Хаусдорфово $\iff$ \emph{диагональ} $\defset{(a, a)}{a \in X}$ замкнута в $X \times X$.
        \provetwhen{
            Пусть $\Delta$ --- диагональ. Докажем, что дополнение к $\Delta$ открыто.
            Рассмотрим $(a, b) \in (X\times X) \sm \Delta$. Из определения Хаусдорфовости $\exists U_a, U_b \in \Omega$, отделяющие $a$ и $b$.
            Но тогда $U_a \times U_b$ с одной стороны открыто, а с другой --- не пересекается с диагональю.
        }{
            Диагональ замкнута, значит, $(X\times X) \sm \Delta$ открыто.
            Рассмотрим $a, b \in X$. $(X\times X) \sm \delta$ открыто, значит, $\exists U_a \times U_b$, открытое в произведении --- элемент базы произведения, содержащий $(a, b)$.
            Получается, $U_a$ и $U_b$ отделяют $a$ и $b$.
        }
    }

    \subsubsection{Третья аксиома отделимости}
    \definition[$X$ удовлетворяет третьей аксиоме отделимости, T3]{
        $\forall \Fc$ --- замкнутого множества, и $\forall x \notin \Fc$: существуют окрестности $U_x \ni x$ и $U_\Fc \supset \Fc$, их отделяющие.
    }
    \theorem{
        Пространство удовлетворяет T3 $\iff \forall x \in X, \forall U_x \in \Omega: {\exists V_x \in \Omega}$ --- подокрестность, такая, что $\Cl V_x \subset U_x$.
        \provetwhen{
            Найдём для точки $x$ и окрестности $U_x \ni x$ подходящую $V_x$.
            Для этого применим третью аксиому отделимости для $\{x\}$ и $X \sm U_x$.

            Пусть нашлись окрестности $V_x$ и $W \supset X \sm U_x$. Таким образом, $\Cl V_x \subset X \sm W \then \Cl V_x \subset U_x$.
        }{
            Пусть $x \in X$ и $\Fc \subset X$ --- точка из замкнутого множества. Рассмотрим $U = X \sm \Fc$.
            Согласно посылке теоремы, существует $V_x \subset U: \Cl V_x \subset U$.

            Легко проверить, что T3 выполняется, можно рассмотреть $V_x \ni x$ и $X \sm \Cl V_x \supset \Fc$.
        }
    }
    \definition[Пространство регулярно]{
        Удовлетворяет T1 и T3.
    }

    \subsubsection{Четвёртая аксиома отделимости}
    \definition[$X$ удовлетворяет четвёртой аксиоме отделимости T4]{
        $\forall \Fc_1, \Fc_2$ --- замкнутые --- $\exists U_1 \supset \Fc_1$ и $U_2 \supset \Fc_2$ --- непересекающиеся открытые множества.
    }
    \definition[Пространство нормально]{
        Удовлетворяет T1 и T4.
    }
    \theorem{
        Верна цепочка следствий: \[\{\text{нормальность}  = T1 + T4\} \then \{\text{регулярность} = T1 + T3\}\then \{\text{Хаусдорфовость} = T2 \}\then T1\]
        \provehere{Оставляется, как упражнение читателю.}
    }
    \theorem{
        Метрическое пространство нормально.
        \prove{
            Проверим T4 (T2 проверено выше~(\cref{Hausdorfability_check})).

            Заметим, что расстояние от точки до замкнутого множества (не содержащего её) больше нуля: $d(x, \Fc) > 0$.
            В случае расстояния --- нуля --- точка бы принадлежала множестве из-за замкнутости.

            Пусть $\Fc_1, \Fc_2$ --- два замкнутых непересекающихся множества. В качестве $U_1$ и $U_2$ рассмотрим точки, находящиеся ближе к одному множеству, нежели к другому.
            \[U_1 = \bigdefset{x \in X}{d(x, \Fc_1) < d(x, \Fc_2)} \quad U_2 = \bigdefset{x \in X}{d(x, \Fc_2) < d(x, \Fc_1)}\]

            Эти множества открыты, так как функция расстояния 1-липшицева: всякая точка $x \in U_1$ содержится в $U_1$ вместе с шаром радиуса $\frac12(d(x, \Fc_2) - d(x, \Fc_1))$.
        }
    }

    \subsubsection{Лемма и теорема Урысона}
    \definition[Функция Урысона]{
        Пусть $A, B$ --- два замкнутых непересекающихся множества.
        Всякая функция $\phi: X \map [0; 1]$, такая. что $\phi\big|_A \equiv 0$ и $\phi\big|_B \equiv 1$.
    }
    \note{
        Такую функцию легко построить в метрическом пространстве: \[f(x) =\min\left(1, \dfrac{d(x, A)}{d(x, B)}\right)\]
    }
    \intfact[Лемма Урысона]{
        Топологическое пространство нормально $\iff$ для любых двух замкнутых непересекающихся множеств существует функция Урысона.
        \note{
            Обратно это очевидно: в качестве открытых множеств, содержащих $A$ и $B$ можно взять $\phi^{-1}([0; \nicefrac{1}{3}))$ и $\phi^{-1}((\nicefrac{2}{3}; 1])$.
        }
    }
    \intfact[Теорема Урысона о метризации]{
        Всякое нормальное пространство со счётной базой метризуемо.
    }

    \subsection{Связность}
    \definition[Топологическое пространство связно]{
        Его нельзя разбить на два непустых открытых множества.
    }
    \examples{
        \item Антидискретное пространства связно
        \item Дискретное пространство мощности хотя бы 2 не связно.
        \item $\R \sm \{0\}$ не связна.
        \item $[a, b] \cup [c, d]$ не связно ($a < b < c < d$).
    }
    \theorem{
        Следующие условия эквивалентны:
        \numbers{
            \item $X$ связно.
            \item $X$ нельзя разбить на 2 непересекающихся замкнутых множеств.
            \item $A \subset X$ одновременно и открытое, и замкнутое $\then A = \o$ или $A = X$.
            \item $\nexists f: X \map \{0, 1\}$, где $f$ --- сюръективное непрерывное отображение, а на $\{0; 1\}$ введена дискретная топология.
        }
        \provebullets{
            \item $(1) \iff (2)$. Дополнение открытого замкнуто и наоборот.
            \item $(1) \iff (3)$. $\exists A \ne \o, X$, одновременно открытое и замкнутое $\iff A \sqcup (X \sm A) = X$, где оба открыты.
            \item $(1) \iff (4)$. $\exists f \iff f^{-1}(\{0\})$ и $f^{-1}(\{1\})$ открыты.
        }
    }
    \theorem{
        Отрезок $[0; 1]$ связен в стандартной топологии.
        \prove{
            Предположим противное: $[0; 1] = U \sqcup V$, где $U, V \in \Omega_{[0; 1]}$.

            Без потери общности считаем, что $0 \in U$.
            Из открытости $\exists \eps > 0: [0; \eps) \subset U$.

            Пусть $a = \sup\bigdefset{\eps \in [0; 1]}{[0; \eps) \subset U}$.

            Если $a \in V$, то из открытости $V$ получаем, что $a$ --- не точная верхняя грань (в районе некоторой $\delta: (a - \delta; a + \delta) \subset V$ все точки в $V$, грань можно уменьшить).

            Если $a \in U$, то из открытости $U$ получаем, что $a$ --- не точная верхняя грань (есть больше). Здесь может так случиться, что $a = 1$, но в таком случае $U = [0; 1]$ и $V = \o$, опять же противоречие.
        }
    }
    \theorem{
        Для подмножества прямой $X \subset \R$ следующие условия эквивалентны:
        \numbers{
            \item $X$ связно.
            \item $X$ выпукло ($\forall a, b \in X: (a, b) \subset X$).
            \item $X$ --- интервал в обобщённом смысле ($\langle a, b \rangle$, где $a \le b$, $a \in [-\infty; +\infty), b \in (-\infty; +\infty]$).
        }
        \provebullets{
            \item $(1) \then (2)$. Предположим, что это не так: возьмём отрезок $(a, b)$ такой, что точка $x$ внутри не принадлежит отрезку.
            Тогда нашлось разбиение $X = (X \cap (-\infty; x)) \sqcup (X \cap (x; +\infty))$.
            \item $(2) \then (1)$. Предположим, что это не так: $X = U \sqcup V$.
            Но тогда возьмём две точки $a \in U$, $b \in V$, без потери общности $a < b$, тогда из выпуклости $X$: $[a, b] = (U \cap [a, b]) \sqcup (V \cap [a, b])$ --- противоречие со связностью отрезка.
            \item $(2) \then (3)$. $X = \langle \inf X, \sup X \rangle$.
            \item $(3) \then (2)$. Очевидно.
        }
    }

    \subsubsection{Связность и непрерывность}
    \theorem{
        Непрерывный образ связного пространства связен: $\forall f: X \map Y$: $f$ --- непрерывно и $X$ --- связно, значит, $Y$ связно.
        \provehere{
            Пусть $U \sqcup V = f(X)$, где $U, V \in \Omega_{f(X)}$.

            Тогда $f^{-1}(U) \sqcup f^{-1}(V) = X$, противоречие со связностью $X$.
        }
    }
    \corollary{
        Связность --- топологическое свойство.
    }
    \theorem[О промежуточном значении]{
        Пусть $f: X \map \R$ --- непрерывное отображение.

        Если $X$ связно, то $\forall a, b \in f(X): f(X) \supset [a, b]$.
        \provehere{$X$ связно $\then f(X)$ связно $\then f(X)$ выпукло.}
    }
    \definition[Компонента связности пространства $X$]{
        Связное подмножество, не содержащееся ни в каком, строго большем по включению, связном подмножестве.
    }
    \newlection{3 декабря 2022 г.}
    \lemma{
        Объединение любого семейства попарно пересекающихся связных множеств связно.
        \prove{
            Пусть данное семейство $\{A_\alpha\}_{\alpha \in \Lambda}$. $\forall \alpha \in \Lambda: A_\alpha$ связно и $\forall \alpha, \beta \in \Lambda: A_\alpha \cap A_\beta \ne \o$.

            Положим $Y = \bigcup\limits_{\alpha \in \Lambda}A_\alpha$.

            От противного: пусть $Y = U \sqcup V$, где $U, V$ --- открыты.
            Рассмотрим произвольное $A_{\alpha_0}$. Так как оно связно, то он содержится либо полностью в $U$, либо полностью в $V$. Не умаляя общности, в $U$.

            Так как $\forall \beta \in \Lambda: A_{\beta} \cap A_{\alpha_0} \ne \o$, то $\forall \beta \in \Lambda: A_\beta \subset U$.

            Но тогда из-за связности все $A_\beta \subset U$, откуда $V = \o$.
        }
    }
    \theorem{\down
    \bullets{
        \item Всякая точка лежит в некоторой компоненте связности.
        \item Причём различные компоненты связности не пересекаются.
        \provehere{
            Всякая точка $x \in X$ содержится в объединении всех связных множеств, её содержащих (эти множества есть, например, $\{x\}$ связно).
            Нетрудно видеть, что эти объединения связны, максимальны по включению и дизъюнктны.
        }
    }
    }
    \corollary{
        Компоненты связности дают разбиения топологического пространства.
    }

    \subsubsection{Свойства связности}
    \numbers{
        \item Любое связное подмножество подпространства содержится в некоторой компоненте связности.
        \item Пространство несвязно $\iff$ в нём есть хотя бы две компоненты связности.
        \item Замыкание связного множества связно.
        \provehere{
            Пусть замыкание несвязно.
            Тогда оно представимо в виде объединения двух замкнутых множеств $\Cl A = \Fc_1 \sqcup \Fc_2$.
            Так как исходное множество $A$ связно, то оно содержится полностью в одном из них, пусть в $\Fc_1$.

            Согласно свойству замыкания, $\Cl A \subset \Fc_1$, значит, $\Fc_2 = \o$.
        }
    }
    \corollary{Компоненты связности замкнуты.}
    \note{
        Число компонент связности --- топологический инвариант.
    }

    \subsection{Линейная связность}
    Пусть $X$ --- топологическое пространство.
    \definition[Путь в $X$]{
        Непрерывное отображение $\alpha: \underset{\text{станд.}}{[0; 1]} \map X$.

        $\alpha(0)$ называют \emph{началом пути}, а $\alpha(1)$ --- \emph{концом пути}.
    }
    \definition[Линейно связное топологическое пространство $X$]{
        Любые две точки $X$ можно соединить путём.
    }
    Говорят, что $A \subset X$ линейно связно, если $A$ связно в индуцированной топологии;\ это значит, что между всякой парой точек $a, b \in A$ существует путь, полностью лежащий в $A$.
    \example{
        Отрезок евклидового пространства --- путь. Таким образом, $\R^n$ линейно связно, как и его выпуклые подмножества.
    }

    \subsubsection{Линейная связность и непрерывность}
    Пусть $X, Y$ --- топологические пространства, причём $X$ --- линейно связно.
    \theorem{
        Если есть непрерывная функция $f: X \map Y$, то $f(X)$ линейно связно.
        \provehere{
            Пусть $x, y \in f(X)$. Тогда есть путь, соединяющий $a, b$ --- какие-то два прообраза $x$ и $y$ соответственно, между ними есть путь $\alpha$.

            Композиция непрерывных функций непрерывна, значит, $f \circ \alpha$ --- путь между $x$ и $y$.
        }
    }
    \lemma{
        Способность быть соединёнными путём --- отношение эквивалетности.
        \provebullets{
            \item Рефлексивность. Постоянное отображение непрерывно.
            \item Симметричность. Если $\alpha : [0; 1] \map X$ --- путь, то $\alpha \circ (1 - x)$ --- тоже путь.
            \item Транзитивность. Если $\alpha$ --- путь между $x, y$, а $\beta$ --- путь между $y, z$, то $\gamma: [0; 1] \map X$
            \[\gamma(t) = \all{\alpha(2t),&t \in \left[0; \frac{1}{2}\right] \\ \beta(2t - 1),&t \in \left[\frac{1}{2}; 1\right]}\]
            $\gamma$ непрерывна, так как $\alpha([0; 1])$ и $\beta([0; 1])$ --- фундаментальное покрытие $\gamma([0; 1])$~(\cref{fundamental_covering_def}).
        }
    }
    \definition[Компоненты линейной связности]{
        Классы эквивалентности по отношению способности быть соединёнными путём.
    }
    \note{
        Число компонент линейной связности --- топологический инвариант.
    }

    \subsection{Связность и линейная связность}
    \theorem{
        Из линейной связности следует связность.
        \provehere{
            $\forall x, y \in X: \exists \alpha$ --- путь между $x$ и $y$.
            Так как отрезок связен, то образ пути $\alpha([0; 1])$ тоже связен.

            Таким образом, $\alpha([0; 1]) \subset \mathcal{C}$, где $\mathcal{C}$ --- компонента связности точки $x$.

            Получаем, что $\forall y \in X: y \in \mathcal{C}$, откуда $\mathcal{C} = X$
        }
    }
    \corollary{
        Компоненты линейной связности содержатся в компонентах связности.
    }
    \singlepage{
        \counterexample[Связное, но не линейно связное множество]{\down
        \bullets{
            \item Рассмотрим пространство $\R^2$.

            В нём график $y = \cos\left(\frac{1}{x}\right)$ \[A \coloneqq \defset{(x, y) \in \R^2}{y = \cos\left(\frac{1}{x}\right), x > 0}\] линейно связен, значит, связен.

            Рассмотрим $X \coloneqq A \cup \{(0, 0)\}$. Так как $\Cl_{\R^2}A = A \cup \defset{(0, y)}{y \in [-1; 1]}$, то $\Cl_X A = X$, то есть $X$ связно.

            \item В то же время точка $(0, 0)$ образуют одноточечную компоненту линейной связности.
            От противного: пусть есть путь $\alpha: [0, 1] \map X$ с началом в $(0, 0)$.
            Обозначим $T = \alpha^{-1}((0, 0))$.

            \bullets{
                \item Докажем, что $T$ открыто.
                Пусть $t_0 \in T$ --- произвольный элемент.
                Рассмотрим единичный шар $B_1((0,0))$.
                Из непрерывности пути $\exists \delta > 0: \alpha(B_\delta(t_0)) \subset B_1((0, 0))$.

                Пусть $\exists t_1 \in B_\delta(t_0): \alpha(t_1) \ne (0, 0)$.
                Запишем путь покомпонентно: $\alpha(t) = (x(t), y(t))$.
                Оба отображения $x, y$ непрерывны, по теореме о промежуточном значении все значения из $(x(t_0), x(t_1))$ достигаются.

                В частности, достигается $\frac{1}{2 \pi n_*} = x(t_*)$ для достаточно большого $n_*$ и $t_*$ между $t_0$ и $t_1$.
                Тогда заключаем, что $y(t_*) = 1$ и приходим к противоречию --- $\alpha(t_*) \notin B_1((0, 0))$.

                Таким образом $\forall t_1 \in B_\delta((0, 0))$: $\alpha(t_1) = (0, 0)$ и $T$ открыто.
                \item $T$ замкнуто, как прообраз замкнутого.
                Значит, $T$ и замкнуто, и открыто, но так как это --- непустое подмножество $[0, 1]$, то $T = [0, 1]$, откуда все пути с началом в $(0, 0)$ постоянны.
            }

        }
        }
    }

    \subsubsection{Пространства, в которых всякая точка имеет некоторую линейно связную окрестность}
    \examples{
        \item Какое-то открытое евклидово подмножество $U \subset \R^n$.
        \item \up \definition[Топологическое многообразие размерности $n$]{
            Хаусдорфовое пространство $X$ со счётной базой, такое, что $\forall x \in X: \exists U_x \in \Omega_X: U_x \sim \R^n$.
        }
        Так, примером многообразия размерности $n$ является $S^n$ --- стандартная сфера в $\R^{n + 1}$.
    }
    \theorem{
        Для пространств, в которых всякая точка имеет некоторую линейно связную окрестность, линейная связность совпадает со связностью, причём компоненты связности открыты.
        \provehere{
            Пусть $W$ --- компонента линейной связности.
            Рассмотрим произвольную $a \in W$ и её линейно связную окрестность $U_a$.
            Из-за линейной связности $U_a \subset W$, значит, $W$ открыто.

            Если какая-то компонента связности состоит из некоторых компонент линейной связности, то она бьётся на некоторые открытые множества, противоречие.
        }
    }

    \subsection{Негомеоморфность}
    \theorem{
        Следующие множества попарно негомеоморфны:
        \[[0; 1]  \qquad [0; 1) \qquad \R \qquad S^1 \qquad \R^2\]
        \provebullets{
            \item В $[0; 1]$ есть две точки ($0$ и $1$), такие, что $[0; 1] \sm \{0, 1\}$ по-прежнему связно.
            \item В $[0; 1)$ есть одна точка ($0$), но нету двух, таких, что при выкидывании их вместе пространство останется связным.
            \item В $\R$ нет ни одной точки, при выкидывании которой пространство останется связным.
            \item В $S^1$ есть одна точка, при выкидывании которой пространство останется связным, причём это любая точка, а в полуинтервале --- не любая.
            \item В $\R^2$ есть как минимум 3 точки, при выкидывании которых пространство не потеряет связность --- например, $(0, 0), (0, 1), (0, 2)$.
            \note{$\R^2$ не потеряет связность при выкидывании конечного числа точек, так как оно останется линейно связным.}
        }
    }

    \subsection{Компактные пространства и множества}
    \definition[Компактное топологическое пространство]{
        Из любого открытого покрытия пространства можно выделить конечное подпокрытие.
    }
    \examples{
        \item Конечное пространство компактно.
        \item Бесконечное дискретное пространство некомпактно --- из покрытия одноточечными множествами не выделить конечное подпокрытие.
        \item Полуинтервал $(0, 1]$ некомпактен --- можно рассмотреть бесконечное покрытие $(0; 1] = \bigcup\limits_{n = 1}^{\infty}(\frac{1}{n}; 1]$.
        \item Пусть $A \subset X$.
        \note{
            Следующие условия равносильны:
            \bullets{
                \item $A$ компактно в индуцированной топологии.
                \item Для любого $\Gamma \subset \Omega_X$, такого, что $\bigcup\Gamma \supset A$ можно выделить конечное подмножество $\Gamma$ с тем же свойством.
            }
        }
        \item  \up \lemma[Лемма Гейне --- Бореля]{
            Отрезок $[0, 1]$ компактен.
            \provehere{
                От противного. Зафиксируем покрытие отрезка $[0, 1]$, из которого не выделить конечное подпокрытие.

                Положим $[a_0, b_0] \coloneqq [0, 1]$.

                Построим по индукцию систему вложенных отрезков со сколь угодно малыми длинами $[a_i, b_i]$, такую, что из покрытия $[a_i, b_i]$ не выделить конечное.
                В самом деле, если из покрытия $[a_i, b_i]$ не выделить конечное, то это же верно и либо для $\left[a_i, \frac{a_i + b_i}{2}\right]$, либо для $\left[\frac{a_i + b_i}{2}, b_i\right]$.

                Рассмотрим $\{c\} = \bigcap\limits_{i = 1}^{n}[a_i, b_i]$.
                По определению покрытия найдётся открытое множество $U_c \ni c$, значит, есть открытый шар $B_r(c) \subset U_c$.

                Значит, найдётся отрезок, лежащий внутри данного шара.
                Для него получилось неверно, что из его покрытия не выделить конечное, противоречие.
            }
        }
    }
    \theorem{
        $X$ компактно, $A \subset X$ замкнуто $\then$ $A$ компактно.
        \provehere{
            Рассмотрим произвольное открытое покрытие $A$, назовём его $\Gamma$.
            Заметим, что $\Gamma \cup (X \sm A)$ --- открытое покрытие $X$, получается, из него можно выделить конечное подпокрытие $\tilde{\Gamma}$.

            Значит, $\tilde{\Gamma} \sm (X \sm A)$ --- конечное подпокрытие $A$, так как $A \cap (X \sm A) = \o$.
        }
    }
    \theorem{
        Произведение двух компактов --- компакт.
        \provehere{
            \indentlemma{
                Заметим, что для проверки на компактность достаточно проверять только покрытия элементами из базы $\Sigma = \{V_{\beta_1}, \dots, V_{\beta_n}\}$.

            }{
                Рассмотрим произвольное покрытие $\Gamma$, $\forall U \in \Gamma: U = \bigcup\limits_{\text{какие-то }\beta}V_\beta$, где $V_\beta \in \Sigma$.

                Из покрытия всеми такими $V_\beta$ можно выделить конечное подпокрытие $\tilde{\Sigma}$.
                Тогда сопоставим всякому $V_\beta \in \tilde{\Sigma}$ одно любое $U \in \Gamma: V_\beta \subset U$.
                Это искомое конечное подпокрытие.
            }
            Рассмотрим произвольное $\Gamma$ --- покрытие $X \times Y$ множествами из базы --- они имеют вид $U_\alpha \times V_\beta$, где $U_\alpha \in \Omega_X, U_\beta \in \Omega_Y$.

            Посмотрим на произвольный $x \in X$.
            Множество $\{x\} \times Y$ компактно, так как оно гомеоморфно $Y$.
            Значит, можно выделить конечное подпокрытие, покрывающее $\{x\} \times Y$, назовём это покрытие $\{U_i^x \times V_i^x\}_{i = 1..N_x} \subset \Gamma$.

            Сопоставим всякому $x:$ $W_x = \bigcap\limits_{i = 1}^{N_x}U_i^x$.
            Это пересечение конечного числа открытых множеств, оно открыто.

            Так как $X$ компактно, то можно выбрать некоторое конечное множество $\tilde{X} \subset X$, такое, что $\bigcup\limits_{\tilde{x} \in \tilde{X}}W_{\tilde{x}} = X$.

            Получаем конечное подпокрытие $X \times Y$, оно равно \[\bigcup\limits_{\tilde{x} \in \tilde{X}}\left\{U_i^{\tilde{x}} \times V_i^{\tilde{x}}\right\}_{i = 1..N_{\tilde{x}}}\]
        }
    }
    \intfact[Теорема Тихонова]{Тихоновское произведение любого числа компактных пространств компактно.}
    \theorem{\label{vdgbhskja}
    Пусть $X$ --- хаусдорфово пространство, а $A \subset X$ --- компактно.
    Тогда $A$ замкнуто.
    \provehere{
        Докажем, что $X \sm A$ открыто.
        Рассмотрим $y \in X \sm A$.

        Согласно хаусдорфовости, $\forall a \in A: \exists U_a \ni a, U_y \ni y: U_a \cap U_y = \o$.

        Получили покрытие множества $A$ открытыми множествами; выделим из них конечное подпокрытие $\{U_{a_i}\}_{i = 1..N}$.

        Каждой такой окрестности $U_{a_i}$ соответствует своя окрестность точки $y$.
        Пересечение конечного числа открытых множеств открыто, получили, что $y$ содержится в $X \sm A$ вместе с некоторой своей окрестностью.
    }
    }
    \newlection{6 декабря 2022 г.}
    \theorem{
        $X$ --- хаусдорфово и компактно $\then X$ --- нормально.
        \provebullets{
            \item T1. Очевидно из хаусдорфовости.
            \item T4. Рассмотрим $A, B \subset X$ --- замкнутые множества.
            Они компактны, как замкнутые подмножества компакта.

            Зафиксируем $a \in A$.

            Из хаусдорфовости $\forall b \in B$ найдутся непересекающиеся окрестности $\exists V_{a,b} \ni a, U_{a,b} \ni b$, обозначим $\mathcal{U}_a \coloneqq \bigcup\limits_{b \in B}U_{a,b}$.

            $\mathcal{U}_a \supset B$, значит, можно выделить конечное подпокрытие $\tilde{\mathcal{U}}_a\coloneqq \bigcup\limits_{i = 1}^{N}U_{a,b_i} \supset B$.

            Обозначим $\mathcal{V}_a \coloneqq \bigcap\limits_{i = 1}^{N}V_{a,b_i}$.
            Заметим, что $a \in \mathcal{V}$, причём $\mathcal{V}_a \cap \mathcal{U}_a = \o$, а $\mathcal{V}_a \cap \tilde{\mathcal{U}}_a = \o$ и подавно.

            Теперь аналогично переберём все $a \in A$.
            Здесь уже из покрытия $A \subset \bigcup\limits_{a \in A}\mathcal{V}_a$ выберем конечное подпокрытие $A \subset \bigcup\limits_{j = 1}^{M}\mathcal{V}_{a_j}$.
            Соответствующее пересечение $B \subset \bigcap\limits_{j = 1}^{M}\tilde{\mathcal{U}}_{a_j}$ открыто.
        }
    }
    \theorem{\label{gvdchbsx}
    Компактное метрическое пространство $(X, d)$ ограничено.
    \provehere{
        Пусть $a \in X$ --- произвольная точка.
        Так как $\forall x \in X: d(a, x) \in \R$, то ${\bigcup\limits_{n = 1}^{\infty}B_n(a) = X}$.

        Выделив из покрытия конечное подпокрытие, найдём такое $n \in \N: B_n(a) = X$.
        Тогда, согласно неравенству треугольника, расстояние между любыми двумя точками не превышает $2n$.
    }
    }
    \corollary{
        Компактное множество в метрическом пространстве замкнуто и ограничено.
        \provehere{
            Из~(\cref{gvdchbsx}) ограничено, из~(\cref{vdgbhskja}) --- замкнуто.
        }
    }
    \theorem{
        $A \subset \R$ --- компактно $\iff$ $A$ замкнуто и ограничено.
        \provetwhen{
            См. следствие.
        }{
            Так как $A$ ограничено, то $\exists R \in \R: A \subset [-R; R]^n$. Заметим, что $[-R; R]^n$ --- компактно, как произведение компактов.
            Тогда $A$ --- замкнутое подмножество компакта, откуда $A$ --- компактно.
        }
    }

    \subsubsection{Компактность на языке замкнутых множеств:}
    \definition[Центрированный набор подмножеств $\{A_\alpha\}_{\alpha \in \Lambda}$]{
        Такой набор, что любое его конечное подмножество имеет непустое пересечение.
    }
    \example{$A_1 \supset A_2 \supset A_3 \dots$ центрирован.}
    \theorem{
        $X$ компактно $\iff$ любой центрированный набор замкнутых подмножеств $X$ имеет непустое пересечение.
        \provehere{
            \indent{\fact{
                $\{X \sm B_\alpha\}_{\alpha \in \Lambda}$ --- покрытие $\iff \bigcap\limits_{\alpha \in \Lambda} B_\alpha = \o$.
            }}
            \bullets{
                \item[$\then$.]    От противного: пусть есть центрированный набор замкнутых множеств с пустым пересечением. Тогда $\{X \sm A_\alpha\}$ --- открытое покрытие, из него можно выделить конечное подпокрытие.
                Тогда мы нашли конечное подмножество с пустым пересечением $\then$ набор не центрирован.
            }
            \item[$\when$.] От противного: пусть $U_\alpha$ --- открытое покрытие, из которого не выделить конечное подпокрытие.
            Это значит, что $\bigcap(X \sm U_{\alpha_i}) \ne \o$ для любого конечного подмножества индексов $\{\alpha_i\}$.

            Но тогда получается, что $\{X \sm U_{\alpha}\}_{\alpha \in \Lambda}$ центрирован по определению, значит, $U_\alpha$ --- не покрытие.
            Противоречие.\qedhere
        }
    }
    \corollary{
        Пусть $\{A_\alpha\}_{\alpha \in \Lambda}$ --- центрированный набор замкнутых множеств.

        Если $\exists \alpha_0 \in \Lambda$: $A_{\alpha_0}$ компактно, то $\bigcap\limits_{\alpha \in \Lambda}A_\alpha \ne \o$.
        \provehere{
            Сузим набор на $A_{\alpha_0}:$ рассмотрим $\{B_\alpha\}_{\alpha\in\Lambda} = \defset{A_\alpha \cap A_{\alpha_0}}{\alpha \in \Lambda}$.
            Получили центрированный набор замкнутых подмножеств компакта $A_{\alpha_0}$.
            Значит, пересечение непусто.
        }
    }
    \theorem{
        Непрерывный образ компакта --- компакт.
        \provehere{
            Рассмотрим непрерывное отображение $f: X \map Y$, где $X$ --- компакт.
            Докажем, что $f(X)$ --- компакт. Рассмотрим произвольное открытое покрытие $f(X) = \bigcup\limits_{\alpha \in \Lambda}U_\alpha$.
            Но $\forall \alpha \in \Lambda: U_\alpha$ открыто в $f(X)$, значит, $f^{-1}(U_\alpha)$ открыто в $X$.
            Объединение прообразов --- прообраз объединения, значит, $\{f^{-1}(U_\alpha)\}$ --- покрытие $X$.
            Из него можно выделить открытое подпокрытие.
        }
    }
    \corollary{
        Компактность --- топологическое свойство.
    }
    \theorem[Вейерштрасс]{
        Непрерывная функция $f: X \map \R$ на компакте достигает свои наибольшее и наименьшее значения.
        \provehere{
            $f(X)$ --- образ компакта --- компакт, значит, содержит свои предельные точки.
            $f(X) \ni \inf f, \sup f$.
        }
    }
    \theorem{
        Пусть $f: X \map Y$ --- непрерывная биекция, где $X$ --- компактно, а $Y$ --- хаусдорфово.
        Тогда $f$ --- гомеоморфизм.
        \provehere{
            Фактически, достаточно доказать, что $f^{-1}$ непрерывно.

            Пусть $\Fc \subset X$ --- произвольное замкнутое подмножество $X$ (откуда $\Fc$ --- компакт).

            $f(\Fc)$ --- образ компакта, компакт, в хаусдорфовом пространстве компакт замкнут $\then$ $f(\Fc)$ --- замкнут.
        }
    }
    \definition[Топологическое вложение]{
        Такое отображение $f: X \map Y$, что $f$ --- гомеоморфизм между $X$ и $f(X)$.
    }
    \counterexample[Пример инъективного неперывного отображения --- не вложения]{
        Улитка --- открытый интервал сворачивается в букву $\rho$.
        Обратное не непрерывно, так как интервал не компактен.
    }
    \corollary{
        Непрерывная инъекция $f: \underbrace{X}_{\text{компакт}} \map \underbrace{Y}_{\text{хасудорфово}}$ --- непременно вложение.
    }
    \theorem[Лемма Лебега]{
        Пусть $X$ --- компактное метрическое пространство, $\{U_\alpha\}_{\alpha \in \Lambda}$ --- открытое покрытие.

        Тогда $\exists r > 0: \forall a \in X: \exists U_{\alpha}: B_{r}(a) \subset U_{\alpha}$.
        \definition[Число Лебега]{ Такой радиус $r$. }
        \provehere{
            Сопоставим каждой точке $a \in X$ радиус $r(a)$, такой, что $\exists U_\alpha: B_{r(a)}(a) \subset U_\alpha$.

            Тогда $\{B_{\frac{1}{2}r(a)}(a)\}$ --- тоже открытое покрытие, выделим из него конечное подпокрытие $\{B_{\frac{1}{2}r(a_i)}(a_i)\}_{i = 1}^{n}$.
            Тогда числом Лебега является, например, $r \coloneqq \min\limits_{i = 1}^{n}\frac{1}{2}r(a_i)$.

            В самом деле, $\forall a \in X: \exists B_{\frac{1}{2}r(a_i)}(a_i) \ni a \then |a - a_i| < \frac12r(a_i) \then B_{r}(a) \subset U_{\alpha_a}$.
        }
    }
    \corollary[Лемма Лебега для отображений]{
        Пусть $(X, d)$ --- компактно, дано открытое покрытие $\{U_\alpha\}_{\alpha \in \Lambda}$.
        Для непрерывного отображения $f: X \map Y$ найдётся радиус $r > 0: \forall x \in X: \exists U_\alpha: f(B_{r}(x)) \subset U_\alpha$.
        \provehere{
            Рассмотреть $\{f^{-1}(U_\alpha)\}_{\alpha \in \Lambda}$.
        }
    }
    \ok
    Пусть $(X, d_x)$ и $(Y, d_y)$ --- два метрических пространства.
    \definition[Равномерно неперрывное отображение]{
        Такое отображение $f: X \map Y$, такое, что $\forall \eps > 0: \exists \delta > 0: \forall x, y \in X: d_x(x, y) < \delta \then d_y(f(x), f(y)) < \eps$.
    }
    \theorem{
        Любое непрерывное отображение $f: \underbrace{X}_{\text{компакт}} \map Y$ --- равномерно непрерывно.
        \provehere{
            Рассмотрим покрытие $\{B_{\frac12\eps}(y)\}_{y \in Y}$. В качестве $\delta$ подойдёт число Лебега для покрытия $\{f^{-1}\left(B_{\frac12\eps}\right)\}$.
        }
    }
    \ok
    \definition[Предел последовательности $\{a_i\}_{i \in \N} \subset X$]{
        Такая точка $b \in X$, что \[\forall U_b \in \Omega_X: \exists M \in \N: \forall n > M: a_n \in U_b\]
    }
    \examples{
        \item Постоянная последовательность всегда сходится к своему образу.
        \item Если последовательность сходится к пределу $b$, то любая подпоследовательность тоже сходится к $b$.
        \item В антидискретном пространстве любая последовательность сходится к любому пределу.
        \item В дискретном пространстве последовательность сходится $\iff$ стабилизируется.
    }
    \theorem{
        В хаусдорфовом пространстве всякая последовательность имеет не более одного предела.
        \provehere{От противного.}
    }
    \definition[Секвенциальное замыкание $A \subset X$]{
        Совокупность пределов последовательностей, имеющих все точки в $A$.
        Обозначается $\text{SCl} A$.
    }
    \counterexample{
        Не всегда секвенциальное замыкание --- множество предельных точек.
        Можно рассмотреть прямую с топологией не более, чем счётных дополнений:
        \[\text{SCl} (0, 1) = (0, 1)\text{, в то время как } \Cl (0, 1) = \R\]
    }
    \theorem{
        $\text{SCl} A \subset \Cl A$.
        \provehere{Предел всякой последовательности --- точка прикосновения для $A$, поэтому очевидно.}
    }
    \theorem{
        В пространстве $X$, удовлетворяющем первой аксиоме счётности, $\forall A \subset X: \text{SCl} A = \Cl A$.
        \provehere{
            Пусть $b \in \Cl A$.
            Рассмотрим счётную базу $\Sigma_b = \{V_i\}_{i \in \N}$.
            Построим убывающую счётную базу $\left\{U_i = \bigcap\limits_{j = 1}^{i}V_j\right\}_{i \in \N}$

            Построим последовательность $\{a_i\}_{i \in \N}$ так, чтобы выполнялось $a_i \in U_i \cap A$.
            Она сходится к $b$.
        }
    }


    \section{Полные метрические пространства}
    Пусть $(X, d)$ --- метрическое пространство.
    \definition[Фундаментальная последовательность]{
        $\{a_i\}_{i \in \N}$. Для любого $\eps > 0: \exists M \in \N: \forall n, m > M: d_x(a_n, a_m) < \eps$.
        Их также называют \emph{последовательность Коши} или \emph{сходящаяся в себе последовательность}.
    }
    Свойства:
    \bullets{
        \item Сходится $\then$ фундаментальна.
        \item Фундаментальна $\then$ ограничена (лежит в неком шаре).
        \item Фундаментальна, и содержит сходящуюся подпоследовательность $\then$ сходится туда же.
    }
    \definition[Полное метрическое пространство]{
        В нём всякая фундаментальная последовательность имеет предел.
    }
    \examples{
        \item $\R$ полно.
        \item $\R \sm \{0\}$ не полно.
        \item \up \theorem{$\R^n$ полно.
        \provehere{
            Рассмотрим фундаментальную последовательность $\{a_k\}_{k \in \N} = \{(a^1_k, a^2_k, \dots, a^n_k)\}_{k \in \N}$.
            По каждой координате последовательность фундаментальна, из полноты $\R$ всякая имеет предел $b^i$, значит, вся последовательность сходится к $(b^1, \dots, b^n)$.
        }
        }
    }
    \theorem{
        Замкнутое подмножество $Y$ полного пространства $X$ полно.
        \provehere{
            $a_n \underset{n \to \infty}\Map b \in X$, $b$ --- точка прикосновения для $Y$, значит, $b \in Y$.
        }
    }
    \examples{
        \item Отрезок --- замкнутое подмножество прямой.
        \item Интервал не является полным, так как не замкнут, хотя и подмножество прямой.
    }
    \precaution{
        Полнота --- не топологическое свойство, например, $(0, 1) \sim \R$.
    }
    \theorem[О вложенных шарах]{
        Метрическое пространство полно $\iff$ любая последовательность вложенных замкнутых шаров с радиусом, стремящимся к 0, обладает непустым пересечением.
        \note{
            Более общая формулировка говорит о последовательности вложенных замкнутых множеств, с диаметрами, стремящимися к 0.
            Доказательство не меняется.
        }
        \provetwhen{
            $D_{r_1}(a_1) \supset D_{r_2}(a_2) \supset$\ldots выберем в каждом шаре по точке.
            Последовательность фундаментальна, $\exists a$ --- предел.
            Покажем, что $\forall i \in \N: a \in D_{r_i}(a_i)$:

            Например, от противного: $\exists i: d(a_i, a) > r_i$, значит, для $\eps \coloneqq d(a_i, a) - r_i$, согласно неравенству треугольника, $\forall j > i: B_\eps(a) \cap D_{r_j}(a_j) = \o$.
        }{
            Используя данное свойство, построим точку, являющуюся пределом последовательности $\{a_i\}_{i \in \N}$.

            Для этого рассмотрим последовательность шаров радиусами $\nicefrac{1}{2^n}$.

            Согласно фундаментальности, для $\eps = \nicefrac{1}{2^n}$ найдётся $M_n: \forall n, m \ge M_n: d(a_n, a_m) < \eps$.

            Определим последовательность вложенных шаров $D_{\frac{1}{2^{n-1}}}(a_{M_n})$.
            Несложно проверить, что шары вложены, а точка в их пересечении является пределом некоторой подпоследовательности $\{a_i\}$.
        }
    }

    \subsection{Нигде не плотные множества}
    \definition[Нигде не плотное множество $A$]{Множество, внутренность замыкания которого пуста.}
    \definition[Внешность $A$]{
        Внутренность дополнения.
        Обозначается $\overset{\circ}A$ или $\text{Ext}(A)$.
    }
    $X = \Int A \sqcup \Fr A \sqcup \overset{\circ}A$.

    \newlection{13 декабря 2022 г.}
    \theorem{
        Следующие условия равносильны:
        \numbers{
            \item Множество $A$ нигде не плотно.
            \item $\text{Ext}(A)$ всюду плотно.
            \item $\forall U \in \Omega: \exists V \ni \Omega: V \subset U \land V \cap A = \o$.
        }
        \provebullets{
            \item $(1) \iff (2)$. $\Int \Cl A = \o \iff \forall x \in X: \forall U_x \in \Omega: U_x \cap (X \sm \Cl A) \ne \o \iff U_x \cap \text{Ext} A \ne \o \iff \text{Ext} A$ всюду плотно.
            \item $(2) \iff (3)$. $V \cap A = \o \iff V \subset \text{Ext}(A)$.
        }
    }
    \theorem[Бэр]{
        Полное метрическое пространство нельзя покрыть счётным набором нигде не плотных множеств.
        \provehere{
            От противного: пусть $\{A_i\}_{i \in \N}$ покрывают полное пространство $X$.

            Рассмотрим произвольный открытый шар $B_0$.
            Будем поддерживать инвариант: $B_n \cap \bigcup\limits_{i = 1}^{n}A_i = \o$, причём радиус шара $B_n$ меньше $\frac{1}{n}$.

            При переходе от $B_n$ к $B_{n + 1}$ заметим, что так как $B_n \cap \bigcup\limits_{i = 1}^{n}A_i = \o$, то $B_n \cap \bigcup\limits_{i = 1}^{n + 1}A_i = B_n \cap A_{n + 1}$.
            Так как $A_{n + 1}$ нигде не плотно, то найдётся внутри $B_n$ открытое множество $U$, такое, что $U \cap A_{n + 1} = \o$.
            Внутри $U$ найдётся шар достаточно маленького радиуса, положим его за $B_{n + 1}$.

            Внутри каждого шара $B_i$ возьмём замкнутый шар меньшего радиуса $D_i$, так, чтобы последовательность получилась вложенной.
            Из полноты пространства у них есть общая точка;\ эта точка не покрыта последовательностью $\{A_i\}_{i \in \N}$.
        }
    }
    \counterexample[Неполное метрическое пространство, которое можно покрыть счётным набором нигде не плотных множеств]{
        $\Q = \bigcup\limits_{r \in \Q}\{r\}$.
    }
    \corollary{
        Полное (метрическое) пространство без изолированных точек несчётно.

        ($b \in X$ --- изолированная точка пространства $X \iff \{b\}$ открыто в $X$.)
        \provehere{

            От противного: множество счётно.
            Так как внутренность замыкания $\{b\}$ пуста, то $\{b\}$ нигде не плотно.
            Отсюда множество покрывается одноточечными множествами, противоречие.
        }
    }
    \definition[Пополнение метрического пространства $X$]{
        Метрическое пространство $\overline{X}$, такое, что
        \bullets{
            \item $\overline{X}$ полное.
            \item $X \subset \overline{X}$.
            \item $X$ всюду плотно в $\overline{X}$.
        }
    }
    \intfact{
        У любого метрического пространства есть пополнение.

        План доказательства: Сказать, что последовательности Коши $u$ и $v$ эквивалентны $u \sim v$, если $\lim\limits_{n \to \infty}d(u_n, v_n) = 0$.
        Ввести $\overline{X} = \{\text{последовательности Коши}\}/\sim$.
        Доказать, что $\overline{X}$ всюду плотно, и распространить на него метрику из $X$.
    }


    \section{Секвенциальная компактность}
    \definition[$X$ секвенциально компактно]{
        Любая последовательность содержит сходящуюся подпоследовательность.
    }
    \definition[$b$ --- точка накопления для $A$]{
        $\forall U_b \in \Omega: |U_b \cap A| = \infty$.
    }
    \theorem{
        В компактном пространстве любое бесконечное множество содержит точку накопления.
        \provehere{
            От противного: всякая точка $x$ имеет окрестность $U(x)$, пересекающуюся с $A$ лишь по конечному множеству точек.
            Тогда $X = \bigcup\limits_{x \in X}U(x)$.
            Выберем конечное подпокрытие, получим противоречие с бесконечностью $A$.
        }
    }
    \theorem{\label{compactivity_then_sequential_compactivity}
    Метрическое пространство $X$ компактно $\then$ $X$ секвенциально компактно.
    \provehere{
        Рассмотрим произвольную последовательность $\{u_i\}_{i \in \N}$, выберем в ней сходящуюся подпоследовательность.

        Если множество $\defset{u_i}{i \in \N}$ конечно, то $\exists v: u_i = v$ бесконечно часто.
        Тогда выделим постоянную подпоследовательность, сходящуюся к $v$.

        Иначе $\defset{u_i}{i \in \N}$ бесконечно.
        Рассмотрим в $X$ $b$ --- точку накопления для $\defset{u_i}{i \in \N}$.

        Введём последовательность шаров $B_{\nicefrac{1}{n}}(b)$, в шаре $B_{\nicefrac{1}{n}}$ выберем $n$-ю точку для подпоследовательности.
        Так как внутри всякого шара бесконечно много точек, то процесс обречён на успех.
    }
    }


    \section{Вполне ограниченные метрические пространства}
    Пусть $X$ --- метрическое пространство.
    \definition[$\eps$-сеть]{ Такое $A \subset X$, что
        $\forall x \in X: \exists a \in A: d(x, a) < \eps$.
    }
    \definition[$X$ вполне ограничено]{
        $\forall \eps > 0$ существует конечная $\eps$-сеть.
    }
    \example[Не компактное, но вполне ограниченное пространство]{
        Интервал, например, $(0, 1)$.
    }
    \theorem{
        Если $X$ --- метрическое пространство, то следующие условия эквивалентны:
        \bullets{
            \item $X$ компактно.
            \item $X$ секвенциально компактно.
            \item $X$ полно и вполне ограничено.
        }
        \provebullets{
            \item $(1) \then (2)$.~(\cref{compactivity_then_sequential_compactivity})
            \item $(2) \then (3)$.
            \bullets{\item \underline{Вполне ограниченность.}
            От противного: $\exists \eps > 0$, такое, что нет конечной $\eps$-сети.
            Построим последовательность без сходящейся подпоследовательности.
            Пусть на очередном шаге последовательность $\{a_i\}_{i = 1..n}$.
            Это не $\eps$-сеть (так как конечна), возьмём $a_{n + 1}$ так, чтобы выполнялось $\min\limits_{i = 1..n}d(a_{n + 1}, a_i) \ge \eps$.

            Попарное расстояние между любой парой точек не меньше $\eps$.
            \item \underline{Полнота.} Во всякой фундаментальной последовательности есть подпоследовательность, сходящаяся к $v$.
            Тогда вся последовательность тоже сходится к $v$.
            }
            \item $(3) \then (1)$. От противного: пусть $\{U_\alpha\}_{\alpha \in \Lambda}$ --- покрытие, из которого не выделить конечное подпокрытие.
            Построим последовательность вложенных замкнутых множеств $\{C_i\}_{i \in \N}$.
            \bullets{
                \item Рассмотрим конечную 1-сеть $A_1$.
                Шары $\defset{D_1(a)}{a \in A_1}$ покрывают всё пространство;\ из отсутствия конечного подпокрытия найдётся шар $C_1 \coloneqq D_i$, который не покрыть конечным числом элементов из $U_\alpha$.
                \item На $n$-м шаге возьмём $\frac{1}{n}$-сеть для шара $D_{n - 1}$.
                Найдётся шар $D_j$ радиуса $\frac{1}{n}$, такой, что его не покрыть конечным числом элементов $\{U_\alpha\}$.
                Положим $C_n \coloneqq C_{n - 1} \cap D_j$.
            }
            Согласно теореме о вложенных шарах (точнее замечания к ней), пересечение $\bigcap\limits_{i = 1}^{n}C_i$ состоит из одной точки, назовём её $c$.

            $c \in U_\alpha$ для некоего $\alpha \in \Lambda$, причём лежит вместе с некоторым шаром.
            Тогда начиная с некоторого места $C_n \subset U_\alpha$, откуда противоречие с тем, что шары $C_n$ нельзя покрыть конечным числом элементов покрытия.
        }
    }
    \theorem{
        Метрическое пространство $X$ компактно $\then$ выполняется вторая аксиома счётности.
    }
    \provehere{
        \indentlemma{
            Метрическое пространство $X$ вполне ограничено $\then$ выполняется вторая аксиома счётности.
        }{
            Возьмём $A$ --- объединение по $n \in \N$ всех $\frac{1}{n}$-сетей.

            $A$ всюду плотно, так как пересекается с любым шаром --- с шаром радиуса $r$ $A$ имеет общую точку в $\ceilfrac{1}{r}$ сети.

            Отсюда $X$ сепарабельно ($A$ счётно).
        }
    }
    \theorem{
        В пространстве со второй аксиомой счётности компактность равносильна секвенциальной компактности.
        \provetwhen{
            Из компактности и первой аксиомы счётности (следует из второй) следует секвенциальная компактность.
            Доказательство аналогично частному случаю~(\cref{compactivity_then_sequential_compactivity}).
        }{
            Выделим из покрытия $\bigcup U_\alpha = X$ конечное подпокрытие.

            По теореме Линделёфа~(\cref{Lindeloef_theorem}) в пространстве с 2AC найдётся счётное подпокрытие ${\bigcup\limits_{i = 1}^{\infty}U_i = X}$.
            Положим $\Fc_i \coloneqq X \sm U_i$, обозначим $W_n = \bigcap\limits_{i = 1}^{n}\Fc_i$.

            От противного: для всякого конечного $n \in \N: W_n \ne \o$.
            Тогда выберем в последовательности множеств
            $W_1 \supset W_2 \dots$ последовательность точек $a_i \in W_i$.
            В ней есть подпоследовательность, которая сходится к некой точке $b \in \bigcap\limits_{i = 1}^{\infty}W_i$. ($b \in \text{SCl} W_i \then b \in \Cl W_i$)

            Получаем, что $\forall i \in \N: b \notin U_i$, то есть, противоречие, $U_i$ --- не покрытие.
        }
    }


    \section{Факторпространства}
    Пусть $(X, \Omega)$ --- топологическое пространство, $\sim$ --- отношение эквивалентности на $X$.
    \definition[Факторпространство $X$ по отношению $\sim$]{
        Множество $X/_\sim$, такое, что ${U \subset X/_\sim}$ открыто $\iff p^{-1}(U)$ открыто в $X$.
        Здесь $p: X \map X/_\sim$ --- (каноническая) проекция, $x \mapsto \overline{x}$.
    }
    \fact{Проекция $p$ непрерывна.}

    \subsection{Свойства}
    \bullets{
        \item Факторпространство связного пространства связно.
        \item Факторпространство линейно связного пространство линейно связно.
        \item Факторпространство сепарабельного пространства сепарабельно.
        \item Факторпространство компактного пространства компактно.
    }
    \examples[Без доказательства]{
        \item Отрезок со склеенными концами --- окружность. $[0, 1]/_{0 \sim 1} \sim S^1$.
        \item Квадрат со склеенными противоположными сторонами --- тор $T^2 = S^1 \times S^1$.
        \item Восьмиугольник, склеенный по формуле $aba^{-1}b^{-1}cdc^{-1}d^{-1}$ --- сфера с двумя ручками;\ два склеенных через дырку тора.
        \item Если склеим квадрат вот так, получим цилиндр.
        \item Если склеим квадрат по-другому, получим лист Мёбиуса.
    }

    \subsection{Частные случаи факторизации}
    \bullets{
        \item Стягивание подпространства $A \subset X$ в точку. $x \sim y \iff \any{x = y \\ x, y \in A}$.
        Так, в отрезке $[0, 1]$ при стягивании подпространства $\{0, 1\}$ в точку опять же получим окружность.
        \item В замкнутом круге $D_1(0)$ при стягивании окружности $S_1(0)$ получим $S^2$.
        \item Проективная плоскость --- отождествление диаметрально противоположных точек на окружности круга.
        \item \up
        \definition[Дизъюнктное объединение топологических пространств]{
            $U$ открыто в $X \sqcup Y$ $\iff U \cap X$ открыто в $X$ и $U \cap Y$ открыто в $Y$.
        }
        Теперь можно, например, склеить из двух отрезков $AB$ и $CD$ окружность, отождествив $A \sim C$ и $B \sim D$.
    }
    $X, Y$ --- топологические пространства.
    Пусть $A \subset X, f: A \map Y$.
    \definition[Склейка $X$ и $Y$ по отображению $f$]{
        Факторпространство $X \sqcup Y /_\sim$, где $\sim$ --- наименьшее по включению отображение эквивалентности, такое, что $x \sim f(x)$.

        Обозначается $X \sqcup_f Y$.
    }
    \properties{
        \item Фактортопология является топологией (проверить).
        \item Факторпространство связного пространства связно.
        \item Факторпространство линейно связного пространства связно.
        \item Факторпространство компактного пространства компактно.
    }
    \newlection{20 декабря 2022 г.}
    \theorem{
        $Y \map X \sqcup_f Y$ --- топологическое вложение.
        \provehere{
            Проекция $X \sqcup Y \map X \sqcup_f Y$ непрерывна по определению фактортопологии.
            Значит, сужение проекции $Y \map X \sqcup_f Y$ непрерывно.

            Обратно: рассмотрим любое открытое $U \subset Y$.
            Докажем, что его образ в $p(U)$ открыт в $X \sqcup_f Y$.
            По определению склейки по отображению $p(U) = p(U \sqcup f^{-1}(U))$.
            Это множество открыто по определению фактортопологии, и из-за непрерывности $f$.
        }
    }
    \theorem{
        Пусть $X, Y$ --- топологические пространства, $X$ склеивают по $\sim$, $f: X/\sim \map Y$.
        Утверждается, что условие непрерывности $f$ равносильно условию непрерывности $f \circ p$.
        \provetwhen{
            Композиция непрерывных непрерывна.
        }{
            Проверим по определению непрерывность $f$. Рассмотрим открытое $U \subset Y$.
            Так как $f \circ p$ непрерывна, то $p^{-1}(f^{-1}(U))$ открыто.
            Но $p$ можно <<отменить>>: $p(p^{-1}(f^{-1}(U)))$ открыто по определению фактортопологии.
        }
    }
    \theorem[О пропускании через фактор]{
        Пусть $X, Y$ --- топологические пространства, $\sim$ --- отношение эквивалентности на $X$.
        Рассмотрим $g: X \map Y$, такое, что оно уважает $\sim$, то есть $x_1 \sim x_2 \then f(x_1) = f(x_2)$.

        Утверждается, что найдётся \textbf{непрерывная} $f: X/\sim \map Y$, такое, ч то $g = f \circ p$, где $p$ --- каноническая проекция.
        \provehere{
            $f\left(\overline{x}\right) = g(x)$.
            Определение корректно, так как $g$ уважает $\sim$.

            Заметим, что $f$ непрерывна, так как $f \circ p$ непрерывна.
        }
    }
    \theorem{\label{surjective_factor}
    Пусть $f: \underset{компактно}{X} \map \underset{хаусдорфово}{Y}$ --- непрерывное сюръективное отображение.

    Рассмотрим отношение эквивалентности на $X$: $x \sim y \iff f(x) = f(y)$.

    Утверждается, что $X/\sim$ гомеоморфно $Y$.
    \provehere{
        Гомеоморфизм $f: X/\sim \map Y$ существует и непрерывен согласно предыдущей теореме.

        Он очевидно инъективен и сюръективен из-за сюръективности $f$.

        $f^{-1}$ непрерывно, так как $f(U)$ открыто для открытого $U$
    }
    }
    \newlection{16 февраля 2022 г.}


    \section{Многообразия}
    \definition[$m$-мерное многообразие]{
        Хаусдорфово топологическое пространство со счётной базой, любая точка которого имеет окрестность, гомеоморфную $\R^m$.
    }
    \counterexample{
        Две прямые, склеенные везде, кроме пары точек. Не хаусдорфово.
    }
    \intfact[Теорема об инварианте размерности]{
        Никакие два непустые открытые подмножества $U \subset \R^n$ и $V \subset \R^n$ не совпадают.
    }
    \examples{
        \item $\R^n$ и всякое его открытое подмножество.
        \item Для $n = 0: \R^0 = \{\text{pt}\}$. Многообразием является всякое счётное дискретное топологическое пространство.
        \item $S^n$ --- сфера размерности $n$. В качестве окрестности точки $x$ рассмотрим $S^n \sm \{y\}$, где $y$ --- произвольная точка сферы, $y \ne x$.
        \item $\R p^n$. Рассмотрим $S^n \subset \R^n$. Введём на сфере отношение эквивалентности: $x = -x$, где $x$ --- точки сферы.
        $\R p^n \cong S^n/_\sim$. Образы координат при проекции на <<координатную полусферу>>.

        В частности $\R p^1 \cong \R p$.
    }
    \definition[$m$-мерное многообразие с краем] {
        Хаусдорфово топологическое пространство со счётной базой, любая точка которого имеет окрестность, гомеоморфную $\R^m$ \textbf{или} $\R^m_+ \coloneqq \defset{(x_1, \dots, x_m)}{x_1 \ge 0} \subset \R^m$.
    }
    \definition[Край многообразия]{
        Множество точек, для которых не существует окрестности, гомеоморфной $\R^m$.
    }
    \intfact{
        $\R^m_+$ негомеоморфно никакому открытому подмножеству $\R^n$ для любого $n$.
    }
    \intfact{Край многообразия размерности $n$ --- многообразие размерности $n - 1$.}
    \definition[Замкнутое многообразие]{
        Компактное многообразие без края.

        \example{Например, сфера $S^n$.}
        \counterexample{Полупространство.}
    }
    \intfact{
        Всякое замкнутое связное многообразие размерности 1 гомеоморфно окружности $S^1$.
    }
    \examples[Примеры двумерных многообразий]{
        \item
    }
    \examples[Двумерные многообразия]{
        \item Лента Мёбиуса: двумерное многообразие с краем.
        Склейка прямоугольника по формуле $axay$ или треугольника --- по формуле $aax$.
        Край ленты Мёбиуса --- окружность
        \item Тор: склейка прямоугольника по формуле $aba^{-1}b^{-1}$.
        \item Бутылка Клейна: склейка прямоугольника по формуле $aba^{-1}b$.
        \item $\R p^2 \cong S^2/_\sim$ эквивалентно склейке $aa$.~(\cref{surjective_factor})
    }

    \subsection{Модельные поверхности}
    \bullets {
        \item Сфера с $n$ (открытыми) дырками.
        \item Сфера с $p$ ручками: к сфере с $p$ дырками приклеить по ручке (тор с дыркой) каждой дырке.
        \item Сфера с $q$ плёнками к сфере с $q$ дырками приклеить по плёнке (ленте Мёбиуса) каждой дырке.

        Утверждается, что если у дырки отождествить противоположные точки, то получится заклеивание дырки плёнкой.

        \item Сфера с $n$ дырками, $p$ ручками и $q$ плёнками.

        Утверждается, что если есть хотя бы одна плёнка, то $p$ ручек и $q$ плёнок гомеоморфны $2p + q$ плёнкам.
    }
    \definition[Развёртка]{
        Конечное множество непересекающихся многоугольников плоскости, у которых стороны разбиты на паре, и выбран линейный гомеоморфизм между сторонами одной пары.
        Соответствующее факторпространство называется замкнутой 2-мерная поверхность.
    }
    \fact{Замкнутая 2-мерная поверхность является замкнутым двумерным многообразием.
    \provehere{
        Компактность очевидна; проверим, что у каждой точки (внутренности многоугольника, середины ребра, вершины) есть окрестность, гомеоморфная $\R^2$.
    }
    }
    \note{Если поверхность связна, то у неё есть развёртка, состоящая из одного многоугольника.}
    \definition[Ориентируемая развёртка]{
        Всегда ребро $a$ склеивается с ребром $a^{-1}$.
    }
    \definition[Каноническая развёртка первого типа]{
        $4p$-угольник, в котором стороны склеены по правилу $a_1 b_1 a_1^{-1} b_1^{-1} a_2 b_2 a_2^{-1} b_2^{-1}\ldots$
        Для данного $p$ каноническая развёртка фиксирована. Для $p = 1$ это тор;\ для $p = 2$ поверхность развёртки называется \emph{крендель}.
        В общем случае это сфера с $p$ ручками.
    }
    \definition[Каноническая развёртка второго типа]{
        $2q$-угольник, в котором стороны склеены по правилу $a_1 a_1 a_2 a_2 \ldots$
        Для $q = 1$ поверхность развёртки --- $\R p^1$. В общем виде --- сфера с $q$ плёнками.
    }
    После хитрого склеивания видно, что бутылка Клейна имеет формулу $a^{-1}a^{-1}cc$, то есть изоморфна развёртке второго типа для $q = 2$.
    \fact{
        Развёртка первого типа для фиксированного $p$ изоморфна сфере с $p$ ручками.
        \provehere{
            Будем приклеивать ручки по индукции.
            Заметим, что склейка пятиугольника по формуле $aba^{-1}b^{-1}x$ даёт тор с дыркой.
            Ну, дальше как-нибудь приклеим.
        }
    }
    \fact{Аналогично доказываем, что развёртка второго типа изоморфна сфере с $q$ плёнками.}
    \theorem{
        \bullets{
            \item Любая связная замкнутая двумерная поверхность с ориентируемой развёрткой изоморфна сфере с $p$ ручками для некоего $p$.
            \item        Любая связная замкнутая двумерная поверхность с неориентируемой развёрткой изоморфна сфере с $q$ плёнками для некоего $q$.}
    }
    \newlection{2 марта 2022 г.}
    Пусть $\Fc$ --- замкнутое двумерное многообразие
    \definition[Топологический треугольник]{
        Пара $(T, \phi)$, где $T \subset \Fc$ --- подпространство, а $\phi: \underset{\subset \R^2}{\triangle} \map T$ --- гомеоморфизм из произвольного треугольника плоскости (треугольник берётся вместе со внутренностью) на $T$.
        Образы рёбер треугольника называются \emph{рёбрами} топологического треугольника, образы вершин --- \emph{вершинами}.
    }
    \definition[Триангуляция замкнутого двумерного многообразия]{
        Конечный набор топологических треугольников $K = \{T_i, \phi_i\}_{i = 1..n}$, такой, что выполняются условия:
        \bullets{
            \item Треугольники покрывают всё пространство: $\bigcup\limits_{i = 1}^{n}T_i = \Fc$.
            \item Пересечение любых двух треугольников --- их общее ребро, общая вершина, либо пустое.
        }
    }
    \definition[Триангулируемая поверхность]{
        Поверхность, у которой существует триангуляция.}
    \intfact{
        У любого замкнутого компактного двумерного многообразия есть триангуляция.

        Hint: доказательство использует сильный вариант теоремы Жордана: замкнутая несамопересекающаяся кривая бьёт плоскость на две компоненты, такие, что одна из них --- диск.
    }
    \proposal{
        Всякое компактное двумерное многообразие можно представить, как факторпространство некоторой развёртки.
        \provehere{
            Рассмотрим произвольную триангуляцию, это частный случай развёртки:
            треугольников --- прообразов $\phi_i$ в триангуляции --- конечное число;
            их можно расположить на одной плоскости гомеоморфизмом из многих плоскостей в одну.

            Совместим два треугольника в один, если их два ребра --- общий прообраз какого-то ребра топологического треугольника многообразия.

            Надо аккуратно проследить за тем, чтобы гомеоморфизм был линейным, и, видимо, всё получится.

            Утверждение: факторпространство объединения треугольников по условию развёртки --- исходное многообразие.

            В общем, я не понимаю, какого уровня строгости ожидать, и не очень въезжаю вообще в то, что рассказывается.
        }
    }
    \intfact{
        Пространства, задаваемые различными каноническими развёртками негомеоморфны.
        Доказательство в конце семестра будет использовать фундаментальную группу.
    }
    \theorem{
        Если развёртка ориентируемая, то она гомеоморфна поверхности, задаваемой развёрткой I типа.
        Если развёртка неориентируемая, то она гомеоморфна поверхности, задаваемой развёрткой II типа.
        \provehere{
            Приведём произвольную развёртку к каноническому виду, используя следующие операции над развёртками:
            \numbers{
                \item Подразделение многоугольника на два. Плюс один многоугольник, плюс одно правило склейки.
                \item Обратная предыдущей: склеивание.
                \item Свёртывание $aa^{-1} = \text{ничего}$ (свёртка разрешается, если в многоугольнике хотя бы 3 ребра).
            }
            Займёмся комбинаторикой.

            \numbers{
                \item Так как факторпространство развёртки --- поверхность --- связна, то можно считать, что развёртка --- один многоугольник (склеим, если их несколько).
                Теперь все правила склейки --- одно циклическое слово типа $abca^{-1}deb^{-1}d^{-1}c^{-1}\ldots$

                \item Убираем вхождения подстрок типа $aa^{-1}$. Если в какой-то момент осталась строка $aa^{-1}$, то наша поверхность --- сфера.

                \item Приводим к развёртке, в которой все вершины эквивалентны.
                Как?
                Пусть есть две неэквивалентные вершины $A \not\sim B$. Если такие нашлись, то можно считать, что они --- соседние.

                Пусть $A - a - B - b - C$, где $a, b$ --- правила склейки.
                Заметим, что $b \ne a^{-1}$, иначе бы мы свернули $B$, а ещё $b \ne a$, так как $B \not\sim A$.
                Проведём ребро $d = AC$, разрежем по нему, склеим по $b$.

                Заметим, что вершин, эквивалентных $A$ стало на одну меньше, эквивалентных $B$ --- на одну меньше, остальных количество не поменялось.

                Иначе говоря, отрезали треугольник $ABC$, и приклеили его в другое место.
                Видимо, так всегда можно сделать, хотя у меня это вызывает не очень много доверия.

                Такими действиями можно переклеиваниями все вершины сделать эквивалентными.
                \item Выделение лент Мёбиуса.
                Если где-то есть два вхождения символа $c$ одного направления, то есть слово имеет вид $c \omega_1 c \omega_2$, то разрежем по диагонали $d$, склеим, получим $d d \omega_1 (\omega_2)^{-1}$.

                Повторим этот шаг столько, сколько можно, теперь все правила склейки одного направления идут подряд.

                \item Выделение ручек. Если предыдущий шаг не привёл к канонической развёртки, то найдётся две буквы $c$ и $c^{-1}$.

                Утверждается, что найдутся ещё два символа $d, d^{-1}$, такие, что в циклическом порядке $d$ идёт между $c$ и $c^{-1}$, а $d^{-1}$ --- нет.
                Это следует из того, что все вершины эквивалентны: если бы таких $d, d^{-1}$ не нашлось бы, то между вершинами от $c$ до $c^{-1}$ и между вершинами от $c^{-1}$ до $c$ не было бы связки эквивалентности.

                Итак, слово развёртки имеет вид $c \omega_1 d \omega_2 c^{-1} \omega_3 d^{-1} \omega_4$.

                Разрежем по диагонали $a$, соединяющей соответствующие концы $c$ и $c^{-1}$ и склеим по $d$.
                Получим слово $c \omega_1 \omega_4 a \omega_3 \omega_2 c^{-1} a^{-1}$.
                Не, ну это нереально понять без картинок (я ещё наверняка везде набагал при записи слов)\ldots

                Теперь проведём диагональ $b$ между соответствующими концами $a$ и $a^{-1}$, разрежем по нему и склеим по $c$.

                Получим слово  $\omega_1 \omega_4 a^{-1} b a b^{-1} a^{-1} \omega_3 \omega_2$.
                Выделили ручку. Повторяем это тоже, пока можно.
                \item Замена ручек лентами Мёбиуса.
                Пусть есть хотя бы одна ручка и хотя бы одна плёнка. Слово имеет вид $c c \omega_1 a b a^{-1} \omega_2$.
                Разрежем по центральной диагонали $d$ (соединяющей середины ручки и плёнки), склеим по $c$, получим слово $abd (\omega_2)^{-1} b a d \omega_1$.
                Из $a, b, d$ склеим три плёнки, повторяем, пока можно.
            }
        }
    }

    \subsection{Клеточные пространства}
    <<Сейчас мы определим способ построения более страшных пространств, но всё ещё не очень плохих>>
    По-другому клеточные пространства называют CW-комплексы.
    C значит closure finiteness, W значит Weak?? Возможно, раньше определение сильной и слабой топологии было противоположным.
    \definition[Клеточное пространство размерности $0$]{
        Дискретное пространство --- любой (возможно, несчётный) набор точек, каждая из которых --- открытое множество.

        Эти точки называют (нульмерными) \emph{клетками}.
    }
    \definition[Диск размерности $k$]{
        Замкнутый шар в $\R^k$.
        Его граница $\delta D^k$ --- сфера $S^{k - 1}$.
    }
    \definition[Клеточное пространство размерности $n \in \N$]{
        Топологическое пространство, полученное из клеточного пространства размерности $n - 1$, в него вклеили множество дисков $\{D_\alpha^n\}_{\alpha \in \Lambda}$, приклеивая по их границам: по отображению $\phi_\alpha = \delta D_\alpha^n \map X^{n - 1}$, где $X^{n - 1}$ --- предыдущее клеточное пространство размерности $n - 1$.

        Внутренности вклеенных дисков называют \emph{клетками}.

        Промежуточные клеточные пространства называются $k$-\emph{мерными остовами (скелетами)}

        Дополнительным условием является то, что $\phi_\alpha(\delta D_\alpha)$ содержится в конечном числе клеток соответствующего многообразия размерности $n - 1$.
    }
    <<Не запрещается что-то плохое>>, например, всю границу диска $D^2$ вклеить в среднюю точку одного из отрезков $D^1$.

    \definition[Клеточное разбиение топологического пространства]{
        Конкретное представление топологического пространства в виде клеточного пространства.

        Так, сфера $S^2$ является клеточным пространством <<точка + ничего + приклеиваем диск по точке>> = <<точка + экватор + приклеиваем два диска по экватору>>.
    }
    \definition[Клеточное пространство размерности $\omega$]{
        Рассмотрим цепочку клеточных пространств
        \[X^0 \subset X^1 \subset \dots \subset X^n \subset X^{n + 1} \subset \dots\]
        Можно проверить, что включение --- включение подпространств в топологическом смысле, открыте множества сохраняются.

        Тогда определим предельное клеточное пространство размерности $\omega$ на множестве $\bigcup\limits_{i = 0}^{\infty}X^i$.

        Топологию на данном объединении определим следующим образом: $U$ открыто в $X \iff \forall n: U \cap X^n$ открыто в $X^n$.
    }
    Имеют место следующие два утверждения:
    \bullets{
        \item Можно показать, что определённая выше топология --- самая сильная, такая, что $\text{in}: X^n \hookrightarrow X$ --- непрерывное отображение.
        \item Можно показать, что определённая выше топология --- самая сильная, такая, что $\text{in}: X^n \hookrightarrow X$ --- вложение.
    }
    Пусть $X$ --- конечное (состоит из конечного числа клеток) клеточное пространство.
    \definition[Эйлерова характеристика клеточного пространства $X$] {
        $\chi(X) \bydef \sum\limits_{k = 0}^{n}(-1)^k |I_k|$, где $|I_k|$ --- число $k$-мерных клеток.
    }
    Используя гомологии, можно доказать, что эйлерова характеристика не зависит от разбиения пространства на клетки.
    \definition[Род двумерной поверхности]{
        Наибольшее число дизъюнктных окружностей, которые можно вырезать так, чтобы она оставалась связной.
    }

    \fact{
        Род сферы с $p$ ручками и без плёнок: $\text{род}(S_{p,0}) = p$.
        Род сферы с $q$ плёнками и без ручек: $\text{род}(S_{0,q}) = q$.

        В частности, род сферы 0, род тора --- 1.

        Вырезание дырки не меняет род.
    }
    Посчитаем эйлерову характеристику сферы с $p$ ручками.

    Рассмотрим каноническую развёртку, ей соответствует естественное клеточное разбиение из одной нульмерной клетки (общая вершина), одной двумерной (поверхность) и $2p$ одномерных (так как в развёртке $4p$ вершин и столько же рёбер, но каждая пара рёбер отождествлена).
    $\chi(S_{p,0}) = 2 - 2p$.

    Аналогично эйлерова характеристика сферы с $q$ плёнками равна $\chi(S_{0, q}) = 2 - q$.

    <<Если считать, что всё, что мы сформулировали, мы знаем, то можно получить следующую теорему>>
    \theorem{
        Двумерная компактная поверхность (возможно, с краем) однозначно задаётся тройкой параметров: число компонент края, ориентируемость (наличие хотя бы одной плёнки), эйлеровой характеристикой.
        \provehere{
            Сведение к случаю поверхности без края очевидно --- заклеить все дырки дисками.

            В зависимости от ориентируемости определяем, поверхность с ручками или плёнками, а потом эйлерова характеристика показывает, сколько их.
        }
    }
    Отсюда видно, что всякая такая поверхность имеет развёртку в виде многоугольника, у которого каждая сторона либо сама по себе, либо склеена ровно с одной другой.


    \chapter{Геометрия}
    \newlection{9 марта 2022 г.}


    \section{Евклидово пространство}
    Пусть $X$ --- векторное пространство над $\R$.
    \definition[Скалярное произведение]{
        Отображение $\langle \cdot, \cdot \rangle \map \R$ со следующими свойствами
        \numbers{
            \item Симметричное: $\forall x, y \in X: \langle x, y \rangle = \langle y, x \rangle$.
            \item Билинейное:
            \begin{gather*}
                \forall x, y, z \in X: \langle x, y + z\rangle =\langle x, y \rangle + \langle x, z \rangle
                \forall x, y \in X, \lambda \in \R: \langle x, \lambda y\rangle =\lambda\langle x, y \rangle
            \end{gather*}
            \item Положительная определённость: $\langle x, x \rangle \ge 0$. $\langle x, x \rangle = 0 \iff x = 0$.
        }
    }
    \definition[Евклидово пространство]{
        Векторное пространство с заданным на нём скалярным произведением.
    }
    \example{
        $\R^n$ со стандартным скалярным произведением.
    }
    \definition[Норма или длина вектора $x \in X$]{
        $|x| = \sqrt{\langle x, x \rangle}$.
    }
    \definition[Расстояние между $x, y \in X$] {
        $d(x, y) = |x - y|$
    }
    Свойства нормы и расстояния:
    \bullets{
        \item    $|x + y|^2 = |x|^2 + 2\langle x, y \rangle + |y|^2$.
        \item $|x| > 0$ для $x \ne 0$.
        \item $|\lambda x| = |\lambda||x|$, в частности, $|-x| = |x|$.
        \item $d(x, y) = d(x + z, y + z)$.
        \item Неравенство Коши --- Буняковского --- Шварца (далее КБШ): \[|\langle x, y \rangle| \le |x| \cdot |y|\]
        причём равенство достигается тогда и только тогда, когда $x$ и $y$ линейно зависимы.
        \provehere{
            Если $x = 0$ или $y = 0$, то доказывать нечего.
            Пусть оба не равны нулю.
            \[\forall \lambda \in \R: 0 \le |x - \lambda y| = \lambda^2 |y|^2 - 2\lambda\langle x, y \rangle + |x|^2\]
            Выбрав $\lambda = \frac{\langle x, y \rangle}{|y|^2}$ --- при нём правая часть принимает наименьшее значение --- получаем $\langle x, y \rangle^2 \le |x|^2 \cdot |y|^2$ (и равенство достигается при $|x - \lambda y| = 0$), что и требовалось доказать.
        }
        \corollary[Неравенство треугольника для нормы]{
            $|x + y| \le |x| + |y|$.
            \provehere{
                Возвести в квадрат обе части и применить КБШ.
            }
        }
        \corollary[Неравенство треугольника для расстояний]{
            $d(x, z) \le d(x, y) + d(y, z)$.
            \provehere{
                \[d(x, z) = |x - z| = |(x - y) + (y - z)| \le |x - y| + |y - z| = d(x, y)+ d(y, z)\qedhere\]
            }
        }
    }
    \definition[Угол между векторами $x, y \ne 0$] {
        $\angle(x, y) = \arccos\left(\frac{\langle x, y \rangle}{|x| \cdot |y|}\right)$
    }
    Свойства:
    \bullets{
        \item $\angle \in [0, \pi]$.
        \item Для $\lambda \ne 0: \angle (x, \lambda y) = \angle(x, y) \cdot \sgn(\lambda)$.
        \item \up \theorem[Теорема косинусов]{
            $|x - y|^2 = |x|^2 + |y|^2 - 2|x|\cdot |y|\cos\angle(x, y)$.
            \provehere{Мы так определили угол.}
        }
        \item \up \theorem[Неравенство треугольника для углов]{
            $\angle(x, z) \le \angle(x, y) + \angle(y, z)$.
            \provehere{
                Положим $\alpha = \angle(x, y), \beta = \angle(y, z)$. Если $\alpha + \beta \ge \pi$, то доказывать нечего.

                Построим на плоскости треугольник со сторонами-векторами $x', z'$, такими, что $|x'| = |x|, |z'| = |z|$ и угол между ними равен $\alpha + \beta$.
                Пусть чевиана $u'$ в треугольнике составляет угол $\alpha$ со стороной $x'$ и имеет длину $|u|$.

                Отложим вектор $u$ длины $|u|$ сонаправлено вектору $y$.

                По теореме косинусов $|x - u| = |x' - u'|$, $|u - z| = |u' - z'|$, согласно неравенству треугольника для $x$ и $z$ $|x - z| \le |x - u| + |u - z| = |x' - u'| + |u' - z'| = |x' - z'|$.

                Отсюда получаем $\cos\angle(x, z) = \frac{|x|^2 + |z|^2 - |x - z|^2}{|x| \cdot |z|} \ge \frac{|x'|^2 + |z'|^2 - |x' - z'|^2}{|x'| \cdot |z'|} = \cos\angle(x', z')$.

                Таким образом, $\angle(x, z) \le \angle(x', z') = \angle(x, y) + \angle(y, z)$.
            }
        }
        \corollary[Угловой метод на сфере]{
            Пусть $S = \bigdefset{x \in X}{|x| = 1}$.
            На сфере есть метрика $d_S(x, y) = \angle(x, y)$.
        }
        \corollary{
            $\angle(x, y) + \angle(y, z) + \angle(x, z) \le 2\pi$.
            \provehere{
                \[\angle(x, z) \le \underbrace{\angle(x, -y)}_{\pi - \angle(x, y)} + \underbrace{\angle(-y, z)}_{\pi - \angle(y, z)}\]
            }
        }
    }


    \section{Ортогональные векторы}
    Пусть $(X, \langle \cdot, \cdot \rangle)$ --- евклидово пространство.

    \definition[Вектора $x, y \in X$ ортогональны]{$\langle x, y \rangle = 0$.
    Записывают $x \perp y$.}
    Свойства ортогональности:
    \bullets{
        \item $0 \perp x$.
        \item $y \perp x_1, \dots, y \perp x_n \then y \perp (\alpha_1 x_1 + \dots + \alpha_n x_n)$.
        \item \up \theorem[Пифагор]{
            \[x \perp y \quad \then \quad |x + y|^2 = |x|^2 + |y|^2\]
        }
    }
    \definition[Ортонормированный набор векторов]{
        Множество единичных векторов $\{v_1, \dots, v_n\}$, попарно ортогональных.
    }
    Пусть $v_1, \dots, v_n$ --- ортонормированный набор векторов.

    Свойства:
    \bullets{
        \item \up \[\left\langle \sum\limits_{i = 1}^{n}\alpha_i v_i, \sum\limits_{i = 1}^{n}\beta_i v_i \right\rangle = \sum\limits_{i = 1}^{n}\alpha_i \beta_i\]
        \item Ортонормированный набор векторов линейно независим.
        \provehere{
            $\alpha_1 v_1 + \dots + \alpha_n v_n = 0 \quad \underset{\text{из предыдущего}}{\iff} \quad \alpha_1^2 + \dots + \alpha_n^2 = 0$.
        }
    }
    \theorem[Ортогонализация по Граму --- Шмидту] {
        Для любого линейно независимого набора векторов $v_1, \dots, v_n \in X$
        $\exists ! \{e_1, \dots, e_n\} \subset X$ --- ортонормированный набор векторов, такой, что
        \[\forall k = 1..n: \qquad \Lin(v_1, \dots, v_k) = \Lin(e_1, \dots, e_k) \qquad\text{и} \qquad \langle e_k, v_k \rangle > 0\]
        \provehere{Докажем и существование, и единственность по индукции.

        \underline{База:} $n = 1$, можно принять $e_1 = \frac{v_1}{|v_1|}$.
        Очевидно, других вариантов нет.

        \underline{Переход:} пусть для $\{v_1, \dots, v_{n - 1}\}$ выбран набор векторов $\{e_1, \dots, e_{n - 1}\}$ с необходимыми свойствами.

        Выберем $w_n = v_n - \sum\limits_{j = 1}^{n - 1}\langle v_n, e_j \rangle \cdot e_j$.
        Это \emph{ортогональная проекция} $v_n$ на линейное пространство $\Lin(e_1, \dots, e_{n - 1})$.

        Заметим, что $\forall 1 \le i < n: \langle w_n, e_i \rangle = \langle v_n, e_i \rangle - \sum\limits_{j = 1}^{n}\langle v_n, e_j \rangle \cdot \langle e_j, e_i \rangle = \langle v_n, e_i \rangle - \langle v_n, e_i \rangle = 0$.

        Таким образом, $e_n = \frac{w_n}{|w_n|}$ подойдёт.

        Единственность: пусть в качестве $e_k$ был выбран другой вектор, $\tilde{e_k}$.
        По условию теоремы $\tilde{e_k} \in \Lin(e_1, \dots, e_k)$.
        Но тогда есть всего два варината.
        Либо $\exists i \ne k: \langle\tilde{e_k}, e_i\rangle \ne 0$, это запрещено отро-условием..
        Либо $\tilde{e_k} = \lambda e_k$ для некоего $\lambda$ (откуда из нормированности следует $|\lambda| = 1$ и знак равен 1: определяется исходя из $\langle \tilde{e_k}, v_k \rangle$).
        }
    }
    $X$ --- конечномерное пространство.
    Тогда в $X$
    \bullets{
        \item Есть ортонормированный базис.
        \item Любой ортонормированный набор можно дополнить до базиса.
    }
    \definition[Изоморфизм евклидовых пространств $X$ и $Y$]{
        Существует изоморфизм $f$: линейное отображение $f: X \map Y$, такое, что $f: X \map Y$ --- биекция, сохраняющая скалярное произведение.
    }
    \theorem{
        Любые два евклидовых пространства одной размерности изоморфны.
        \provehere{
            Определим изоморфизм на ортонормированных базисах и продолжим по линейности.
        }
    }

    \ok
    Ниже $X$ всегда конечномерно.
    \definition[Ортонормированное дополнение $A \subset X$]{
        $A^\perp \bydef \defset{x \in X}{\langle x, a \rangle = 0, a \in A}$.
    }
    Свойства ортонормированного дополнения:
    \bullets{
        \item $A^\perp$ --- линейное пространство.
        \item $A \subset B \then A^\perp \supset B^\perp$.
        \item $A^\perp = \Lin(A)^\perp$.
    }
    \theorem{
        Пусть $V \subset X$ --- линейное подпространство.
        Тогда $X = V \oplus V^\perp$.
    }
    \theorem{
        Пусть $V \subset X$ --- линейное подпространство.
        Тогда верны следующие условия:
        \numbers{
            \item $X = V \oplus V^\perp$.
            \item $(V^\perp)^\perp = V$.
        }
        \provehere{
            Выберем $\{e_1, \dots, e_k\}$ --- ортонормированный базис $V$.

            Дополним его до $\{e_1, \dots, e_k, e_{k + 1}, \dots, e_n\}$ --- ортонормированного базиса $X$.

            Проверим, что $\Lin(e_{k + 1}, \dots, e_n) = V^\perp$. Здесь верно включение в обе стороны.
        }
    }
    \newlection{16 марта 2023 г.}
    \properties[Ортогональное подпространство]{
        \item Можно определить ортогональную проекцию $\Pr_V: X \map V$ --- ведь раз $V \oplus V^\perp = X$, то всякий вектор раскладывается в сумму элементов $V$ и $V^\perp$.
        $\Pr_V$ --- по определению тот вектор из прямой суммы, который лежит в $V$.
        \item $\Pr_V$ непрерывна.
        \item $\forall x \in X: \Pr_V(x)$ --- ближайшая к $x$ точка в $V$.
        Доказательство --- применение теоремы Пифагора.
        \item Пусть $H \le X$ --- подпространство размерности $\div X - 1$, то есть гиперплоскость.
        \definition[Нормаль к гиперплоскости $H$]{
            Вектор, перпендикулярный всем векторам гиперплоскости $H$.
        }
        Нормаль существует и единственна с точностью до домножения на скаляр: это вектор, порождающий $H^\perp$.
        $\langle n \rangle = H^\perp$.
    }
    \lemma[Конечномерная лемма Рисса]{
        Пусть $L: X \map \R$ --- линейное отображение, где $X$ --- евклидово пространство.
        $\exists ! v \in V: L(u) \equiv \langle u, v \rangle$.
        \provehere{
            Выберем базис $(e_1, \dots, e_n) \subset X$.
            Тогда линейное отображение однозначно задаётся вот так:
            \gather{
                L(u \coloneqq e_1 u_1 + \dots + e_nu_n) = L(e_1) u_1 + \dots + L(e_n) u_n \\
                L(u) = \left\langle \vect{L(e_1) & \dots & L(e_n)}, u\right\rangle
            }
        }
    }
    \fact{
        Для любого линейное отображение $L: X \map \R$, не равное нулю $\not\equiv 0$: $\Ker(L)$ --- гиперплоскость.
        Любая гиперплоскость --- ядро некой скалярной функции.
        \provebullets{
            \item Теорема о размерности ядра и образа.
            \item Для $v$ --- нормали к гиперплоскости --- $L(x) \equiv \langle v, x \rangle$ подойдёт.
        }
    }
    \fact{
        Расстояние от точки $x \in X$ до гиперплоскости $H \le X$ равно $\frac{\langle x, v \rangle}{|v|} \equiv \left\langle x, \frac{v}{|v|} \right\rangle$, где $\langle v \rangle = H^\perp$.

        В самом деле, $x$ раскладывается в сумму $x = x^\perp + x^{\|}$, а $\d(x, H) = \left\langle x^\perp + x^{\|}, \frac{v}{|v|} \right\rangle = |x^\perp|$.
    }


    \section{Ортогональные преобразования}
    Пусть $X, Y$ --- евклидовые пространства (не обязательно одной размерности).
    \definition[Изометричное отображение]{
        Такое линейное отображение $f: X \map Y$, что $\langle x_1, x_2 \rangle = \langle f(x_1), f(x_2) \rangle$.
        В случае равенства пространств $X = Y$ $f$ называется \emph{ортогональным преобразованием} $X$.
    }
    \properties[Изометричные преобразования]{
        \item Для всякого линейного отображения $f$: изометричность равносильна тому, что $f$ сохраняет длины векторов.
        \item Изометричные преобразования инъективны (если $f(x) = f(y)$, то $\|f(x) - f(y)\| = 0$, то есть $\|x - y\| = 0$).
    }
    Группа ортогональных преобразований для пространства $\R^n$ называется $\bigO(n)$.
    \provehere{
        \indent{
            \problem{
                Как выглядят ортогональные преобразования в $\R^2$?
            }{
                Посмотрим, куда перешёл один из ортогональных векторов: матрица перехода имеет вид $\vect{\cos \alpha & * \\ \sin \alpha & *}$.
                Второй столбец должен быть нормирован и ортогонален первому, поэтому матрица перехода имеет вид $\vect{\cos \alpha & \sin \alpha \\ -\sin \alpha & \cos \alpha}$ (поворот на угол $\alpha$), либо $\vect{\cos \alpha & \sin \alpha \\ \sin \alpha & -\cos \alpha}$ (какие-то поворот и отражение; главное, что раскладывается в прямую сумму $\id$ и $-\id$).
            }
        }
        Положим $X_+ \coloneqq \defset{x \in X}{f(x) = x}$, $X_- \coloneqq \defset{x \in X}{f(x) = -x}$.
        Очевидно, $X_+ \cap X_- = \{0\}$.

        Найдём в $(X_+ \oplus X_-)^\perp$ плоскость поворота;\ тогда по индукции всё получится.

        Формальнее, положим изначально $V = X_+ \oplus X_-$
        Рассмотрим единичную сферу $S$ в подпространстве $V^\perp$, проверим, что происходит при отображении $f$ с точками сферы.

        $f$ ортогонально, поэтому $f(S) = S$, откуда для всякой точки $x_0 \in S$ можно рассмотреть $\angle(x_0, f(x_0))$.
        Это функция от $x_0$, она достигает минимума на компактной сфере.
        Пусть $z_0 \in S$ --- точка минимума $\angle(x_0, f(x_0))$.

        Так как $S \cap X_+ = \{0\}$, то $\angle(z_0, f(z_0)) > 0$.
        Проверим, что вектора $z_0$ и $f(z_0)$ действительно образуют плоскость, которая $f$-инвариантна.
        Это достаточно проверить на базовых векторах $z_0$ и $f(z_0)$.

        От противного: $f(f(z_0)) \notin \langle z_0, f(z_0) \rangle$.
        Рассмотрим середины сторон $z_0 - f(z_0)$ и $f(z_0) - f(f(z_0))$, если $f(f(z_0))$ не лежит в плоскости $\langle z_0, f(z_0) \rangle$, то угол между серединами строго меньше угла $\angle(z_0, f(z_0)) = \angle(f(z_0), f(f(z_0)))$.

        Таким образом, плоскость $f$ инвариантна, она не бьётся на рямую сумму $\id$ и $-\id$, значит, в ней поворот, её можно прямо приплюсовать к $V$ и продолжить по индукции.
    }

    \subsection{Ориентация векторного пространства}
    \definition[Два базиса одинаково ориентированы]{
        Матрица перехода между ними имеет положительный определитель.
    }
    \theorem{
        Одинаковая ориентируемость базисов --- отношение эквивалентности на множестве базисов данного пространства.
        \provehere{
            Детерминант мультипликативен.
        }
    }
    \definition[Ориентированное векторное пространство]{
        Векторное пространство, в котором один выделен один из классов эквивалентности ориентации базисов.
    }
    В таком случае базисы из данного класса эквивалентности называются положительно ориентированными, остальные --- отрицательно ориентированными.

    В пространстве $\R^n$ стандартная ориентация базиса совпадает с ориентацией стандартного базиса $\vect{1 & 0 & \dots & 0}, \dots, \vect{0 & 0 & \dots & 1}$
    \definition[Смешанное произведение]{
        Пусть $(X, \langle \cdot, \cdot \rangle)$ --- ориентируемое векторное пространство размерности $n$. Рассмотрим вектора $v_1, \dots, v_n \in X$.

        Смешанное произведение $[v_1, v_2, \dots, v_n] \bydef \det A$, где $A$ --- матрица разложения векторов $v_1, \dots, v_n$ по произвольному ортонормированному базису.
    }
    Так как определитель матрицы перехода между двумя ортонормированными базисами равен 1, то определение корректно.
    Из свойств определителя сразу получаем следующее:
    \properties[Смешанное произведение]{
        \item Линейность по каждому аргументу.
        \item Кососимметричность (транспозиция меняет знак).
        \item Равенство нулю эквивалентно линейной зависимости.
        \item $[v_1, \dots, v_n] > 0 \iff (v_1, \dots, v_n)$ --- положительный базис.
    }
    \definition[Векторное произведение]{
        Пусть $(X, \langle \cdot, \cdot \rangle)$ --- \textbf{трёхмерное} ориентируемое векторное пространство. Рассмотрим вектора $u, v \in X$.

        Их векторное произведение $u \times v$ --- такой вектор $h \in X$, что $\forall x \in X: \langle h, x \rangle = [u, v, x]$.

        Существование и единственность такого $h$ следует из леммы Рисса.
    }

    \properties[Векторное произведение]{
        \item По определению $\langle u \times v, w \rangle = [u, v, w]$.
        \item Из кососимметричности смешанного произведения $u \times v = -v \times u$.
        \item Билинейность.
        \item $u \times v = 0 \iff u$ и $v$ линейно зависимы.
        \provehere{
            $u$ и $v$ линейно зависимы $\iff [u, v, x] = 0$ всегда.
        }
        \item Для положительного ортонормированного базиса $(e_1, e_2, e_3)$:
        \gather{
            e_1 \times e_2 = e_3 \\
            e_2 \times e_3 = e_1 \\
            e_3 \times e_1 = e_2
        }
    }
    \theorem[Геометрический смысл векторного произведения]{\down
    \bullets{
        \item Векторное произведение $w \coloneqq u \times v$ ортогонально каждому из векторов $u, v$.
        \item $(u, v, w)$ образуют положительный базис.
        \item $|w|$ --- площадь параллелограмма, натянутого на $u$ и $v$.
    }
    \provebullets{
        \item По определению $\langle u \times v, v \rangle = [u, v, v] = 0$.
        \item Применим ортогонализацию Грама --- Шмидта для $u, v$, получим вектора $e_1 = a \cdot u, e_2 = b \cdot u + c \cdot v$, где $a, c > 0$.
        Введём $e_3$ так, что $(e_1, e_2, e_3)$ --- положительный ортонормированный базис.
        По определению $u \times v = a c e_3$, откуда
    }
    }
    \newlection{23 марта 2023 г.}

    \subsection{Формула в координатах}
    Пусть $(e_1, e_2, e_3)$ --- положительный ортонормированный базис, разложим по базису $x = x_1 e_1 + x_2 e_2 + x_3 e_3$ и $y = y_1 e_1 + y_2 e_2 + y_3 e_3$.
    Тогда \[x \times y = \abs{\arr{c c}{x_2 & x_3 \\ y_2 & y_3}}e_1 - \abs{\arr{c c}{x_1 & x_3 \\ y_1 & y_3}}e_2 + \abs{\arr{c c}{x_1 & x_2 \\ y_1 & y_2}} e_3\]
    Иногда формально пишут \[x \times y = \abs{\arr{c c c}{x_1 & x_2 & x_3 \\ y_1 & y_2 & y_3 \\ e_1 & e_2 & e_3}}\]


    \section{Матрицы Грама}
    Пусть $(V, \langle \cdot, \cdot \rangle)$ --- евклидово пространство.

    Матрица Грама $G(v_1, \dots, v_k)$ --- это матрица $\left(\langle v_i, v_j \rangle\right)_{i, j = 1}^{n}$.
    \properties{
        \item $x_i x_j g_{i,j}$ --- это что?
        \item $\det G = [v_1, \dots, v_k]$ или что-то типа
        \item $\det G = 0 \iff v_1, \dots, v_k$ линейно зависимы.
    }


    \chapter{Аффинные пространства}
    \definition[Аффинное пространство]{
        Тройка $(X, \overrightarrow{Х}, +)$, где $X$ --- непустое множество, $\overrightarrow{X}$ --- векторное пространство (его называют \emph{ассоциированное или присоединённое}), а операция \emph{откладывания вектора} $+: X \times \overrightarrow{X} \map X$, удовлетворяет свойствам:
        \bullets{
            \item $\forall x, y \in X: \exists ! u \in \overrightarrow{X}: y = x + u$. Такой $u$ обозначают $\overrightarrow{xy}$.
            \item Выполнена следующая ассоциативность: $\forall x \in X, u, v \in \overrightarrow{X}: (x + u) + v = x + (u + v)$
        }
    }
    \example[Основной, и в некотором смысле единственный]{
        Пусть $X$ --- векторное пространство.
        Выберем $\dir{X} = X$, операция сложения наследуется из $X$.
    }
    На самом деле всё сводится к этому примеру, в дальнейшем будем аффинные пространства $(X, \dir{X}, +)$ обозначать $X$.
    \properties{
        \item $x + \dir{xy} = y$ по определению.
        \item Правило треугольника: $\dir{xy} + \dir{yz} = \dir{xz}$.
        \item $\dir{xx} = \dir{0}$: в самом деле, $\dir{xx} + \dir{xx} = \dir{xx}$.
        \item $x + \dir{0} = x$: в самом деле, $x + \dir{xx} = x$.
        \item $\dir{yx} = -\dir{xy}$.
        \item Если так получилось, что $x + \dir{u} = y + \dir{u}$, то $(x + \dir{u}) - \dir{u} = (y + \dir{u}) - \dir{u} \then x = y$.
        \item Если так получилось, что $\dir{xy} = \dir{0}$, то $y = x + \dir{xy} = x + \dir{0} = x$.
    }
    Рассмотрим аффинное пространство $(X, \dir{X}, +)$, выберем произвольный элемент $O \in X$ --- \emph{начало отсчёта}.
    Утверждается, что начало отсчёта создаёт биекцию между $X$ и $\dir{X}$.
    \[\phi_O: X \leftrightarrow \dir{X} \qquad x \leftrightarrow \dir{Ox}\]
    Проверка инъективности и сюръективности остаются, как упражнение читателю.

    Отображение $\phi_O: X \map \dir{X}$ называется \emph{векторизацией} аффинного пространства $X$.

    \definition[Линейная комбинация относительно начала отсчёта $O$]{
        Для коэффициентов $t_i \in \R, \dir{p_i} \in X$ --- вектор $\dir{v} = \sum\limits_{i}t_i \dir{p_i}$ или точка $O + \dir{v}$.
    }
    \bullets{\item Барицентрические (аффинные) линейные комбинации --- такие комбинации, что $\sum\limits_{i}t_i = 1$.
    \item Сбалансированные --- такие комбинации, что $\sum\limits_{i}t_i = 1$.
    }
    \theorem{
        Барицентрическая линейная комбинация точек --- точка, не зависящая от начала отсчёта.

        Сбалансированная линейная комбинация векторов --- вектор, не зависящий от начала отсчёта.
        \provehere{
            Запишем две барицентрические координаты с началами отсчёта в $O$ и в $O'$:
            \gather{\dir{v} = \sum\limits_{i}t_i \cdot \dir{O p_i}; \qquad \dir{v'} = \sum\limits_{i}t_i \cdot \dir{O' p_i} = \sum\limits_{i}t_i \cdot \left(\dir{O'O} + \dir{Op_i}\right) = \underbrace{\left(\sum\limits_{i}t_i\right)\dir{O'O} }_{\text{0 для сбалансированной}}+ \sum\limits_{i}t_i \dir{O p_i}\\
            O + \dir{v} = O' + \dir{v'} \underset{\text{для барицентрической}}= O' + \dir{O'O} + \dir{v} = O + \dir{v}
            }
        }
    }
    Пусть $X$ --- аффинное пространство.
    \definition[$Y \subset X$ --- аффинное подпространство]{
        $\exists V \le \dir{X}, p \in Y: Y = p + V$.
        Подпространство $V$ называется \emph{направление} $Y$.
    }
    \properties{
        \item Если $Y = p + V$ --- аффинное подпространство $X$, то $\forall q \in Y: Y = q + V$.
        \item $Y$ --- аффинное пространство с ассоциированным $V$.
        \item $\forall q \in Y$: для отображения векторизации $\phi_q: \phi_q(Y) = V$.
    }
    \definition[Размерность ассоциированного пространства]{
        Размерность соответствующего ассоциированного векторного пространства. $\dim X \bydef \dim \dir{X}$.
    }
    \definition[Параллельный перенос на вектор $v \in \dir{X}$]{
        Отображение $T_{\dir{v}}: X \map X; \quad x \mapsto z + \dir{v}$.
    }
    \properties{
        \item $T_{\dir{v} + \dir{u}} = T_{\dir{v}} + T_{\dir{u}}$.
        \item $T_{\dir{0}} = \id$.
        \item $T_{\left(-\dir{v}\right)} = \left(T_{\dir{v}}\right)^{-1}$
    }
    \corollary{Параллельные переносы --- подгруппа группы биекций множества $X$.}
    \definition[Аффинные подпространства параллельны]{
        Их направления совпадают.
    }
    \definition[Прямая]{Аффинное подпространство размерности 1.}
    \definition[Гиперплоскость в конечномерном пространстве $X$]{
        Аффинное подпространство размерности $\dim X - 1$.
    }
    \theorem{
        Пересечение любого множества аффинных подпространств --- либо пустое множество, либо аффинное подпространство.
        \provehere{
            Обозначим пересекаемые подпространства за $(Y_i, \dir{V_i}, +)$.

            Пусть пересечение непусто, рассмотрим $p \in \bigcap\limits_{i}Y_i$.
            Всякое подпространство $Y_i$ имеет вид $Y_i = p + \dir{V_i}$.

            Пересечение имеет вид $p + \bigcap\limits_{i}\dir{V_i}$.
        }
    }
    \definition[Аффинная оболочка точек $A \subset X$]{
        Пересечение всех аффинных подпространств, содержащих $A$.
        Иначе говоря, наименьшее аффинное подпространство, содержащее $A$.
        Обозначается $\Aff(A)$.
    }
    Обозначим за $B_p(A)$ образ $A$ при векторизации с началом отсчёта в произвольной точке $p \in A$:
    \[B_p(A) \coloneqq \phi_p(A) = \defset{\dir{pa}}{a \in A}\]
    \proposal{ $A \subset Y$ для некоего аффинного подпространства ${Y = p + V \iff B_p(A) \subset V}$. }
    \proposal{$\phi_p(\Aff A) = \Lin(B_p(A))$.}
    \note{Мне откровенно лень писать доказательства здесь.}
    \theorem{
        Аффинная оболочка множества $A$ совпадает с множеством барицентрических комбинаций точек множества $A$.
        \provehere{
            //todo
        }
    }
    \definition[Множество точек $\{p_1, \dots, p_n\} \subset X$ аффинно независимо]{
        Существует нетривиальная сбалансированная комбинация: $\underbrace{\sum\limits_{i}t_i}_{\text{не все 0}} = 0$, причём $\sum\limits_{i}t_i \dir{p_i} = \dir{0}$.
        Ранее было показано, что начало отсчёта можно выбрать произвольно.
    }
    \theorem{
        Для множества точек $A \subset X$ следующие условия равносильны:
        \bullets{
            \item Аффинно независимы
            \item Векторы $p_1 p_k$ независимы
            \item $\dim Aff = n - 1$.
            \item Всякая точка из $\Aff$ представима в барицентрическом виде единственным образом.
        }
    }
    \newlection{30 марта 2023 г.}
    \provebullets{
    \item[$1 \iff 2$] $\sum\limits_{i} t_i p_i = 0 \iff \sum\limits_{i} t_i\dir{p_1 p_i}$
    \item[$2 \iff 3$] \ldots
    \item [$1 \then 4$] От противного: две барицентрические комбинации быть не могут, ноль --- тоже по предыдущей теореме.
    \item[$4 \then 1$] Пусть аффинно зависимы, найдём два барицентрических представления какой-то точки.
    }
    Рассмотрим аффинное пространство $(X, \dir{X}, +)$ размерности $n$.
    \definition[Аффинный точечный базис]{
        Линейно независимое множество $\{p_1, \dots, p_{n + 1}\} \in X$
    }
    \definition[Аффинный базис]{
        Фиксированный нуль $O \in X$, линейно независимое множество векторов $e_1, \dots, e_n \in \dir{X}$.
    }
    В аффинном точечном базисе любая точка представима единственным образом, как барицентрическая комбинация базиса.
    Коэффициенты в разложении точки по этому базису называют \emph{барицентрическими координатами}.

    Если же рассматривается разложение по аффинному базису, то коэффициенты --- \emph{аффинные координаты}.
    \section{Аффинные отображения}
    $(X, \dir{X}, +), (Y, \dir{Y}, +)$ --- аффинные пространства.

    $\forall \Fc: X \map Y$ определим соответствующее отображение $\tilde{\Fc}_p: \dir{X} \map \dir{Y}; \quad \tilde{\Fc}_p(\dir{v}) = \dir{\Fc(p)\Fc(p + \dir{v})}$.
    \definition[Аффинное отображение $\Fc: X \map Y$]{
    Для некоторой точки $p \in X$: отображение $\tilde{\Fc}_p$ линейно.
    }
    \lemma{
        Если для некоторой точки $p: \tilde{\Fc}_p$ аддитивно, то $\forall q \in X: \tilde{\Fc}_q\equiv \tilde{\Fc}_p$.
    }
    \note{
    Получили переформулировку: $\Fc$ аффинное, если $\exists L: \dir{X} \map \dir{Y}$, такое, что $\dir{\Fc(p)\Fc(q)} = L(\dir{pq})$.
    }
    \fact{
    Для фиксированных точек $x \in X, y \in Y$ и линейного отображения $L: \dir{X} \map \dir{Y}$ существует и линейное аффинное отображение $\Fc:X \map Y$, такое, что $\Fc(x) = y, \tilde{\Fc} = L$.
    }
    \definition[Коллинеарные точки]{
    Точки, лежащие на одной прямой; аффинное зависимые точки.
    }
    \theorem{
    Пусть $X, Y$ --- аффинные пространства, $\Fc$ --- инъективное отображение, переводящее прямые $l \subset X$ в прямые $\Fc(l) \subset Y$.
    \provehere{
    Назовём отображение $\R^2 \map \R^2$ хорошим, если это биекция, переводящая прямые в прямые.
    \lemma{
    Хорошее $f$ переводит неколлинеарные точки в неколлинеарные.
    \provehere{
    От противного: три неколлинеарные точки $A, B, C$ перешли в прямую $l$. Тогда $AB$, $BC$, $AC$ как прямые, тоже перешли в $l$.

    Дальше любая прямая плоскости пересекает хотя бы 2 из трёх прямых среди $AB, BC, AC$, значит, вся плоскость перешла в $l$.
        Противоречие с биективностью.
    }
    }
    \lemma{
    Хорошее отображение переводит параллельные прямые в параллельные прямые.
    \provehere{
    Параллельные $\equiv$ непересекающиеся.
    }
    }
    }
    }
\end{document}