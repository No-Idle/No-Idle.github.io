\documentclass[a4paper]{article}

\usepackage{../mathstemplate}

\date{IV семестр, весна 2024 г.}
\title{Кластерные алгебры и кластерные категории.\\ Неофициальный конспект}
\author{Лектор: Михаил Александрович Антипов\\ Конспектировал Леонид Данилевич}

\begin{document}
    \shorthandoff{"}
    \maketitle
    \tableofcontents
    \newpage
    \setcounter{lection}{0}
    \newlection{11 февраля 2026 г.}
    Вспомним грассманиан $\Gr(k, n)$ --- многообразие, параметризующее $k$-мерные подпространства в $n$-мерии.
    \newlection{18 февраля 2026 г.}

\definition[Колчан]{Произвольный ориентированный граф, в котором всё разрешено: петли, кратные рёбра, может быть даже бесконечное число вершин или рёбер...}
\definition[Кластерный колчан]{Конечный колчан без петель и рёбер туда-обратно (без пары рёбер вида $i \to j$ и $j \to i$.}


Пусть $Q$ --- конечный кластерный колчан, $1 \le i \le n$ --- незамороженная вершина.
\definition[Мутация колчана Q в незамороженной вершине i]{На колчанном языке это новый колчан $M_i(Q)$, в котором множество вершин то же самое, а множество рёбер претерпевает следующие изменения:
\numbers{
\item Для каждой пары вершин $k, \ell$, таких, что есть рёбра $k \to i \to \ell$, добавляем ребро $k \to \ell$ (если рёбер $k \to i$ $n_k$ штук, а рёбер $i \to \ell$ $n_\ell$ штук, то мы добавим $n_k \cdot n_\ell$ рёбер.
\item разворачиваем стрелки, инцидентные $i$
\item стираем всевозможные противонаправленные пары.
} 
}
Отметим, что это инволлюция: две мутации в одной и той же вершине не меняют колчан.
\example{Пусть $v$ --- исток. Тогда мутация в вершине $v$ --- разворот рёбер, инцидентных $v$.}
// времени мноо рпошло сегодня... Ладно, может хотя бы определение кластерной алгберы дадим сегодня. (18:14)
\exercise{Если $Q$ --- дерево на $n$ вершинах, то мутациями в источниках и стоках можно получить любую ориентацию всех $n - 1$ рёбер.}
// Даже среди алгебраистов известны факт: ктегории модулей почти эквивалентны в некотором смысле.
Но если делать мутации не в источниках и стоках, то даже какой-нибудь путь может преерпевать очень значительные изменения:
\intfact{Пусть $Q_1$ и $Q_2$ --- два кластерных колчана без ориентированных циклов на одном множестве вершин. Предположим, что они эквивалентны: существует последовательность мутаций, превращающих один в другой. Тогда $Q_2$ получается из $Q_1$ только при помощи мутаций в источниках и стоках. В частности, они изоморфны как неориентированные графы.
\prove{ Насколько известно лектору, комбинаторное доказательство этой теоремы неизвестно, а вот с помощью кластерых категорий доказательство известно уже давно.
} }
На матричном языке мутация выглядит так: из матрицы $B_Q = (b_{i,j})$ получается матрица $B_{M_i(Q)} \eqqcolon M_i(B_Q) = (b`_{i,j})$:
$b`_{pq} = \all{-b_{pq}, i = p or i = q \\ b_{pq} + b_{pi}b_{iq} product > 0 \\ b_{pq} - b_{pi}b_{iq} product < 0 \\ b_{pq} otherwise}

\definition[Матрица $B` \in M_{n}(\mathbb{Z})$ кососимметризуема]{Существуют $D = \diag{d_1, \ldots, d_n$ при $d_i > 0$ и кососимметричная $B \in M_n(\mathbb{Z})$: $B` D = D B$. }
Говоря русским языком, кососимметричная матрица с точностю до перемасштабирования строк (что?..).
\definition[Кластерная матрица]{$\tilde{B} \in M_{m,n}(\mathbb{Z})$, такая что верхний квадрат $n \times n$ --- кососимметризуемая матрица, а нижний прямоугольник --- любая.}
Мутацию в вершине $i$ определим по матричной формуле\eqref{}...
\statement{Мутация по-прежнему инволюция; кососимметризуемость верхнего квадрата сохраняется после мутации; $M_i$ коммутирует с транспониованием $B \mapsto B^t$ и разворотом всех рёбер $B \mapsto -B$. Если $b_{ij} = b_{ji}=0$, то мутации $M_i$ и $M_j$ коммутируют при действии на даную кластерную матрицу. \prove{Не совсем очевидно, но проверяется в лоб.}}
Пусть $K$ --- поле, $\tilde{B} \in M_{m\times n}(\mathbb{Z})$ --- кластерная матрица. Обозначим через $T_n$ регулярное дерево степени $n$ с неориентированными рёбрами, покрашенными числами $1, \ldots, n$. Иными словами, граф Кэли для $C_2 \star \ldots \star C_2$. (n times). Пусть $t_0 \in V(T_n)$ --- отмеченная вершина.
Кластерная алгебра геометрического типа.
\definition[Кластерная алгебра $A(B)$]{
\bullets{
\item Отображение $V(T_n) \to M_{m\times n}(\mathbb{Z})$ (на самом деле в кластерные матрицы), такое что для любых вершин $t_1$ и $t_2$, соединённых ребром цвета $i$, выполнено соотношение $B(t_2) = M_i(B(t_1))$. 
\item Отображение $x: V(T_n) \to K(x_1, \ldots, x_{m})^n$. Вектор $x(t) = \vect{x_1^t, \ldots, x_n^t}$ называется \emph{кластером} в вершине $t$. При этом $x(0) = (x_1, \ldots, x_n)$. При этом $y_j = y_j`$ при $j \ne i$, а $y_i y_i` = \prod{b_{ji}>0}y_j^{b_{ji}} + prod_{b_{ji}<0}y_j^{-b_{ji}}  B(t) = b_{ij}, ещё есть t`
}
}
А ещё есть расширенный кластер --- это просто дописали $x_{m+1}, \ldots, x_n$ в конце (замороженные переменные)
Кластерные переменные --- элементы кластера.
Кластерная алгебра --- это на самом деле неважно
Кластерная алгебра --- это подалгебра в $K(x_1, \ldots, x_m)$, порождённая кластерным переменными.
Оказывается, однородное кольцо грассманина и других интересных многообразий --- кластерные алгебры.
y_j = x_j при j>n.
\end{document}
